\documentclass[12pt]{article}

\usepackage{algorithm}
\usepackage{algorithmic}
\usepackage{amsmath}
\usepackage{amssymb}
\usepackage[utf8]{inputenc}
\usepackage{listings}
\usepackage[spanish]{babel}
\usepackage{qtree}

% Para codigo bonito
\newcommand{\singlespace}{\renewcommand{\baselinestretch}{1.0}}

\newcommand{\algoritmo}[2]
{
{
  \singlespace
  \begin{algorithm}[h]
    \caption{#1}
    \begin{algorithmic}[1]
      #2
    \end{algorithmic}
  \end{algorithm}
}
}

\newcommand{\theHalgorithm}{\arabic{algorithm}}

\lstset{mathescape=true,frame=single}

\newcommand{\asig}{\ensuremath{\leftarrow}}
\renewcommand{\lstlistingname}{Listado} %listings en español

\begin{document}

% Título

\title{Tarea 1 de Bases de Datos Heterogéneas}
\author{Daniel Izquierdo \\ \#08-86809}
\date{30 de abril de 2009}

\maketitle

\begin{enumerate}

% Pregunta 1
 \item 

\begin{eqnarray*}
 T_1 & \asig & \sigma_{(\text{NombreBeb}=\text{``malta''})}(\text{BEBIDAS}) \\
 T_2 & \asig & \pi_{\text{CI}}(T_1 \text{ Join } \text{GUSTA}) \\
 T_3 & \asig & \pi_{\text{CI}}(\text{BEBEDOR}) - T_2
\end{eqnarray*}

% Pregunta 2
 \item 

% Pregunta 3
 \item 

% Pregunta x
 \item 

\end{enumerate}

\end{document}
