\documentclass[12pt]{article}

\usepackage{amsmath}
\usepackage{amssymb}
\usepackage[utf8]{inputenc}
\usepackage[spanish]{babel}

\newcommand{\asig}{\ensuremath{\leftarrow}}

\begin{document}

% Título

\title{Bases de Datos Heterogéneas \\ Tarea 1}
\author{Daniel Izquierdo \\ \#08-86809}
\date{30 de abril de 2009}

\maketitle

\begin{enumerate}

% Pregunta 1
 \item

\begin{eqnarray*}
 T_1 & \asig & \sigma_{(\text{NombreBeb}=\text{``malta''})}(\text{BEBIDAS}) \\
 T_2 & \asig & \pi_{\text{CI}}(T_1 \text{ Join } \text{GUSTA}) \\
 T_3 & \asig & \pi_{\text{CI}}(\text{BEBEDOR}) - T_2 \\
\end{eqnarray*}

% Pregunta 2
 \item

\begin{eqnarray*}
 T_1 & \asig & \sigma_{(\text{Nombre}=\text{``Luis Pérez''})}(\text{BEBEDOR}) \\
 T_2 & \asig & \pi_{\text{CodFS}}(T_1 \text{ Join } \text{FRECUENTA}) \\
 T_3 & \asig & \pi_{\text{CodFS}}(\text{FUENTES\_SODA}) - T_2 \\
\end{eqnarray*}

% Pregunta 3
 \item

\begin{eqnarray*}
 T_1 & \asig & \pi_{\text{CI}}(\text{GUSTA} \text{ Join } \text{FRECUENTA}) \\
\end{eqnarray*}

% Pregunta 4
 \item

\begin{eqnarray*}
 T_1 & \asig & \pi_{\text{CI,CodBeb}}(\text{BEBEDOR} \times \text{BEBIDAS}) \\
 T_2 & \asig & T_1 - \text{GUSTA} \\
\end{eqnarray*}

A partir de ahora supongo $\text{NO\_GUSTA} = T_2$.

% Pregunta 5
 \item

\begin{eqnarray*}
 T_1 & \asig & \pi_{\text{CI,CodFS}}(\text{NO\_GUSTA} \text{ Join } \text{VENDE}) \\
 T_2 & \asig & \pi_{\text{CI,CodFS}}(\text{BEBEDOR} \times \text{FUENTES\_SODA}) \\
 T_3 & \asig & T_2 - T_1 \\
\end{eqnarray*}

% Pregunta 6
 \item

\begin{eqnarray*}
 T_1 & \asig & \pi_{\text{CI,CodFS}}(\text{BEBEDOR} \times \text{FUENTES\_SODA}) \\
 \text{NO\_FREC} & \asig & T_1 - \text{FRECUENTA} \\ % NO_FRECUENTA
 T_2 & \asig & \pi_{\text{CodFS,CodBeb}}(\text{FUENTES\_SODA} \times \text{BEBIDAS}) \\
 \text{NO\_VENDE} & \asig & T_2 - \text{VENDE} \\ % NO_VENDE
 T_3 & \asig & \pi_{\text{CodFS,CI}}(\text{NO\_VENDE} \text{ Join } \text{GUSTA}) \\
 T_4 & \asig & T_1 - T_3 \\
 T_5 & \asig & \text{NO\_FREC} \text{ Join } T_4 \\
\end{eqnarray*}

% Pregunta 7
 \item

\begin{eqnarray*}
 T_1 & \asig & \sigma_{(\text{NombreBeb}=\text{``malta''})}(\text{BEBIDAS}) \\
 T_2 & \asig & \sigma_{(\text{NombreBeb}=\text{``frescolita''})}(\text{BEBIDAS}) \\
 T_3 & \asig & \sigma_{(\text{NombreBeb}=\text{``coca-cola''})}(\text{BEBIDAS}) \\
 T_4 & \asig & \pi_{\text{CI}}(T_1 \text{ Join } \text{GUSTA}) \\
 T_5 & \asig & \pi_{\text{CI}}(T_2 \text{ Join } \text{GUSTA}) \\
 T_6 & \asig & \pi_{\text{CI}}(T_3 \text{ Join } \text{GUSTA}) \\
 T_7 & \asig & (T_4 - T_5) - T_6 \\
\end{eqnarray*}

% Pregunta 8
 \item 

% Pregunta 9
 \item 

% Pregunta 10
 \item 

% Pregunta 11
 \item 

% Pregunta 12
 \item 

% Pregunta 13
 \item 

% Pregunta 14
 \item 

% Pregunta 15
 \item 

% Pregunta 16
 \item 

% Pregunta 17
 \item 

% Pregunta 18
 \item 

\end{enumerate}

\end{document}
