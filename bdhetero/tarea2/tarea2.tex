\documentclass[10pt]{article}

\usepackage{amsmath}
\usepackage{amssymb}
\usepackage[utf8]{inputenc}
\usepackage[spanish]{babel}
\usepackage{qtree}

\newcommand{\asig}{\ensuremath{\leftarrow}}

\begin{document}

% Título

\title{Bases de Datos Heterogéneas \\ Tarea 2}
\author{Daniel Izquierdo \\ \#08-86809}

\maketitle

\renewcommand{\labelenumi}{\alph{enumi}.}
\begin{enumerate}

% Pregunta a
 \item

\Tree
[.$\Pi_{\text{PELICULA.NOMBRE}}$
  [.{$\sigma$ \\
   PRODUCE.PRESUPUESTO $ = 1000000$ (C1) \\
   PRODUCE.AÑO $ = 2004$ {\bf OR} PRODUCE.AÑO $ = 2005$ (C2) \\
   COMPAÑIA.NOMBRE $ = $ ``MGM'' (C3) \\
   PELICULA.COD\_PELICULA $ = $ PRODUCE.COD\_PELICULA (C4) \\
   PRODUCE.COD\_COMPAÑIA $ = $ COMPAÑIA.COD\_COMPAÑIA (C5) \\
   }
    [.$\times$
      [.$\times$ PELICULA PRODUCE ]
      COMPAÑIA
    ]
  ]
]

% Pregunta b
\item

\renewcommand{\labelenumii}{\arabic{enumii}.}
\newcommand{\andsql}{\text{ {\bf AND} }}

Usaré conmutatividad y asociatividad de $\andsql$ y neutralidad de
$\mathtt{true}$ sobre $\andsql$ libremente.

\begin{enumerate}
 \item Cascada de selecciones:
       \begin{eqnarray*}
       & \sigma_{C1 \andsql C2 \andsql C3 \andsql C4 \andsql C5}(\ldots) \\
       \equiv & \\
       & \sigma_{C1 \andsql C2 \andsql C3 \andsql C4}(\sigma_{C5}(\ldots))
       \end{eqnarray*}

 \item Conversión del producto cartesiano en join:
       \begin{eqnarray*}
       & \sigma_{C5}((\text{PELICULA} \times \text{PRODUCE}) \times \text{COMPAÑIA}) \\
       \equiv & \\
       & (\text{PELICULA} \times \text{PRODUCE}) \Join_{C5} \text{COMPAÑIA} \\
       \end{eqnarray*}

\item Distributividad de la selección con respecto al join:
       \begin{eqnarray*}
       & \sigma_{(C1 \andsql C2 \andsql C4) \andsql C3}(\ldots \Join_{C5} \ldots)  \\
       \equiv & \\
       & \sigma_{C1 \andsql C2 \andsql C4}(\ldots) \Join_{C5} \sigma_{C3}(\text{COMPAÑIA})  \\
       \end{eqnarray*}

       Se puede verificar que se cumplen las condiciones sobre los atributos
       para aplicar esta equivalencia.

 \item Cascada de selecciones:
       \begin{eqnarray*}
       & \sigma_{(C1 \andsql C2) \andsql C4}(\ldots) \\
       \equiv & \\
       & \sigma_{C1 \andsql C2}(\sigma_{C4}(\ldots))
       \end{eqnarray*}

 \item Conversión del producto cartesiano en join:
       \begin{eqnarray*}
       & \sigma_{C4}(\text{PELICULA} \times \text{PRODUCE}) \\
       \equiv & \\
       & \text{PELICULA} \Join_{C4} \text{PRODUCE} \\
       \end{eqnarray*}

\item Distributividad de la selección con respecto al join:
       \begin{eqnarray*}
       & \sigma_{\mathtt{true} \andsql (C1 \andsql C2)}(\text{PELICULA} \Join_{C4} \text{PRODUCE})  \\
       \equiv & \\
       & PELICULA \Join_{C4} \sigma_{C1 \andsql C2}(\text{PRODUCE})  \\
       \end{eqnarray*}

       Se puede verificar que se cumplen las condiciones sobre los atributos
       para aplicar esta equivalencia.

 \item Cascada de selecciones:
       \begin{eqnarray*}
       & \sigma_{C1 \andsql C2}(\text{PRODUCE}) \\
       \equiv & \\
       & \sigma_{C1}(\sigma_{C2}(\text{PRODUCE}))
       \end{eqnarray*}

\end{enumerate}

% Pregunta c
 \item

\begin{itemize}

\item $\sigma_{C2}(\text{PRODUCE})$: 391 bloques, produce 2000 tuplas

\item $\sigma_{C1}(\sigma_{C2}(\text{PRODUCE}))$: 79 bloques, produce hasta 2000 tuplas

\item $PELICULA \Join_{C4} \sigma_{C1 \andsql C2}(\text{PRODUCE})$: \\
      (hash join) 2931+237=3168 bloques, produce hasta 2000 tuplas

\item $\sigma_{C4}(\text{COMPAÑIA})$: 25 bloques, produce hasta hasta 1 tupla

\item $(...) \Join_{C5} (...)$: (hash join) 235+3=238 bloques

\item Proyección: 238 bloques

\end{itemize}

El costo es $391+79+3168+25+238+238=4139$ accesos a bloques.

% Pregunta d
 \item

Hay que hacer proyección con eliminación de duplicados. Usando la técnica basada
en \emph{hashing}, el costo es 

$391+79+3168+25+238+(2N)/1024$ donde $N$ es el número de tuplas resultado.

% Pregunta e
 \item

% Pregunta f
 \item

% Pregunta g
 \item

% Pregunta h
 \item

% Pregunta i
 \item

% Pregunta j
 \item

% Pregunta k
 \item

\end{enumerate}

\end{document}
