\documentclass[10pt]{article}

\usepackage{amsmath}
\usepackage{amssymb}
\usepackage[utf8]{inputenc}
\usepackage[spanish]{babel}
\usepackage{qtree}

\newcommand{\asig}{\ensuremath{\leftarrow}}

\begin{document}

% Título

\title{Bases de Datos Heterogéneas \\ Tarea 2}
\author{Daniel Izquierdo \\ \#08-86809}

\maketitle

\renewcommand{\labelenumi}{\alph{enumi}.}
\begin{enumerate}

% Pregunta a
 \item

\Tree
[.$\Pi_{\text{PELICULA.NOMBRE}}$
  [.{$\sigma$ \\
   PRODUCE.PRESUPUESTO $ = 1000000$ (C1) \\
   PRODUCE.AÑO $ = 2004$ {\bf OR} PRODUCE.AÑO $ = 2005$ (C2) \\
   COMPAÑIA.NOMBRE $ = $ ``MGM'' (C3) \\
   PELICULA.COD\_PELICULA $ = $ PRODUCE.COD\_PELICULA (C4) \\
   PRODUCE.COD\_COMPAÑIA $ = $ COMPAÑIA.COD\_COMPAÑIA (C5) \\
   }
    [.$\times$
      [.$\times$ PELICULA PRODUCE ]
      COMPAÑIA
    ]
  ]
]

% Pregunta b
\item

\renewcommand{\labelenumii}{\arabic{enumii}.}
\newcommand{\andsql}{\text{ {\bf AND} }}

Usaré conmutatividad y asociatividad de $\andsql$ y neutralidad de
$\mathtt{true}$ sobre $\andsql$ libremente.

\begin{enumerate}
 \item Cascada de selecciones:
       \begin{eqnarray*}
       & \sigma_{C1 \andsql C2 \andsql C3 \andsql C4 \andsql C5}(\ldots) \\
       \equiv & \\
       & \sigma_{C1 \andsql C2 \andsql C3 \andsql C4}(\sigma_{C5}(\ldots))
       \end{eqnarray*}

 \item Conversión del producto cartesiano en join:
       \begin{eqnarray*}
       & \sigma_{C5}((\text{PELICULA} \times \text{PRODUCE}) \times \text{COMPAÑIA}) \\
       \equiv & \\
       & (\text{PELICULA} \times \text{PRODUCE}) \Join_{C5} \text{COMPAÑIA} \\
       \end{eqnarray*}

\item Distributividad de la selección con respecto al join:
       \begin{eqnarray*}
       & \sigma_{(C1 \andsql C2 \andsql C4) \andsql C3}(\ldots \Join_{C5} \ldots)  \\
       \equiv & \\
       & \sigma_{C1 \andsql C2 \andsql C4}(\ldots) \Join_{C5} \sigma_{C3}(\text{COMPAÑIA})  \\
       \end{eqnarray*}

       Se puede verificar que se cumplen las condiciones sobre los atributos
       para aplicar esta equivalencia.

 \item Cascada de selecciones:
       \begin{eqnarray*}
       & \sigma_{(C1 \andsql C2) \andsql C4}(\ldots) \\
       \equiv & \\
       & \sigma_{C1 \andsql C2}(\sigma_{C4}(\ldots))
       \end{eqnarray*}

 \item Conversión del producto cartesiano en join:
       \begin{eqnarray*}
       & \sigma_{C4}(\text{PELICULA} \times \text{PRODUCE}) \\
       \equiv & \\
       & \text{PELICULA} \Join_{C4} \text{PRODUCE} \\
       \end{eqnarray*}

\item Distributividad de la selección con respecto al join:
       \begin{eqnarray*}
       & \sigma_{\mathtt{true} \andsql (C1 \andsql C2)}(\text{PELICULA} \Join_{C4} \text{PRODUCE})  \\
       \equiv & \\
       & PELICULA \Join_{C4} \sigma_{C1 \andsql C2}(\text{PRODUCE})  \\
       \end{eqnarray*}

       Se puede verificar que se cumplen las condiciones sobre los atributos
       para aplicar esta equivalencia.

 \item Cascada de selecciones:
       \begin{eqnarray*}
       & \sigma_{C1 \andsql C2}(\text{PRODUCE}) \\
       \equiv & \\
       & \sigma_{C1}(\sigma_{C2}(\text{PRODUCE}))
       \end{eqnarray*}

\end{enumerate}

% Pregunta c
 \item

\begin{itemize}

\item $\sigma_{C2}(\text{PRODUCE})$: 391 bloques, produce 2000 tuplas

\item $\sigma_{C1}(\sigma_{C2}(\text{PRODUCE}))$: 79 bloques, produce hasta 2000 tuplas

\item $PELICULA \Join_{C4} \sigma_{C1 \andsql C2}(\text{PRODUCE})$: \\
      (hash join) 2931+237=3168 bloques, produce hasta 2000 tuplas

\item $\sigma_{C4}(\text{COMPAÑIA})$: 25 bloques, produce hasta hasta 1 tupla

\item $(...) \Join_{C5} (...)$: (hash join) 235+3=238 bloques

\item Proyección: 238 bloques

\end{itemize}

El costo es $391+79+3168+25+238+238=4139$ accesos a bloques.

% Pregunta d
 \item

Hay que hacer proyección con eliminación de duplicados. Usando la técnica basada
en \emph{hashing}, el costo es 

$391+79+3168+25+238+(2N)/1024$ donde $N$ es el número de tuplas resultado.

% Pregunta e
 \item

Para los operadores de selección no tenemos índices sobre los atributos de selección. Por esto
se implementan como un recorrido completo de la relación. Para el join más profundo no tenemos índice
sobre el atributo de join y tampoco cabe la relación izquierda en memoria principal, por lo que no podemos
usar Block Nested Loop Join ni Index Nested Loop Join. Entre los operadores restantes, el más barato
es Hash Join.

Para el join menos profundo, no podemos usar Block Nested Loop Join por falta de memoria principal, y tampoco
Index Nested Loop Join por ser una operación sobre resultados intermedios. Entonces el operador más barato
es también Hash Join.

% Pregunta f
 \item

\begin{itemize}
 \item Tamaño 1: PELICULA: 977 bloques. Selección sobre PRODUCE: 470 bloques. Selección sobre COMPAÑIA: 25 bloques.
       Se elige selección sobre COMPAÑIA. Costo hasta ahora: $25$ bloques, y tenemos una tupla de COMPAÑIA.
 \item Tamaño 2: Sólo se puede hacer join con PRODUCE. Se hace join de la selección de COMPAÑIA con la selección de PRODUCE.
       Costo hasta ahora: $25+3*1+3*470=25+1413$ bloques, y tenemos $4$ tuplas del join (por las estadísticas).
 \item Tamaño 3: Sólo queda PELICULA. Podemos usar Block Nested Loop Join. Costo hasta ahora: $25+1413+1+1*2931=4370$ bloques.
\end{itemize}

% Pregunta g
 \item

Criterio de equivalencia: incluye las mismas relaciones.

\begin{itemize}
 \item Iteración 1: PELICULA: 977 bloques. Selección sobre PRODUCE: 470 bloques. Selección sobre COMPAÑIA: 25 bloques.
 \item Iteración 2: Hash join entre PELICULA y PRODUCE: 4341 bloques. Hash join entre PRODUCE y PELICULA: 4341 bloques.
       Hash join entre PRODUCE y COMPAÑIA: 1413 bloques. Hash join entre COMPAÑIA y PRODUCE: 1413 bloques. Block Nested Loop Join
       entre COMPAÑIA y PRODUCE: 471 bloques.
 \item Iteración 3: Hash join entre PELICULA y (Block Nested Loop Join entre COMPAÑIA y PRODUCE): 4344 bloques. Hash join entre
       COMPAÑIA y (Hash join entre PRODUCE y PELICULA): 13026 bloques.
\end{itemize}

Resultado: Hash join entre PELICULA y (Block Nested Loop Join entre COMPAÑIA y PRODUCE), costo 4344 bloques.

% Pregunta h
 \item

El algoritmo greedy es un caso particular de programación dinámica donde todos los
planes son equivalentes. Como se vió en la corrida anterior, no se cumple el principio
de optimalidad porque no llegamos a la solución óptima.

% Pregunta i
 \item

% Pregunta j
 \item

Sea $g_1 = 2^n$ y $g_k = 3g_{k-1}(g_{k-1}-1), k > 1$. Entonces una cota superior es

$$
\sum_{k=1}^n g_k
$$

% Pregunta k
 \item

Sea $g_1 = 2^n$ y $g_k = 6g_{k-1}(g_{k-1}-1), k > 1$. Entonces una cota superior es

$$
\sum_{k=1}^n g_k
$$

\end{enumerate}

\end{document}
