\documentclass[12pt]{article}

\usepackage{amsmath}
\usepackage{amssymb}
\usepackage[utf8]{inputenc}
\usepackage[spanish]{babel}

\newcommand{\asig}{\ensuremath{\leftarrow}}

\begin{document}

% Título

\title{Bases de Datos Heterogéneas \\ Proyecto 1 \\ Informe}
\author{Daniel Izquierdo \\ \#08-86809}
\date{28 de mayo de 2009}

\maketitle

\section{Introducción}

Una consulta sobre una base de datos generalmente puede ser evaluada o
implementada de varias maneras diferentes. Cada manera de evaluarla implica
costos de ejecución distintos. Puede haber grandes diferencias en los costos de
distintos planes de evaluación para la misma consulta, y puede haber un número
inmenso de planes distintos. Para cada consulta se desearía conocer su mejor
plan de evaluación en términos de recursos a consumir, pero esto usualmente no
es factible. Se intenta entonces encontrar un plan relativamente bueno, cuyo
costo de ejecución se aleje del peor costo posible.

Un optimizador es un programa que toma una consulta y produce el plan de
evaluación de menor costo o uno de costo relativamente bajo.
En este proyecto se diseñarán e implementarán optimizadores para ciertos tipos
de consultas. Se está considerando un subconjunto SQL' de las consultas
expresables en el lenguaje SQL. Los optimizadores usarán las meta-heurísticas de
programación dinámica y \emph{Simulated Annealing}.

\section{Diseño de la solución}

\section{Almacenamiento del catálogo}

\section{Estudio experimental}

\section{Conclusión}

\end{document}
