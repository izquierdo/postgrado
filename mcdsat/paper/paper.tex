\documentclass{sig-alternate}

\usepackage{enumerate}

\makeatletter
\let\@copyrightspace\relax
\makeatother

\def\sharedaffiliation{%
\end{tabular}
\begin{tabular}{c}}

\newcommand{\denselist}{\topsep 0pt \itemsep -4pt}
\newcommand{\tup}[1]{\langle #1 \rangle}
\newcommand{\vvec}[1]{\mathbf{#1}}
\newcommand{\join}{\bowtie}
\newcommand{\R}{\mathcal{R}}
\newcommand{\Q}{\mathcal{Q}}
\newcommand{\body}{{body}}
\newcommand{\head}{{head}}
\newcommand{\qrule}{:\!\!-}
\newcommand{\arc}{\text{arc}}

\begin{document}
\allowdisplaybreaks

\title{Computing Minimum-Cost Rewritings Using Circuits}
\subtitle{[Extended Abstract]}
\numberofauthors{3}
\author{
\alignauthor Daniel Izquierdo \email{idaniel@ldc.usb.ve}
\alignauthor Mar\'{\i}a Esther Vidal \email{mvidal@ldc.usb.ve}
\alignauthor Blai Bonet \email{bonet@ldc.usb.ve}
\sharedaffiliation
\affaddr{Departamento de Computaci\'on} \\
\affaddr{Universidad Sim\'on Bol\'{\i}var} \\
\affaddr{Caracas, Venezuela}
}
\maketitle

\newtheorem{theorem}{Theorem}

\begin{abstract}
We show how the logical translation used in the McdSat algorithm
can be used as well to compute the minimum-cost rewritings with
respect to a simple additive cost model. Our algorithm exploits
a known relation between d-DNNF theories and arithmetic circuits.
\end{abstract}

\section{Introduction}

Recently, we considered the problem of computing all the rewritings
of a query in terms of materialized views \cite{halevy:survey}.
This problem is important in the context of data integration
\cite{duschka:plan,kwok:plan,lambrecht:plan},
and query optimization and data maintenance
\cite{levy:bucket,zaharioudakis:table,mitra:svbt}.
Our approach consisted in translating the problem of rewriting
into the problem of enumerating the models of a propositional
theory \cite{arvelo:aaai06} which is compiled into deterministic
and decomposable negation normal form (d-DNNF) in order to
enumerate the models efficiently \cite{darwiche:dnnf}.

The theory is a CNF formula whose models are in correspondence
with the rewritings of the query. This theory is constructed
following the MiniCon Descriptions (MCDs) and their possible compositions
\cite{pottinger:minicon}.
The logical approach demonstrated to scale better than the MiniCon
algorithm over a large number of benchmarks often showing performance
improvements of several orders of magnitude.

However, the McdSat algorithm, the name of the logical approach,
was not desgined for rewriting problems involving explicit constants,
nor to compute the best rewritings with respect to a given cost model.
In this work, we address these shortcommings by provinding an
extended logical formulation capable of handling the constants,
and by showing how the d-DNNF can be exploited in order to compute
the best rewritings for a simple yet interesting cost models.

The paper is organized as follows.
We define the problem, the translation, and the cost models (Sect.~2).
We show how the theory can be extended to support explicit constants (Sect.~3)
and how the d-DNNF is exploited to find best rewritings (Sect.~4).
We end with some experiments (Sect.~5) and a discussion (Sect.~6).

\section{Problem, Theories and Costs}

We consider databases of the form $D=\tup{P,T}$ where
$P$ be a set of predicates and $T=\{T_p\}_{p\in P}$ is a collection
of tables that represents the predicates in extensional form.
A conjunctive query $Q$ over $P$ is of the form 
\begin{alignat*}{1}
Q(\vvec{x})\ \qrule\ \  p_1(\vvec{x}_1),\, p_1(\vvec{x}_2),\, \ldots,\, p_m(\vvec{x}_m)\,,
\end{alignat*}
where $p_i\in P$, $\vvec{x}$ is a vector of variables, and each
$\vvec{x}_i$ is a vector of variables and constants.
The result of $Q$ over $D$, denoted as $Q(D)$, is the table with
$|\vvec{x}|$ columns that result of the projection of the relational
join $\join\!\!\{T_{p_i}\}_{i=1}^m$ over $\vvec{x}$.
The atoms in the body of $Q$ are called the (sub)goals of $Q$.

A view $V$ over $D$ is a query over $P$. In the context of data
integration, the database $D$ is an idealized description of 
the information contained across multiple data sources described
as views.
Given a database $D$, a query $Q$ and a collection of 
views $E=\tup{\{V_i\}_i,\{E_i\}_i}$, we are required to find
all the tuples in $Q(D)$ obtainable from the views in $E$.
That is, we need to find all the \emph{rewritings} of the form
\begin{alignat*}{1}
R(\vvec{x})\ \qrule\ \ V_{i_1}(\vvec{x}_1),\, V_{i_2}(\vvec{x}_2),\, \ldots,\, V_{i_n}(\vvec{x}_n)
\end{alignat*}
such that $R(E) \subseteq Q(D)$.
A query rewriting problem (QRP) is a tuple $\tup{P,Q,\{V_i\}}$ where $P$
is a set of predicates, $Q$ is a query over $P$ and $\{V_i\}$ is a collection
of views. We assume \emph{safe} problems in the sense that all 
variables mentioned in the head of the query (resp.\ in the head of each view)
appear in the body of the query (resp.\ in the body of each view).
Further, we only deal with QRPs with no arithmetic predicates inside the
query or views.
A rewriting $R$ is \emph{valid} if for all databases $D=\tup{P,T}$
and extensions $\{E_i\}$, $R(E) \subseteq Q(D)$.
A collection $\R$ of valid rewritings is a solution if
for all databases $D=\tup{P,T}$ and extensions $\{E_i\}$, there
is no other $\R'$ such that $\R(E)\subset\R'(E)\subseteq Q(D)$.
We are interested in finding a rewriting $\R$.

\subsection{Logical Theories}

In \cite{arvelo:aaai06}, we showed that a rewriting
for a query $Q$ with $m$ goals can be obtained by enumerating the
models of a logical theory
$\Delta=\Delta_{rew}\cup\Delta_{mcd}^1\cup\cdots\Delta_{mcd}^m$
where $\Delta_{rew}$ specified how to combine $m$ independent
copies of MCD theories $\Delta_{mcd}$ that cover all goals in $Q$.
Each $\Delta^i_{mcd}$ is a copy of the theory $\Delta_{mcd}$
in which each literal $p$ is tagged as $p^i$.
The theory $\Delta_{mcd}$ consists of different groups of 
clauses that guarantess that its models are in correspondence
with the MCDs, while the theory $\Delta_{rew}$ contains additional
clauses to guarantee a sound and complete composition of the MCDs.
The reader is referred to \cite{arvelo:aaai06} for a comprehensive
description of the propositional theory.

Let us describe the difficulties that arise when constants
are presents in a QRP with the following example:
\begin{alignat*}{1}
Q(x,z)\ &\qrule\ \ p_1(x,y),\, p_2(y,z)\,, \\
V_1(x_1)\ &\qrule\ \ p_1(x_1,A)\,, \\
V_2(x_2)\ &\qrule\ \ p_2(B,x_2)\,, \\
V_3(x_3,y_3)\ &\qrule\ \ p_2(x_3,y_3)
%V_4(x_4,y_4)\ &\qrule\ \ p_1(x_1,A),\, p_2(B,y_4)\,,
\end{alignat*}
where $Q$ is the query and $\{V_1,V_2,V_3\}$ are the 
views. In this case, the only rewriting is 
$R(x,z)\qrule V_1(x),V_3(A,z)$ because the candidate 
$R(x,z)\qrule V_1(x),V_2(z)$ is not valid as it maps
$y$ into constants $A$ and $B$ that denote different
objects.

The main problem when handling constants is to
be sure that different constants are not mapped into
each other either directly or indirectly (via transitivity).

\subsection{Cost Models}

We assume a simple additive cost model in which each view
$V_i$ is associated with a cost $c(V_i)$, and the cost of
a rewriting is the sum of the costs for the views in it.
An optimal or best rewriting is one with minimum cost,
and the optimal cost of a QRP is the cost of a best
rewriting. A QRP has always a well-defined optimal
cost (if there are no rewritings, its cost is $\infty$),
but it may have multiple best rewritings.
The rewriting problem with costs consists in finding
all the rewritings of minimum cost.

%The cost model is simple yet expressive; e.g., if all
%costs are unit, then best rewritings are those with
%minimum number of views, yet in cases where views
%represent different sources, it is natural
%to associate them with different costs.


\section{Extended Theories}

The theory $\Delta_{mcd}$ makes use of propositions $t_{x,y}$
to denote that the variable/constant $x$ in the query is mapped
into the variable/constant $y$ in the view, and propositions $v_i$
to indicate that the MCD uses view $V_i$.
We obtain a sound and complete theory for QRPs with constants if
$\Delta_{mcd}$ is extended with the clauses:
\begin{enumerate}[C1.]\denselist
\item (Inconsistent-1): $t_{x,A} \Rightarrow \neg t_{x,B}$,
\item (Inconsistent-2): $t_{A,x} \Rightarrow \neg t_{B,x}$,
\item (Inconsistent-3): $\neg t_{A,B}$,
\item (Transitivity-1): $v_i\land t_{A,y}\land t_{x,y}\land t_{x,z}\Rightarrow t_{A,z}$,
\item (Transitivity-2): $v_i\land t_{y,A}\land t_{y,x}\land t_{z,x}\Rightarrow t_{z,A}$.
\end{enumerate}
Clauses C1--C3 prune candidate rewritings in which one constant
is directly mapped into a different one, and the last two implement
a restricted propagation of transitivity.
Similarly, $\Delta_{rew}$ must be extended with the clauses:
\begin{enumerate}[C1.]\denselist
\item[C6.] (Inconsistent-4): $t^i_{x,A} \Rightarrow \neg t^j_{x,B}$.
\end{enumerate}
XXXX showed that when the query has $m$ goals and there
are at most $n$ dfferent constant symbols, it is enough
to consider rewritings of length at most XXXXXX \cite{ref}.

Given a QRP problem $\tup{Q,\{V_i\}}$ possibly with constant 
symbols, we construct logical theories whose models are in
correspondence with the rewritings. These theories are then
compiled into d-DNNF from which all rewritings are extracted
efficiently.
Likewise, best rewritings can be computed by performing model
enumeration twice: the first pass computes the optimal cost,
and the second filters out suboptimal rewritings.
However, there is a better way.

\section{Minimum-Cost Models}

Darwiche \& Marquis \cite{darwiche:weighted} show how to compute the
rank $r^*(\Delta)$ of a propositional theory $\Delta$
efficiently when $r$ is a literal ranking function and
$\Delta$ is in d-DNNF. A literal ranking function ranks
the models of the theory in terms of the rank of the
literals $l$ that are true in the model, and ranks the
theory as the best rank of its models:
\[ r(\omega) = \sum_{\omega\vDash l} r(l),\quad r^*(\Delta) = \min_{\omega\vDash\Delta} r(\omega)\,. \]
Furthermore, not only $r^*(\Delta)$ can be computed
efficiently from the d-DNNF but also the models that
have such rank.
The procedure for computing the rank converts the d-DNNF
into a circuit in which the literals are replaced by their
ranks, the `or' nodes by `min', and the `and' nodes by `sum'.
The evaluation of the circuit computes the rank of $\Delta$.

We thus obtain a simple method for computing the best
rewritings when the literal ranking function $r_c$ induced
by the cost model $c$ is used; $r_c$ is defined by
$r_c(l)=c(V_i)$ if $l=v^t_i$ for some $t$, and $r_c(l)=0$
otherwise.

\section{Preliminary Experiments}

[brief explanation of results; no tables, no graphs; leave them for the poster]

\section{Discussion}

We have shown how the propositional theory used in McdSat can
be extended to support constants, and how it can be used to 
compute the best rewritings with respect to an additive
cost model. The cost model is simple yet expressive.
The computation of the best models relies on an arithmetic
circuit that is obtained from the d-DNNF of the propositional
theory.

\small
\bibliography{../../control}
\bibliographystyle{plain}

\end{document}

