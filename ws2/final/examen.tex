\documentclass[12pt]{article}

\usepackage[spanish]{babel}
\usepackage[utf8x]{inputenc}

\begin{document}

% Título

\author{Daniel Izquierdo \\ \#08-86809}
\title{Examen Final \\ Web Semántica 2}
\maketitle

% Contenido

Con el rápido crecimiento que actualmente experimenta el ``Cloud of Linked
Data'' surgen muchas nuevas aplicaciones que involucran la recuperación de datos
de varias fuentes distintas, cada una con su propia estructura. Usuarios y
aplicaciones necesitan cada vez más tener la capacidad de evaluar muchas consultas
contra fuentes de datos heterogéneas. Las fuentes de datos se pueden
caracterizar por tener estructuras distintas y contener resultados parciales de
la consulta que se realiza. Además, escogiendo un criterio de calidad para las
fuentes, resulta que todas pueden tener valores distintos para los parámetros de
calidad. Para evaluar una consulta se deben conocer sus posibles reescrituras en
términos de las fuentes disponibles.

Por otra parte también existe la necesidad de componer servicios web
descubiertos para implementar realizar flujos de trabajo. Un usuario puede
definir un flujo de trabajo como una composición de servicios abstractos que son
implementados por servicios concretos. Al hacerlo, requiere conocer las maneras
de implementar los servicios abstractos con los concretos existentes. También
existen criterios de calidad que se deben tomar en cuenta a la hora de elegir
una implementación dada.

Precisamente por la explosión de datos y servicios que se vive en los momentos,
en ambos casos el espacio de soluciones a la pregunta de cuáles son las
reescrituras buscadas puede llegar a ser muy grande. Se requiere entonces de un
sistema que pueda resolver el problema de manera eficiente y escalable. El
proyecto que estoy desarrollando es un intento de crearlo.

La capacidad de escalar a números realistas de vistas (fuentes de datos o
servicios concretos) y sub-objetivos (pasos del flujo de trabajo) al momento de
reescribir las consultas dadas ya fue lograda por \ref{yarvelo}. Mi contribución
ha sido añadir la capacidad a esta solución de tomar en cuenta constantes en las
consultas y vistas, así como de manejar el concepto de rendimiento para
seleccionar las mejores reescrituras según los parámetros de calidad en un
problema dado.

%En base a la tendencia actual que existe en la Web Semantica de publicar grandes datasets en el "Cloud of Linked Data", identifique las bondades y limitaciones del proyecto que esta desarrollando que permitan o limiten la manipulacion de este tipo de datasets.

%Su respuesta debe estar documentada con referencias bibliograficas actuales, resultados experimentales y analisis del comportamiento observado durante el proceso de experimentacion. El documento debe tener al menos dos paginas o 2000 palabras.

\end{document}
