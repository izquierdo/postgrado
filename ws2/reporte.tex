\documentclass{article}

\usepackage[utf8]{inputenc}
\usepackage[spanish]{babel}

\newcommand{\tablaTripletas}[2]
{
\begin{tabular}{l|c}
 & Número de tripletas \\
 \hline
 Antes & #1 \\
 Después & #2 \\
\end{tabular}
}

\newcommand{\tablaPredicados}[1]
{
\begin{tabular}{l|c|c}
Predicado & Antes & Después \\
\hline
#1
\end{tabular}
}

\newcommand{\tablaInstancias}[1]
{
\begin{tabular}{l|c|c|c}
Clase & Antes & Después \\
\hline
#1
\end{tabular}
}

\newcommand{\resultadoDataSet}[5]
{
  \newpage

  \section{#1}

  \vspace{0.5cm}

  \tablaTripletas{#2}{#3}

  \vspace{1cm}

  \tablaPredicados{#4}

  \vspace{1cm}

  \tablaInstancias{#5}
}

\begin{document}

\title{Aumento de tripletas RDF con axiomatización}
\date{}
\author{Daniel Izquierdo}

\maketitle

% DESCRIPCIÓN

Con respecto al número de tripletas, con el pequeño conjunto de casos se observa
aumento lineal de tripletas generadas por cantidad de tripletas preexistentes.
El número de tripletas se multiplica aproximadamente por diez.

Pasando al aumento de cláusulas por predicado usado podemos ver que los
predicados que más contribuyen son los definidos por RDFS, particularmente
{\tt rdfs:type}. Esto no sorprende porque la mayoría de los axiomas introducen
cláusulas con este predicado. Los predicados definidos por el usuario registran
aumento considerable pero no tan alto comparativamente. Algo importante es la
introducción de muchas tripletas con predicados que no aparecían previamente.

Viendo las tripletas por instancias también se nota que crecen más las tripletas
relacionadas con clases de RDFS. El mayor incremento, por mucho, es el de {\tt
rdfs:Resource}. Podemos observar que se introducen varias instancias de clases
que previamente no estaban instanciadas.


% RESULTADOS

\resultadoDataSet{rdfPrimero}{89}{879}
{
    cursa & 14 & 70 \\
    dicta & 16 & 40 \\
    esProfesor & 0 & 27 \\
    esSupervisor & 9 & 45 \\
    esTutor & 0 & 36 \\
    rdf:type & 40 & 507 \\
    rdfs:subClassOf & 8 & 111 \\
    rdfs:subPropertyOf & 2 & 43 \\
}
{
    estudiante & 0 & 32 \\
    estudiantePostGrado & 4 & 20 \\
    estudiantePostGradoOcasional & 4 & 20 \\
    estudiantePreGrado & 4 & 20 \\
    materia & 0 & 32 \\
    materiaPostGrado & 4 & 20 \\
    materiaPreGrado & 4 & 20 \\
    persona & 0 & 40 \\
    profesor & 4 & 20 \\
    rdfs:Class & 11 & 66 \\
    rdfs:Property & 5 & 37 \\
    rdfs:Resource & 0 & 180 \\
}

\resultadoDataSet{rdfSegundo}{120}{1193}
{
    cursa & 14 & 70 \\
    dicta & 16 & 40 \\
    esProfesor & 0 & 54 \\
    esSupervisor & 18 & 90 \\
    esTutor & 0 & 72 \\
    rdf:type & 62 & 713 \\
    rdfs:subClassOf & 8 & 111 \\
    rdfs:subPropertyOf & 2 & 43 \\
}
{
    estudiante & 0 & 32 \\
    estudiantePostGrado & 4 & 20 \\
    estudiantePostGradoOcasional & 24 & 120 \\
    estudiantePreGrado & 4 & 20 \\
    materia & 0 & 40 \\
    materiaPostGrado & 5 & 25 \\
    materiaPreGrado & 5 & 25 \\
    persona & 0 & 40 \\
    profesor & 4 & 20 \\
    rdfs:Class & 11 & 66 \\
    rdfs:Property & 5 & 37 \\
    rdfs:Resource & 0 & 268 \\
}

\resultadoDataSet{rdfTercero}{162}{1489}
{
    cursa & 14 & 70 \\
    dicta & 16 & 40 \\
    esProfesor & 0 & 108 \\
    esSupervisor & 36 & 180 \\
    esTutor & 0 & 144 \\
    rdf:type & 86 & 793 \\
    rdfs:subClassOf & 8 & 111 \\
    rdfs:subPropertyOf & 2 & 43 \\
}
{
    estudiante & 0 & 32 \\
    estudiantePostGrado & 4 & 20 \\
    estudiantePostGradoOcasional & 48 & 160 \\
    estudiantePreGrado & 4 & 20 \\
    materia & 0 & 40 \\
    materiaPostGrado & 5 & 25 \\
    materiaPreGrado & 5 & 25 \\
    persona & 0 & 40 \\
    profesor & 4 & 20 \\
    rdfs:Class & 11 & 66 \\
    rdfs:Property & 5 & 37 \\
    rdfs:Resource & 0 & 308 \\
}

\end{document}
