\documentclass{article}

\usepackage{algorithm}
\usepackage{algorithmic}
\usepackage{amsmath}
\usepackage{amssymb}
\usepackage[utf8]{inputenc}
\usepackage{listings}
\usepackage[spanish]{babel}
\usepackage{qtree}

% Para codigo bonito
\newcommand{\singlespace}{\renewcommand{\baselinestretch}{1.0}}

\newcommand{\algoritmo}[2]
{
{
  \singlespace
  \begin{algorithm}[h]
    \caption{#1}
    \begin{algorithmic}[1]
      #2
    \end{algorithmic}
  \end{algorithm}
}
}

\newcommand{\theHalgorithm}{\arabic{algorithm}}

\lstset{mathescape=true,frame=single}

\newcommand{\asig}{\ensuremath{\leftarrow}}
\renewcommand{\lstlistingname}{Listado} %listings en español

\begin{document}

% Título

\title{Tarea 1 de Matemáticas para Computación}
\author{Daniel Izquierdo \\ \#08-86809}
\date{30 de abril de 2009}

\maketitle

% Pregunta 1
\section{}

Para empezar, cualquier elemento $a$ del CPO es cota superior del conjunto
vacío, ya que se satisface trivialmente que
$(\forall b : b \in \emptyset : b \leq a)$
porque $b \in \emptyset \equiv \mathbf{false}$.
Supongamos que existe $s$, el supremo del conjunto vacío. Por definición,
para toda cota superior $c$ de $\emptyset$, $s \leq c$. Pero como todos los
elementos del CPO son cota superior, entonces para cada elemento $x$ del CPO
tenemos
$s \leq x$. Entonces $s$ es el menor elemento del CPO.

% Pregunta 2
\section{}

% Pregunta 3
\section{}

% Pregunta 4
\section{}

% Pregunta 5
\section{}

% Pregunta 6
\section{}

\end{document}
