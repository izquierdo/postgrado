\documentclass{article}

\usepackage{algorithm}
\usepackage{algorithmic}
\usepackage{amsmath}
\usepackage{amssymb}
\usepackage[utf8]{inputenc}
\usepackage{listings}
\usepackage[spanish]{babel}
\usepackage{qtree}

% Para codigo bonito
\newcommand{\singlespace}{\renewcommand{\baselinestretch}{1.0}}

\newcommand{\algoritmo}[2]
{
{
  \singlespace
  \begin{algorithm}[h]
    \caption{#1}
    \begin{algorithmic}[1]
      #2
    \end{algorithmic}
  \end{algorithm}
}
}

\newcommand{\theHalgorithm}{\arabic{algorithm}}

\lstset{mathescape=true,frame=single}

\newcommand{\asig}{\ensuremath{\leftarrow}}
\renewcommand{\lstlistingname}{Listado} %listings en español

\begin{document}

% Título

\title{Tarea 1 de Matemáticas para Computación}
\author{Daniel Izquierdo \\ \#08-86809}
\date{30 de abril de 2009}

\maketitle

% Pregunta 1
\section{}

Para empezar, cualquier elemento $a$ del CPO es cota superior del conjunto
vacío, ya que se satisface trivialmente que
$(\forall b : b \in \emptyset : b \leq a)$
porque $b \in \emptyset \equiv \mathbf{false}$.
Supongamos que existe $s$, el supremo del conjunto vacío. Por definición,
para toda cota superior $c$ de $\emptyset$, $s \leq c$. Pero como todos los
elementos del CPO son cota superior, entonces para cada elemento $x$ del CPO
tenemos
$s \leq x$. Entonces $s$ es el menor elemento del CPO.

% Pregunta 2
\section{}

Sean $a, b$ dos letras distintas de $V$ tales que $a < b$. Existe al menos la
cadena infinitamente decreciente

$$
ab > aab > aaab > aaaab > \ldots
$$

Por lo tanto, $(V^*,\preceq)$ no es Noetheriano.

% Pregunta 3
\section{}

% Pregunta 4
\section{}

% 4.1
\subsection{}

Llamaré $P$ al conjunto deseado. Supongo que $0$ es par.

\begin{itemize}
\item Base: $0 \in P$.

\item Si $x,y$ son palabras tales que
$xy \in P \wedge x \neq \epsilon$, entonces $x0y \in P$.

\item Si $x,y$ son palabras tales que
$xy \in P \wedge y \neq \epsilon$, entonces $x1y \in P$.
\end{itemize}

Un número es par si y sólo si su representación binaria termina en cero.
Tenemos como caso base que $0$ representa un número par sin ceros inútiles.
Supongamos que $x,y$ son palabras tales que $xy \in P \wedge x \neq \epsilon$.
Entonces $x0y$ es par porque si $y = \epsilon$ entonces $x0y = x0$ y, si no,
$y$ obligatoriamente termina en cero. $x0y$ no tiene ceros inútiles porque $x$
no puede tenerlos. Por otra parte, si $x, y$ son palabras tales que
$xy \in P \wedge y \neq \epsilon$, entonces $x1y$ es par porque $y$
obligatoriamente lo es. Además, si $x = \epsilon$ entonces $x1y$ = $1y$ y
entonces no tiene ceros inútiles. En caso de que $x \neq \epsilon$, $x$ no puede
tener ceros inútiles y por lo tanto $x1y$ tampoco los tiene. Por todo esto, el
esquema sólo construye representaciones de números pares sin ceros inútiles.

El número más pequeño que estamos considerando es $0$, representado por el
caso base. Sea $w$ la representación de un número par sin ceros inútiles de
longitud $n > 1$ y supongamos que el esquema puede generar todas las
representaciones con esa propiedad de longitud $m < n$. Entonces el esquema
también puede generar $w$: si $w = x1y$ para algún $x, y$, $y$ no puede ser
vacío y $xy$ también es par sin ceros inútiles, de menor longitud que $w$. Si
$w = x0y$, $x$ no puede ser vacío y $xy$ también es par sin ceros inútiles, de
menor longitud que $w$.
Por todo esto, el esquema puede construir todas las representaciones de números
pares sin ceros inútiles.

% 4.2
\subsection{}

Llamaré $P$ al conjunto deseado.

\begin{itemize}
\item Base: $\epsilon \in P$.
\item Base: $0 \in P$.
\item Base: $1 \in P$.
\item Si $w \in P$, entonces $0w0 \in P$.
\item Si $w \in P$, entonces $1w1 \in P$.
\end{itemize}

$\epsilon, 0, 1$ son palíndromes. Si $w$ es palíndrome, es claro que $0w0$ y
$1w1$ también lo son. Entonces el esquema sólo genera palíndromes. Además, se ve
que el esquema genera todos los palíndromes de longitud menor o igual a $1$.
Supongamos que $w$ es palíndrome de longitud $n > 1$ y el esquema genera todas
los palíndromes de longitud $m < n$. Tenemos $w = xpx$ donde $p$ es
una palabra de longitud menor que $n$ y $x = 0 \vee x = 1$.
Como el esquema puede generar $p$, es claro que el esquema también puede generar
$w$.

% Pregunta 5
\section{}

Primero demostraré que $(\forall e \in E :: P_0(e))$ con
$P_0(e) \equiv |e|_( = |e|_)$. Sea $e \in E$. El
caso base es $e \in V$. Si es así, entonces $|e|_( = 0 = |e|_)$. Si no,
supongamos que todas las expresiones de longitud menor que la de $e$ cumplen
$P_0$. Hay dos posibilidades:

\begin{itemize}
\item Existe $d \in E$ tal que $e = d^*$. Supusimos que $|d|_( = |d|_)$.
Entonces

$$|e|_( = |d^*|_( = |d|_( = |d|_) = |d^*|_) = |e|_)$$

\item Existen $f,g \in E$ tales que $e = (f+g)$. Supusimos que
$|f|_( = |f|_)$ y $|g|_( = |g|_)$. Entonces

$$
|e|_( = |(f+g)|_( = 2 + |f|_( + |g|_( = 2 + |f|_) + |g|_) = |(f+g)|_) = |e|_)
$$

\end{itemize}

Esto termina la primera parte de la demostración. Ahora falta ver que
$(\forall e \in E :: P_1(e))$
con
$P_1(e) \equiv p \text{ es prefijo de } e \Rightarrow |p|_( \geq |p|_)$.
Es claro que para prefijos vacíos se cumple $P_1$.
El caso base es $e \in V$. Si es así, entonces el único posible prefijo $p$ de
$e$ es $e$ y tenemos $|p|_( = |e|_( = 0 = |e|_) = |p|_)$. Para cualquier otro
caso, supongamos que todas las expresiones de longitud menor que la de $e$
cumplen $P_1$. Hay dos posibilidades:

\begin{itemize}

\item Existe $d \in E$ tal que $e = d^*$. Los únicos prefijos posibles de $e$
son $d^*$ y $d$. Tenemos

$$|d^*|_( = |d|_( = |d|_) = |d^*|_)$$

y

$$|d|_( = |d|_)$$

\item Existen $f,g \in E$ tales que $e = (f+g)$. Hay dos posibilidades para un
prefijo de $e$: incluye el último paréntesis, o no. En el primer caso tenemos

$$
|(f+g)|_( = |(f+g)|_)
$$

por $(f+g) \in E$ junto con la demostración anterior. En el segundo caso podemos
tener

\begin{eqnarray*}
& & |(f+g|_( = 1 + |f|_( + |g|_( = 1 + |f|_) + |g|_) > |f|_) + |g|_) = |(f+g|_) \\
& & |(f+|_( = 1 + |f|_( = 1 + |f|_) > |f|_) = |(f+|_) \\
& & |(f|_( = 1 + |f|_( = 1 + |f|_) > |f|_) = |(f|_) \\
& & |(|_( = 1 > 0 = |(f|_) \\
\end{eqnarray*}

Con esto queda completa la demostración. Resulta

$$(\forall e \in E :: P_0(e) \wedge P_1(e))$$

\end{itemize}

% Pregunta 6
\section{}

\end{document}
