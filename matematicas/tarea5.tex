\documentclass{article}

\usepackage{algorithm}
\usepackage{algorithmic}
\usepackage{amsmath}
\usepackage{amssymb}
\usepackage[utf8]{inputenc}
\usepackage{listings}
\usepackage[spanish]{babel}
\usepackage{qtree}
\usepackage{enumerate}

% Para codigo bonito
\newcommand{\singlespace}{\renewcommand{\baselinestretch}{1.0}}
\newcommand{\piso}[1]{\left \lfloor #1 \right \rfloor}
\newcommand{\techo}[1]{\left \lceil #1 \right \rceil}
\newcommand{\stirlingU}[2]{\genfrac{[}{]}{0pt}{}{#1}{#2}}
\newcommand{\stirlingD}[2]{\genfrac{\{}{\}}{0pt}{}{#1}{#2}}

\newcommand{\algoritmo}[2]
{
{
  \singlespace
  \begin{algorithm}[h]
    \caption{#1}
    \begin{algorithmic}[1]
      #2
    \end{algorithmic}
  \end{algorithm}
}
}

\newcommand{\theHalgorithm}{\arabic{algorithm}}

\lstset{mathescape=true,frame=single}

\newcommand{\asig}{\ensuremath{\leftarrow}}
\renewcommand{\lstlistingname}{Listado} %listings en español

\begin{document}

% Título

\title{Tarea 5 de Matemáticas para Computación}
\author{Daniel Izquierdo \\ \#08-86809}
\date{25 de junio de 2009}

\maketitle

% Pregunta 1
\section{}

$\stirlingD{n}{m}$ es el número de maneras de particionar un conjunto de $n$
elementos en $m$ subconjuntos no vacíos. Ahora, sea $A$ un conjunto
tal que $|A| = n$ y supongamos que tenemos un conjunto $B$ de contenedores
llamados $b_i$ con $1 \leq i \leq m$. Observemos entonces que los $F_i$
propuestos en el enunciado de la pregunta son las maneras de particionar $A$
colocando cada elemento en alguno de los $b_k$ con $k \neq i$, lo que garantiza
que $b_i$ queda vacío. Observemos también que si quisieramos particionar los $n$
elementos en los $m$ subconjuntos olvidando la restricción de prohibir
subconjuntos vacíos, hay $m^n$ opciones porque podemos elegir entre $m$
subconjuntos para colocar cada elemento.

Con todo esto tenemos que

\[ \left| \bigcup_{i=1}^m F_i \right| \]

es el número de maneras de particionar $A$ en los $b_i$ de manera de que exista
al menos un $b_i$ vacío. Por principio de inclusión-exclusión y utilizando el
hecho de que, para cualquier $q$, cualquier interseccón de $q$ distintos $F_i$
tiene la misma
cardinalidad, nos da

\begin{align*}
\left| \bigcup_{i=1}^m F_i \right| & = \sum_{k=1}^m \binom{m}{k} (-1)^{k-1} |F_1 \cap F_2 \cap \ldots \cap F_k|
\end{align*}

Ahora, sabemos que la intersección de $q$ $F_i$ distintos es el conjunto de las
maneras de particionar $A$ en $m-q$ conjuntos, sin importar que queden conjuntos
vacíos (porque estamos obligando a $q$ conjuntos a estar vacíos, y no nos
importan los demás con tal de que cada elemento de $A$ quede asignado). Entonces
la cardinalidad de una intersección tal es $(m-q)^n$. Por lo tanto queda

\begin{align*}
\left| \bigcup_{i=1}^m F_i \right| & = \sum_{k=1}^m \binom{m}{k} (-1)^{k-1} (m-k)^n
\end{align*}

Si lo que queremos es contar particiones en la que ningún $b_i$ esté
vacío, debemos tomar todas las posibles maneras de particionar y quitarles el
número que acabamos de calcular. Es decir, lo que buscamos es

\begin{align*}
m^n - \left| \bigcup_{i=1}^m F_i \right|
  & = m^n - \sum_{k=1}^m \binom{m}{k} (-1)^{k-1} (m-k)^n \\
  & = m^n + \sum_{k=1}^m \binom{m}{k} (-1)^{k} (m-k)^n \\
  & = \sum_{k=0}^m \binom{m}{k} (-1)^{k} (m-k)^n \\
  & = \sum_k \binom{m}{k} (-1)^{k} (m-k)^n \\
  & = \sum_{m-k} \binom{m}{m-k} (-1)^{m-k} k^n \\
  & = \sum_{m-k} \binom{m}{k} (-1)^{m-k} k^n \\
  & = \sum_k \binom{m}{k} (-1)^{m-k} k^n \\
\end{align*}

Para terminar nos damos cuenta de que $\stirlingD{n}{m}$ es igual a este
número salvo la numeración de los $b_i$. Hay $m!$ maneras de numerarlos
(número de permutaciones distintas de $m$ elementos), factor que
debemos eliminar del cálculo. Entonces nos queda

\[
\stirlingD{n}{m} = \frac{1}{m!} \sum_k \binom{m}{k} (-1)^{m-k} k^n \\
\]

% Pregunta 2
\section{}

% Pregunta 3
\section{}

\renewcommand{\labelenumi}{(\alph{enumi})}
\begin{enumerate}

\item Para un par de enteros $m,n$ con $1 \leq m \leq N$ y $1 \leq n \leq N$
pueden haber tres posibilidades: $m=n$, $m<n$ o $m>n$. Si $m=n$ entonces
$m \perp n \equiv m=n=1$, por lo que se cuenta un sólo caso para el resultado de
$R(N)$.

Analicemos qué ocurre si $m<n$. Para un $m \neq 1$, $\varphi(m)$ cuenta
los coprimos $c < m$. O sea que $\Phi(N)$ cuenta todos los pares $1 \leq n < m
\leq N$ con $n \perp m$ pero también añade el caso $n=m=1$, el cual debemos
eliminar. Entonces cuando $m<n$ debemos agregar $\Phi(N)-1$ a $R(N)$.
Al tener $m>n$ ocurre de manera similar, es decir, tenemos $\Phi(N)-1$ casos
más. Por lo tanto tenemos

\[
R(N) = \Phi(N)-1 + \Phi(N)-1 + 1 = 2\Phi(N) - 1
\]

\item

\end{enumerate}

% Pregunta 4
\section{}

\end{document}
