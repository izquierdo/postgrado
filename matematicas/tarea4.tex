\documentclass{article}

\usepackage{algorithm}
\usepackage{algorithmic}
\usepackage{amsmath}
\usepackage{amssymb}
\usepackage[utf8]{inputenc}
\usepackage{listings}
\usepackage[spanish]{babel}
\usepackage{qtree}
\usepackage{enumerate}

% Para codigo bonito
\newcommand{\singlespace}{\renewcommand{\baselinestretch}{1.0}}
\newcommand{\piso}[1]{\left \lfloor #1 \right \rfloor}
\newcommand{\techo}[1]{\left \lceil #1 \right \rceil}

\newcommand{\algoritmo}[2]
{
{
  \singlespace
  \begin{algorithm}[h]
    \caption{#1}
    \begin{algorithmic}[1]
      #2
    \end{algorithmic}
  \end{algorithm}
}
}

\newcommand{\theHalgorithm}{\arabic{algorithm}}

\lstset{mathescape=true,frame=single}

\newcommand{\asig}{\ensuremath{\leftarrow}}
\renewcommand{\lstlistingname}{Listado} %listings en español

\begin{document}

% Título

\title{Tarea 4 de Matemáticas para Computación}
\author{Daniel Izquierdo \\ \#08-86809}
\date{9 de junio de 2009}

\maketitle

% Pregunta 1
\section{}

\begin{enumerate}[a.]

 \item 5.8

Primero observemos que $p(k) = (1-\frac{k}{n})^n$ es un polinomio sobre $k$ de
grado $n$. $p(k)$ expande a
$(1-\frac{k}{n})(1-\frac{k}{n})\ldots n \text{ veces} \ldots(1-\frac{k}{n})$.
Vemos que el único término de grado $n$ en la expansión es $(-\frac{k}{n})^n$
y que no hay términos de mayor grado. El coeficiente de este término es
$(-\frac{1}{n})^n$. Entonces, por (5.42) del libro tenemos

\begin{align*}
\sum_k \binom{n}{k} (-1)^k (1-\frac{k}{n})^n & = (-1)^n n! (-\frac{1}{n})^n\\
                                             & = (-1)^n n! (-1)^n (\frac{1}{n})^n\\
                                             & = (-1)^{2n} n! \frac{1}{n^n}\\
                                             & = \frac{n!}{n^n}\\
\end{align*}

 \item 5.64

\end{enumerate}

% Pregunta 2
\section{}

% Pregunta 3
\section{}

\end{document}
