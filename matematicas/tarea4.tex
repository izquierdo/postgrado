\documentclass{article}

\usepackage{algorithm}
\usepackage{algorithmic}
\usepackage{amsmath}
\usepackage{amssymb}
\usepackage[utf8]{inputenc}
\usepackage{listings}
\usepackage[spanish]{babel}
\usepackage{qtree}
\usepackage{enumerate}

% Para codigo bonito
\newcommand{\singlespace}{\renewcommand{\baselinestretch}{1.0}}
\newcommand{\piso}[1]{\left \lfloor #1 \right \rfloor}
\newcommand{\techo}[1]{\left \lceil #1 \right \rceil}

\newcommand{\algoritmo}[2]
{
{
  \singlespace
  \begin{algorithm}[h]
    \caption{#1}
    \begin{algorithmic}[1]
      #2
    \end{algorithmic}
  \end{algorithm}
}
}

\newcommand{\theHalgorithm}{\arabic{algorithm}}

\lstset{mathescape=true,frame=single}

\newcommand{\asig}{\ensuremath{\leftarrow}}
\renewcommand{\lstlistingname}{Listado} %listings en español

\begin{document}

% Título

\title{Tarea 4 de Matemáticas para Computación}
\author{Daniel Izquierdo \\ \#08-86809}
\date{9 de junio de 2009}

\maketitle

% Pregunta 1
\section{}

\begin{enumerate}[a.]

 \item 5.8

Primero observemos que $p(k) = (1-\frac{k}{n})^n$ es un polinomio sobre $k$ de
grado $n$. $p(k)$ expande a
$(1-\frac{k}{n})(1-\frac{k}{n})\ldots n \text{ veces} \ldots(1-\frac{k}{n})$.
Vemos que el único término de grado $n$ en la expansión es $(-\frac{k}{n})^n$
y que no hay términos de mayor grado. El coeficiente de este término es
$(-\frac{1}{n})^n$. Entonces, por (5.42) del libro tenemos

\begin{align*}
\sum_k \binom{n}{k} (-1)^k (1-\frac{k}{n})^n & = (-1)^n n! (-\frac{1}{n})^n\\
                                             & = (-1)^n n! (-1)^n (\frac{1}{n})^n\\
                                             & = (-1)^{2n} n! \frac{1}{n^n}\\
                                             & = \frac{n!}{n^n}\\
\end{align*}

 \newpage
 \item 5.64

\begin{align*}
\sum_{k=0}^n \frac{\binom{n}{k}}{\techo{\frac{k+1}{2}}} & = \sum_{0 \leq k} \frac{\binom{n}{k}}{\techo{\frac{k+1}{2}}} \\
  & = \sum_{0 \leq k\text{, } k \text{ impar}} \frac{\binom{n}{k}}{\techo{\frac{k+1}{2}}} +
      \sum_{0 \leq k\text{, } k \text{ par}} \frac{\binom{n}{k}}{\techo{\frac{k+1}{2}}} \\
  & = \sum_{0 \leq k\text{, } k \text{ impar}} \frac{\binom{n}{k}}{\frac{k+1}{2}} +
      \sum_{0 \leq k\text{, } k \text{ par}} \frac{\binom{n}{k}}{\frac{k+2}{2}} \\
  & = \sum_{0 \leq k\text{, } k \text{ impar}} \frac{2 \binom{n}{k}}{k+1} +
      \sum_{0 \leq k\text{, } k \text{ par}} \frac{2 \binom{n}{k}}{k+2} \\
  & = \sum_{0 \leq k} \frac{2 \binom{n}{2k+1}}{(2k+1)+1} +
      \sum_{0 \leq k} \frac{2 \binom{n}{2k}}{2k+2} \\
  & = 2 \left( \sum_{0 \leq k} \frac{\binom{n}{2k+1}}{2k+2} +
      \sum_{0 \leq k} \frac{\binom{n}{2k}}{2k+2} \right) \\
  & = 2 \left( \sum_{0 \leq k} \left( \frac{\binom{n}{2k+1}}{2(k+1)} + \frac{\binom{n}{2k}}{2(k+1)} \right) \right) \\
  & = 2 \left( \sum_{0 \leq k} \frac{\binom{n}{2k+1}+\binom{n}{2k}}{2(k+1)} \right) \\
  & = \sum_{0 \leq k} \frac{\binom{n}{2k+1}+\binom{n}{2k}}{k+1} \\
  & = \sum_{0 \leq k} \frac{\binom{n+1}{2k+1}}{k+1} & \langle \text{por fórmula aditiva} \rangle \\
  & = \sum_{0 \leq k} \frac{2}{n+2} \frac{(n+2) \binom{n+1}{2k+1}}{2(k+1)} \\
  & = \frac{2}{n+2} \sum_{0 \leq k} \frac{n+2}{2k+2} \binom{n+1}{2k+1} \\
  & = \frac{2}{n+2} \sum_{0 \leq k} \binom{n+2}{2k+2} & \langle \text{por (5.5)} \rangle \\
\end{align*}

Ahora bien, sabemos que $\binom{n}{0}-\binom{n}{1}+\ldots+(-1)^n \binom{n}{n} = 0$.
Entonces debemos tener que las partes
$\binom{n}{0}+\binom{n}{2}+\ldots$ y $\binom{n}{1}+\binom{n}{3}+\ldots$ son iguales.
Como sabemos que $\binom{n}{0}+\binom{n}{1}+\ldots+\binom{n}{n} = 2^n$,
entonces debe ser $\binom{n}{0}+\binom{n}{2}+\ldots = \frac{2^n}{2} = 2^{n-1}$.
Por lo tanto,

\begin{align*}
\sum_{k=0}^n \frac{\binom{n}{k}}{\techo{\frac{k+1}{2}}} & = \frac{2}{n+2} \sum_{0 \leq k} \binom{n+2}{2k+2} \\
  & = \frac{2}{n+2} \left( \sum_{0 \leq k} \binom{n+2}{2k} - \binom{n+2}{0} \right) \\
  & = \frac{2}{n+2} \left( 2^{(n+2)-1} - 1 \right) \\
  & = \frac{2^{n+2} - 2}{n+2} \\
\end{align*}

\end{enumerate}

% Pregunta 2
\newpage
\section{}

Hago la demostración por inducción sobre $l$. Esto es permisible porque hay
restricciones $l$ entero y $l \geq 0$ para la identidad. Para el caso $l = 0$:

% \sum_k \binom{l}{m+k} \binom{s+k}{n} (-1)^k & = (-1)^{l+m} \binom{s-m}{n-l}

\begin{align*}
  & \sum_k \binom{0}{m+k} \binom{s+k}{n} (-1)^k = (-1)^{0+m} \binom{s-m}{n-0} \\
  \equiv & \\
  & \sum_k \binom{0}{m+k} \binom{s+k}{n} (-1)^k = (-1)^m \binom{s-m}{n} \\
\end{align*}

$\binom{0}{x} = 0$ si $x \neq 0$ y $\binom{0}{0} = 1$. Entonces el término de la
suma del lado izquierdo sólo es distinto de cero cuando $k = -m$. Queda:

\begin{align*}
  & \sum_k \binom{0}{m+k} \binom{s+k}{n} (-1)^k = (-1)^m \binom{s-m}{n} \\
  \equiv & \\
  & \binom{s-m}{n} (-1)^m = (-1)^m \binom{s-m}{n} \\
  \equiv & \\
  & \mathtt{true}
\end{align*}

Ahora supongamos que la identidad se cumple para todo $k < l$ con un
$l > 0$. Entonces

\begin{align*}
  & \sum_k \binom{l}{m+k} \binom{s+k}{n} (-1)^k \\
  = & \langle \text{fórmula aditiva} \rangle \\
  & \sum_k \left(\binom{l-1}{m+k}+\binom{l-1}{m+k-1}\right) \binom{s+k}{n} (-1)^k \\
  = & \\
  & \sum_k \binom{l-1}{m+k} \binom{s+k}{n} (-1)^k + \sum_k \binom{l-1}{m+k-1} \binom{s+k}{n} (-1)^k \\
  = & \langle \text{hipótesis inductiva} \rangle \\
  & (-1)^{l+m-1} \binom{s-m}{n-l+1} + (-1)^{l+m-2} \binom{s-m+1}{n-l+1} \\
  = & \langle (-1)^2 = 1 \rangle \\
  & (-1)^{l+m-1} \binom{s-m}{n-l+1} + (-1)^{l+m} \binom{s-m+1}{n-l+1} \\
  = & \langle \text{fórmula aditiva} \rangle \\
  & (-1)^{l+m-1} \binom{s-m}{n-l+1} + (-1)^{l+m} \left( \binom{s-m}{n-l+1} + \binom{s-m}{n-l} \right) \\
  = & \\
  & (-1)^{l+m-1} \binom{s-m}{n-l+1} + (-1)^{l+m} \binom{s-m}{n-l+1} + (-1)^{l+m} \binom{s-m}{n-l} \\
  = & \\
  & (-1)(-1)^{l+m-2} \binom{s-m}{n-l+1} + (-1)^{l+m} \binom{s-m}{n-l+1} + (-1)^{l+m} \binom{s-m}{n-l} \\
  = & \langle (-1)^2 = 1 \rangle \\
  & (-1)(-1)^{l+m} \binom{s-m}{n-l+1} + (-1)^{l+m} \binom{s-m}{n-l+1} + (-1)^{l+m} \binom{s-m}{n-l} \\
  = & \\
  & (-1)^{l+m} \binom{s-m}{n-l} \\
\end{align*}

% Pregunta 3
\newpage
\section{}

\end{document}
