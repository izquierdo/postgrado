\documentclass{article}

\usepackage{algorithm}
\usepackage{algorithmic}
\usepackage{amsmath}
\usepackage{amssymb}
\usepackage[utf8]{inputenc}
\usepackage{listings}
\usepackage[spanish]{babel}
\usepackage{qtree}
\usepackage{enumerate}

% Para codigo bonito
\newcommand{\singlespace}{\renewcommand{\baselinestretch}{1.0}}
\newcommand{\piso}[1]{\left \lfloor #1 \right \rfloor}
\newcommand{\techo}[1]{\left \lceil #1 \right \rceil}
\newcommand{\stirlingU}[2]{\genfrac{[}{]}{0pt}{}{#1}{#2}}
\newcommand{\stirlingD}[2]{\genfrac{\{}{\}}{0pt}{}{#1}{#2}}

\newcommand{\algoritmo}[2]
{
{
  \singlespace
  \begin{algorithm}[h]
    \caption{#1}
    \begin{algorithmic}[1]
      #2
    \end{algorithmic}
  \end{algorithm}
}
}

\newcommand{\theHalgorithm}{\arabic{algorithm}}

\lstset{mathescape=true,frame=single}

\newcommand{\asig}{\ensuremath{\leftarrow}}
\renewcommand{\lstlistingname}{Listado} %listings en español

\begin{document}

% Título

\title{Tarea 6 de Matemáticas para Computación}
\author{Daniel Izquierdo \\ \#08-86809}
\date{7 de julio de 2009}

\maketitle

% Pregunta 1 (7.11)
\section{}

\renewcommand{\labelenumi}{(\alph{enumi})}
\begin{enumerate}
 % (a)
 \item
 
 % (b) 
 \item Tomando el caso apropiado de la tabla 320 del libro empezamos:

\begin{align*}
zB'(z) & = \sum_n n b_n z^n \\
       & = \sum_n \left( \sum_{k=0}^n \frac{2^k a_k}{(n-k)!} \right) z^n & \langle \text{suposición} \rangle \\
       & = \sum_n \left( \sum_{k=0}^n (2^k a_k) \frac{1}{(n-k)!} \right) z^n \\
       & = \sum_n \left( \sum_{k \leq n} (2^k a_k) \frac{1}{(n-k)!} \right) z^n
           & \langle a_k=0 \text{ con } k \text{ negativo}  \rangle \\
       & = \sum_n \left( \sum_{k} (2^k a_k) \frac{1}{(n-k)!} \right) z^n
           & \langle \text{factorial indefinido para negativos}  \rangle \\
       & = A(2z) e^z & \langle \text{idents. de } F(z)G(z) \text{, } G(cz) \text{ y } e^z \rangle \\
\end{align*}

Entonces

\[
A(z) = (z/2) B'(z/2) e^{-(z/2)}
\]

 % (c)
 \item 

\begin{align*}
A(z) & = \sum_n \sum_{k=0}^n \binom{r+k}{k} b_{n-k} z^n \\
     & = \sum_n \sum_{k=0}^n \binom{r+k}{k} b_{n-k} z^n \\
     & = \sum_n \sum_{k} \binom{r+k}{k} b_{n-k} z^n & \langle b_k \text{ negativo con } k < 0 \rangle \\
     & = \frac{B(z)}{(1-z)^{r+1}} & \langle \text{idents. de } F(z)G(z) \text{, } 1/(1-z)^c \text{ si } c=r+1 \rangle \\
\end{align*}

Ahora, con esto tenemos

\[
B(z) = (1-z)^{r+1} A(z)
\]

Notemos que si $F(z) = (1+z)^{r+1}$, entonces $(1-z)^{r+1} = F((-1)z)$. Por lo
tanto,

\begin{align*}
B(z) & = (1-z)^{r+1} A(z) \\
     & = \sum_n \left( \sum_k (-1)^k \binom{r+1}{k} a_{n-k} \right) z^n
         & \langle \text{idents. } F(z)G(z) \text{, } (1+z)^c \text{, } G(cz)\rangle  \\
     & = \sum_n \left( \sum_{k=0}^n (-1)^k \binom{r+1}{k} a_{n-k} \right) z^n \\
\end{align*}

Entonces el $f_k(r)$ solicitado es

\[
 f_k(r) = (-1)^k \binom{r+1}{k}
\]


\end{enumerate}

% Pregunta 2 (9.14)
\newpage
\section{}

\begin{align*}
(n+\alpha)^{n+\beta} & = n^{n+\beta}(1+\frac{\alpha}{n})^{n+\beta} \\
                     & = n^{n+\beta} e^{\ln (1+\frac{\alpha}{n})^{n+\beta}}\\
                     & = n^{n+\beta} e^{(n+\beta)\ln (1+\frac{\alpha}{n})}\\
                     & = n^{n+\beta} e^{(n+\beta) (\frac{\alpha}{n}-\frac{1}{2}(\frac{\alpha}{n})^2+O(n^{-3}))}
                         & \langle \text{(9.33)} \rangle \\
                     & = n^{n+\beta} e^{\alpha+\frac{\beta\alpha}{n}-\frac{\alpha^2}{2n}-\frac{\beta\alpha^2}{2n^2}+O(n^{-2}+\beta n^{-3})} \\
                     & = n^{n+\beta} e^{\alpha+\frac{\beta\alpha}{n}-\frac{\alpha^2}{2n}-\frac{\beta\alpha^2}{2n^2}+O(n^{-2})} \\
                     & = n^{n+\beta} e^{\alpha}e^{\frac{2\beta\alpha-\alpha^2}{2n}}e^{\frac{-\beta\alpha^2}{2n^2}}e^{O(n^{-2})} \\
                     & = n^{n+\beta} e^{\alpha}e^{(\frac{2\beta\alpha-\alpha^2}{2})n^{-1}}e^{(\frac{-\beta\alpha^2}{2})n^{-2}}e^{O(n^{-2})} \\
                     & = n^{n+\beta}
                         e^{\alpha}
                         (1 + \frac{2\beta\alpha-\alpha^2}{2}n^{-1}+O(n^{-2}))* \\
                         & (1+\frac{-\beta\alpha^2}{2}n^{-2}+O(n^{-4})
                         (1+O(n^{-2})) \\
                     & = n^{n+\beta} e^{\alpha} (1 + \frac{2\beta\alpha-\alpha^2}{2}n^{-1}+O(n^{-2})) \\
                     & = n^{n+\beta} e^{\alpha} (1 + \alpha (\beta - \frac{1}{2}\alpha) n^{-1}+O(n^{-2})) \\
\end{align*}

% Pregunta 3 (9.27)
\newpage
\section{}

Para empezar veamos que $f(0) = 0$, por lo que $\sum_{k=1}^n f(k) = \sum_{k=0}^n f(k)$.
Usaremos la reescritura (9.78) de la fórmula de Euler.

\begin{align*}
& \sum_{k=0}^n f(k) \\
= & \\
& \sum_{0 \leq k < n+1} f(k) \\
= & \\
& \sum_{0 \leq k < n} f(k) + f(n) \\
= & \\
& \int_0^n f(x)dx - \left. \frac{1}{2} f(x) \right|_0^n + \sum_{k=1}^m \frac{B_{2k}}{(2k)!} \left. f^{(2k-1)}(x) \right|_0^n + \\
& O((2\pi)^{-2m}) \int_0^n | f^{(2m)}(x) | dx + n^\alpha \\
= & \\
& \int_0^n x^\alpha dx - \left. \frac{1}{2} x^\alpha \right|_0^n + \sum_{k=1}^m \frac{B_{2k}}{(2k)!} \left. f^{(2k-1)}(x) \right|_0^n + \\
& O((2\pi)^{-2m}) \int_0^n | f^{(2m)}(x) | dx + n^\alpha \\
= & \\
& \left. \frac{x^{\alpha+1}}{\alpha+1} \right|_0^n
  - \left. \frac{1}{2} x^\alpha \right|_0^n + \sum_{k=1}^m \frac{B_{2k}}{(2k)!} \left. f^{(2k-1)}(x) \right|_0^n + \\
& O((2\pi)^{-2m}) \int_0^n | f^{(2m)}(x) | dx + n^\alpha \\
= & \\
& \frac{n^{\alpha+1}}{\alpha+1}
  - \frac{1}{2} n^\alpha + \sum_{k=1}^m \frac{B_{2k}}{(2k)!} \left. f^{(2k-1)}(x) \right|_0^n 
  + O((2\pi)^{-2m}) \int_0^n | f^{(2m)}(x) | dx + n^\alpha \\
= & \\
& \frac{n^{\alpha+1}}{\alpha+1}
  + \frac{1}{2} n^\alpha + \sum_{k=1}^m \frac{B_{2k}}{(2k)!} \left. f^{(2k-1)}(x) \right|_0^n 
  + O((2\pi)^{-2m}) \int_0^n | f^{(2m)}(x) | dx \\
= & \\
& \frac{n^{\alpha+1}}{\alpha+1}
  + \frac{1}{2} n^\alpha + \sum_{k=1}^m \frac{B_{2k}}{(2k)!} \left. \alpha^{\underline{2k-1}} x^{\alpha-2k+1} \right|_0^n +
  O((2\pi)^{-2m}) \int_0^n | f^{(2m)}(x) | dx \\
\end{align*}

\begin{align*}
= & \\
& \frac{n^{\alpha+1}}{\alpha+1}
  + \frac{1}{2} n^\alpha + \sum_{k=1}^m \frac{B_{2k}}{(2k)!} \alpha^{\underline{2k-1}} n^{\alpha-2k+1} +
  O((2\pi)^{-2m}) \int_0^n | f^{(2m)}(x) | dx \\
= & \\
& \frac{n^{\alpha+1}}{\alpha+1}
  + \frac{1}{2} n^\alpha + \sum_{k=1}^m \frac{B_{2k}}{(2k)(2k-1)!} \alpha^{\underline{2k-1}} n^{\alpha-2k+1} + \\
& O((2\pi)^{-2m}) \int_0^n | f^{(2m)}(x) | dx \\
= & \\
& \frac{n^{\alpha+1}}{\alpha+1}
  + \frac{1}{2} n^\alpha + \sum_{k=1}^m \frac{B_{2k}}{2k} \binom{\alpha}{2k-1} n^{\alpha-2k+1} + 
  O((2\pi)^{-2m}) \int_0^n | f^{(2m)}(x) | dx \\
\end{align*}

\end{document}
