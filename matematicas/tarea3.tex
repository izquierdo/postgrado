\documentclass{article}

\usepackage{algorithm}
\usepackage{algorithmic}
\usepackage{amsmath}
\usepackage{amssymb}
\usepackage[utf8]{inputenc}
\usepackage{listings}
\usepackage[spanish]{babel}
\usepackage{qtree}

% Para codigo bonito
\newcommand{\singlespace}{\renewcommand{\baselinestretch}{1.0}}
\newcommand{\piso}[1]{\left \lfloor #1 \right \rfloor}
\newcommand{\techo}[1]{\left \lceil #1 \right \rceil}

\newcommand{\algoritmo}[2]
{
{
  \singlespace
  \begin{algorithm}[h]
    \caption{#1}
    \begin{algorithmic}[1]
      #2
    \end{algorithmic}
  \end{algorithm}
}
}

\newcommand{\theHalgorithm}{\arabic{algorithm}}

\lstset{mathescape=true,frame=single}

\newcommand{\asig}{\ensuremath{\leftarrow}}
\renewcommand{\lstlistingname}{Listado} %listings en español

\begin{document}

% Título

\title{Tarea 3 de Matemáticas para Computación}
\author{Daniel Izquierdo \\ \#08-86809}
\date{26 de mayo de 2009}

\maketitle

% Pregunta 1
\section{}

Si $n < 1$, la suma da $0$. Suponiendo $n \geq 1$.

\begin{align*}
 & \sum_{1 \leq k \leq n} \piso{log_2k} \\
 = & \\
 & \sum_{1 \leq k < n} \piso{log_2k} + \piso{log_2n} \\
 = & \\
 & \sum_{1 \leq k < n} m[m = \piso{log_2k}] + \piso{log_2n} \\
 = & \\
 & \sum_{1 \leq k < n} m[m \leq log_2k < m+1] + \piso{log_2n} \\
 = & \\
 & \sum_{1 \leq k < n} m[2^m \leq k < 2^{m+1}] + \piso{log_2n} \\
 = & \\
 & \sum m[1 \leq k < n][2^m \leq k < 2^{m+1}] + \piso{log_2n} \\
 = & \\
 & \sum m[2^m \leq k < n < 2^{m+1}] + \sum m[2^m \leq k < 2^{m+1} \leq n] + \piso{log_2n} \\
\end{align*}

Si $n = 2^a$ entonces la primera suma es $0$ porque no puede haber una potencia de $2$
estrictamente entre una potencia de $2$ y la próxima. Para la segunda suma tenemos:

\begin{align*}
\sum m[2^m \leq k < 2^{m+1} \leq 2^a] & = \sum m (2^{m+1}-2^m)[m+1 \leq a] \\
  & = \sum m (2*2^m-2^m)[m+1 \leq a] \\
  & = \sum m 2^m[m < a] \\
  & = \sum_0^{a-1} m 2^m \\
  & = 2((a-1)2^{(a-1)} - 2^{(a-1)} + 1) \\
\end{align*}

Para el último paso busqué la solución de la suma en una tabla.
Para el caso general podemos poner $a = \piso{log_2n}$ y sólo nos resta agregar
los términos para $2^a \leq k < n$ que son todos iguales a $a$ (por el uso de la
función piso) y entonces suman a $(n-2^a)a$. Entonces tenemos

\begin{align*}
 & \sum_{1 \leq k \leq n} \piso{log_2k} \\
 = & \\
 & 2((a-1)2^{(a-1)} - 2^{(a-1)} +1) + \piso{log_2n} + (n-2^a)a \\
 = & \\
 & 2((a-1)2^{(a-1)} - 2^{(a-1)} +1) + (n-2^a+1)a \\
\end{align*}

con

$$
a = \piso{log_2n}
$$

% Pregunta 2
\newpage

\section{}

\begin{align*}
 & \sum_{1 \leq k \leq n} k^3 + \sum_{1 \leq k \leq n} k^2 \\
 = & \\
 & \sum_{1 \leq k \leq n}(k^3+k^2) \\
 = & \\
 & \sum_{1 \leq k \leq n}k^2(k+1) \\
 = & \\
 & \sum_{1 \leq k \leq n}k(k(k+1)) \\
 = & \\
 & \sum_{1 \leq k \leq n}k\left(2 \sum_{1 \leq j \leq k} j\right) \\
 = & \\
 & \sum_{1 \leq k \leq n}2k\left(\sum_{1 \leq j \leq k} j\right) \\
 = & \\
 & 2\sum_{1 \leq k \leq n}k\left(\sum_{1 \leq j \leq k} j\right) \\
 = & \\
 & 2\sum_{1 \leq j \leq k \leq n}kj \\
 = & \langle \text{2.33} \rangle \\
 & 2\left(\frac{1}{2}\left(\left(\sum_{1 \leq k \leq n} k \right)^2 + \sum_{1 \leq k \leq n}k^2 \right)\right) \\
 = & \\
 & \left(\sum_{1 \leq k \leq n} k \right)^2 + \sum_{1 \leq k \leq n}k^2 \\
\end{align*}

Por lo tanto,

\begin{align*}
 & \sum_{1 \leq k \leq n} k^3 \\
 = & \\
 & \left(\sum_{1 \leq k \leq n} k \right)^2 \\
 = & \\
 & \left( \frac{k(k+1)}{2} \right)^2 \\
\end{align*}

% Pregunta 3
\section{} % (2.23)

\subsection{} % (a)

\begin{align*}
\sum_{k=1}^n \frac{2k+1}{k(k+1)} & = \sum_{k=1}^n (2k+1)(\frac{1}{k} - \frac{1}{k+1}) \\ 
  & = \sum_{k=1}^n (\frac{2k+1}{k} - \frac{2k+1}{k+1}) \\ 
  & = \sum_{k=1}^n \frac{2k+1}{k} - \sum_{k=1}^n \frac{2k+1}{k+1} \\
  & = \sum_{k=1}^n \frac{2k}{k} + \sum_{k=1}^n \frac{1}{k} - \sum_{k=1}^n \frac{2k+1}{k+1} \\
  & = 2n + H_n - \sum_{k=1}^n \frac{2k+1}{k+1} \\
  & = 2n + H_n - \sum_{1 \leq k \leq n} \frac{2k+1}{k+1} \\
  & = 2n + H_n - \sum_{1 \leq k-1 \leq n} \frac{2(k-1)+1}{(k-1)+1} \\
  & = 2n + H_n - \sum_{1 \leq k-1 \leq n} \frac{2k-1}{k} \\
  & = 2n + H_n - \sum_{1 \leq k-1 \leq n} \frac{2k}{k} + \sum_{1 \leq k-1 \leq n} \frac{1}{k} \\
  & = 2n + H_n - 2n + \sum_{2 \leq k \leq n+1} \frac{1}{k} \\
  & = 2n + H_n - 2n + (\sum_{1 \leq k \leq n} \frac{1}{k} - \sum_{1 \leq k \leq 1} \frac{1}{k} + \sum_{n+1 \leq k \leq n+1} \frac{1}{k}) \\
  & = 2n + H_n - 2n + (H_n - 1 + \frac{1}{n+1}) \\
  & = 2H_n - 1 + \frac{1}{n+1} \\
\end{align*}

\subsection{} % (b)

Sean $u(x) = 2x+1, \Delta v(x) = \frac{1}{x(x+1)}$. Entonces $\Delta u(x) = 2$
y $v(x) = -\frac{1}{x}$. Usando suma por partes tenemos

% u = 2x+1
% du = 2
% v = -1/x
% dv = 1/(x(x+1))

\begin{align*}
\sum u \Delta v & = uv - \sum Ev \Delta u \\
 & = (2x+1)(-\frac{1}{x}) - \sum -\frac{1}{x+1}(2) \\
 & = -\frac{2x+1}{x} - \sum -\frac{2}{x+1} \\
 & = -\frac{2x+1}{x} + 2\sum \frac{1}{x+1} \\
 & = -\frac{2x+1}{x} + 2\sum x^{\underline{-1}} \\
 & = 2H_x - \frac{2x+1}{x} \\
\end{align*}

Entonces

\begin{align*}
\sum_{k=1}^n \frac{2k+1}{k(k+1)} & = \sum_{k=1}^n u \Delta v \\
             & = 2H_x -\frac{2x+1}{x} \Big |_{1}^{n+1} \\
             & = 2H_{n+1} -\frac{2(n+1)+1}{n+1} - 2H_1 + \frac{2+1}{1} \\
             & = 2H_{n+1} -\frac{2n+3}{n+1} - 2 + 3 \\
             & = 2H_{n+1} -\frac{2n+3}{n+1} + 1 \\
\end{align*}

% Pregunta 4
\section{} % (2.27)

\begin{align*}
\Delta(c^{\underline{x}}) & = c^{\underline{x+1}} - c^{\underline{x}} \\
  & = c(c-1) \ldots (c-x+1)(c-x+2) - c(c-1) \ldots (c-x+1) \\
  & = c(c-1) \ldots (c-x+1)((c-x+2)-1) \\
  & = (c(c-1) \ldots (c-x+1))(c-x-1) \\
  & = c^{\underline{x}}(c-x-1) \\
  & = \frac{c^{\underline{x+2}}}{c-x} \\
\end{align*}

Con esto tenemos el caso particular:

$$
\Delta((-2)^{\underline{x-2}}) = -\frac{(-2)^{\underline{x}}}{x}
$$

y

$$
\Delta(-(-2)^{\underline{x-2}}) = \frac{(-2)^{\underline{x}}}{x}
$$

Entonces

\begin{align*}
\sum_{k=1}^n \frac{(-2)^{\underline{k}}}{k} & = -(-2)^{\underline{x-2}} \Big |_{1}^{n+1} \\
  & = - (-2)^{\underline{n-1}} + (-2)^{\underline{-1}} \\
\end{align*}

% Pregunta 5
\section{} % (2.29)

Observemos que

$$
\frac{k}{4k^2-1} = \frac{1}{4} \left( \frac{1}{2k+1} + \frac{1}{2k-1} \right)
$$

Entonces

\begin{align*}
\sum_{k=1}^n (-1)^k \frac{k}{4k^2-1} & = \sum_{k=1}^n (-1)^k \frac{1}{4} \left( \frac{1}{2k+1} + \frac{1}{2k-1} \right) \\
  & = \sum_{k=1}^n (-1)^k \frac{1}{4} \left( \frac{1}{2k+1} \right) + \sum_{k=1}^n (-1)^k \frac{1}{4} \left( \frac{1}{2k-1} \right) \\
\end{align*}

Ahora, si

$$
f(k) = \frac{1}{4} \left( \frac{1}{2k+1} \right)
$$

y

$$
g(k) = \frac{1}{4} \left( \frac{1}{2k-1} \right)
$$

tenemos

\begin{align*}
 & \sum_{k=1}^n (-1)^k \frac{1}{4} \left( \frac{1}{2k+1} \right) + \sum_{k=1}^n (-1)^k \frac{1}{4} \left( \frac{1}{2k-1} \right) \\
 = & \\
 & \sum_{k=1}^n (-1)^k f(k) + \sum_{k=1}^n (-1)^k g(k) \\
 = & \\
 & -f(1)+f(2)-\ldots+(-1)^n f(n)-g(1)+g(2)-\ldots+(-1)^n g(n) \\
\end{align*}

Pero $g(k+1) = f(k)$ porque

\begin{align*}
g(k+1) & = \frac{1}{4} \left( \frac{1}{2(k+1)-1} \right) \\
       & = \frac{1}{4} \left( \frac{1}{2k+2-1} \right) \\
       & = \frac{1}{4} \left( \frac{1}{2k+1} \right) \\
       & = f(k) \\
\end{align*}

Por lo tanto, en el resultado que teníamos cada $g(k+1)$ se cancela con su $f(k)$ correspondiente
porque tienen signos distintos ($(-1)^k y (-1)^{k+1}$. Al final sólo quedan sin eliminar $g(1)$ y
$f(n)$. Entonces

\begin{align*}
\sum_{k=1}^n (-1)^k \frac{k}{4k^2-1} & = (-1)^1 f(1) + (-1)^n g(n) \\
                                     & = -\frac{1}{4} + (-1)^n \left( \frac{1}{4(2n+1)} \right) \\
                                     & = -\frac{1}{4} + (-1)^n \left( \frac{1}{8n+4} \right) \\
\end{align*}

\end{document}
