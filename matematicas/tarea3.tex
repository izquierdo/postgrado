\documentclass{article}

\usepackage{algorithm}
\usepackage{algorithmic}
\usepackage{amsmath}
\usepackage{amssymb}
\usepackage[utf8]{inputenc}
\usepackage{listings}
\usepackage[spanish]{babel}
\usepackage{qtree}

% Para codigo bonito
\newcommand{\singlespace}{\renewcommand{\baselinestretch}{1.0}}
\newcommand{\piso}[1]{\left \lfloor #1 \right \rfloor}
\newcommand{\techo}[1]{\left \lceil #1 \right \rceil}

\newcommand{\algoritmo}[2]
{
{
  \singlespace
  \begin{algorithm}[h]
    \caption{#1}
    \begin{algorithmic}[1]
      #2
    \end{algorithmic}
  \end{algorithm}
}
}

\newcommand{\theHalgorithm}{\arabic{algorithm}}

\lstset{mathescape=true,frame=single}

\newcommand{\asig}{\ensuremath{\leftarrow}}
\renewcommand{\lstlistingname}{Listado} %listings en español

\begin{document}

% Título

\title{Tarea 2 de Matemáticas para Computación}
\author{Daniel Izquierdo \\ \#08-86809}
\date{12 de mayo de 2009}

\maketitle

% Pregunta 1
\section{}


% Pregunta 2
\newpage

\section{}

\begin{align*}
 & \sum_{1 \leq k \leq n} k^3 + \sum_{1 \leq k \leq n} k^2 \\
 = & \\
 & \sum_{1 \leq k \leq n}(k^3+k^2) \\
 = & \\
 & \sum_{1 \leq k \leq n}k^2(k+1) \\
 = & \\
 & \sum_{1 \leq k \leq n}k(k(k+1)) \\
 = & \\
 & \sum_{1 \leq k \leq n}k\left(2 \sum_{1 \leq j \leq k} j\right) \\
 = & \\
 & \sum_{1 \leq k \leq n}2k\left(\sum_{1 \leq j \leq k} j\right) \\
 = & \\
 & 2\sum_{1 \leq k \leq n}k\left(\sum_{1 \leq j \leq k} j\right) \\
 = & \\
 & 2\sum_{1 \leq j \leq k \leq n}kj \\
 = & \langle \text{2.33} \rangle \\
 & 2\left(\frac{1}{2}\left(\left(\sum_{1 \leq k \leq n} k \right)^2 + \sum_{1 \leq k \leq n}k^2 \right)\right) \\
 = & \\
 & \left(\sum_{1 \leq k \leq n} k \right)^2 + \sum_{1 \leq k \leq n}k^2 \\
\end{align*}

Por lo tanto,

\begin{align*}
 & \sum_{1 \leq k \leq n} k^3 \\
 = & \\
 & \left(\sum_{1 \leq k \leq n} k \right)^2 \\
 = & \\
 & \left( \frac{k(k+1)}{2} \right)^2 \\
\end{align*}

% Pregunta 3
\section{}

% Pregunta 4
\section{}

% Pregunta 5
\section{}

\end{document}
