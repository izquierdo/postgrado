\documentclass{article}

\usepackage{algorithm}
\usepackage{algorithmic}
\usepackage{amsmath}
\usepackage{amssymb}
\usepackage[utf8]{inputenc}
\usepackage{listings}
\usepackage[spanish]{babel}
\usepackage{qtree}

% Para codigo bonito
\newcommand{\singlespace}{\renewcommand{\baselinestretch}{1.0}}
\newcommand{\piso}[1]{\left \lfloor #1 \right \rfloor}
\newcommand{\techo}[1]{\left \lceil #1 \right \rceil}

\newcommand{\algoritmo}[2]
{
{
  \singlespace
  \begin{algorithm}[h]
    \caption{#1}
    \begin{algorithmic}[1]
      #2
    \end{algorithmic}
  \end{algorithm}
}
}

\newcommand{\theHalgorithm}{\arabic{algorithm}}

\lstset{mathescape=true,frame=single}

\newcommand{\asig}{\ensuremath{\leftarrow}}
\renewcommand{\lstlistingname}{Listado} %listings en español

\begin{document}

% Título

\title{Tarea 2 de Matemáticas para Computación}
\author{Daniel Izquierdo \\ \#08-86809}
\date{12 de mayo de 2009}

\maketitle

% Pregunta 1
\section{}

Supongamos que $F$ no termina. Entonces existe una cadena
infinita de llamadas

$$
P_0: F(x_0, y_0) \rightarrow F(x_1, y_1) \rightarrow F(x_2, y_2) \rightarrow \ldots
$$

donde defino $\rightarrow$ como ``llama a''.
Aquí, no puede ser que haya $x_i$ tal que $x_i = 0$ o $y_i$ tal que $y_i = 0$
porque
en estos casos $F$ no hace llamadas recursivas.
%Observemos que hay una
%correspondencia uno a uno $c$ entre llamadas a $F$ y el producto cartesiano de
%los naturales, de manera que $c(F(x,y)) = (x,y)$.
Observemos que cuando
$F(x,y) \rightarrow F(u,v)$, tenemos dos posibilidades:

\begin{itemize}
\item $(u,v) = (x-1,r)$ para algún $r$. Tenemos por definición del orden
lexicográfico que $(x,y) \succeq (x-1,r)$ porque $x-1 < x$.
\item $(u,v) = (x,y-1)$. Tenemos también por definición del orden lexicográfico
que $(x,y) \succeq (x,y-1)$ porque $x = x \wedge y-1 < y$.
\end{itemize}

De esta manera, $F(x,y) \rightarrow F(u,v)$ implica que $(x,y) \succeq (u,v)$.
Luego, la existencia de $P_0$ implica que existe una cadena infinita

$$
(x_0, y_0) \succeq (x_1, y_1) \succeq (x_2, y_2) \succeq \ldots
$$

lo cual es falso porque el orden es Noetheriano. Esta contradicción nos lleva a
concluir que $F$ termina.

% Pregunta 2
\section{}

Primero quiero ver qué debe cumplir $d$ para que $f(n) \leq dn$ siga de suponer
que $n$ es suficientemente grande y $f(m) \leq dm$ para todo $m$ menor que $n$
y también suficientemente grande. Si hacemos tal suposición y tenemos
$\piso{pn}$ y $\piso{qn}$ suficientemente grandes, entonces necesitamos
$f(n) \leq dn$ y tenemos

\begin{align*}
 & f(n) \\
 = & \\
 & f(\piso{pn}) + f(\piso{qn}) + bn \\
 \leq & \\
 & d\piso{pn} + d\piso{qn} + bn \\
 \leq & \\
 & dpn + dqn + bn \\
 = & \\
 & dn(p+q) + bn \\
 = & \\
 & n(d(p+q) + b) \\
\end{align*}

Para poder concluir que $f(n) \leq dn$ necesitamos que

\begin{align*}
 & n(d(p+q) + b) \leq dn \\
 \equiv & \\
 & d(p+q) + b \leq d \\
 \equiv & \\
 & b \leq d - d(p+q) \\
 \equiv & \\
 & b \leq d(1 - (p+q)) \\
 \equiv & \\
 & \frac{b}{1-(p+q)} \leq d \\
\end{align*}

El término de la izquierda en la última ecuación está definido y es positivo
porque $p+q$ es positivo y menor que $1$ y $b$ es positivo. Entonces tenemos
que, bajo esa condición, $f(n) \leq dn$ es cierto si
$f(\piso{pn}) \leq d\piso{pn}$ y $f(\piso{qn}) \leq d\piso{qn}$ también lo son.
Para culminar la prueba necesito números apropiados como base para aplicar
inducción.

% Pregunta 3
\section{}

Estas funciones tienen la propiedad de que $\piso{f(x)} = \piso{f(\techo{x}}$.
La prueba es similar a la dada en el libro para demostrar que
$\techo{f(x)} = \techo{f(\techo{x})}$.
Para todo $x$ entero se cumple trivialmente porque $x = \techo{x}$. Si $x$ no
es entero, tenemos $x < \techo{x}$. Como $f$ es monótona decreciente, esto
implica que $f(x) > f(\techo{x})$. Entonces
$\piso{f(x)} \geq \piso{f(\techo{x})}$.
Cuando es $\piso{f(x)} > \piso{f(\techo{x})}$, debe haber $y$ tal que
$x \leq y < \techo{x}$ y $f(y) = \piso{f(x)}$ porque $f$ es continua.
$y$ debe
ser entero por la propiedad
de $f$ de tomar valores enteros sólo con argumentos enteros.
Pero no puede haber enteros $i$ tales que $x \leq i < \techo{x}$. 
Con esto llegamos
a una contradicción.
Entonces debe ser $\piso{f(x)} = \piso{f(\techo{x})}$.

% Pregunta 4
\section{}

Tenemos

\begin{align*}
 & \techo{\frac{n}{m}} = \piso{\frac{n+m-1}{m}} \\
 \equiv & \\
 & \techo{\frac{n}{m}} - \piso{\frac{n}{m}} = \piso{\frac{n+m-1}{m}} - \piso{\frac{n}{m}} \\
 \equiv & \langle \piso{\frac{n}{m}} \text{ es entero} \rangle \\
 & \techo{\frac{n}{m} - \piso{\frac{n}{m}}} = \piso{\frac{n+m-1}{m} - \piso{\frac{n}{m}}} \\
 \equiv & \\
 & \techo{\frac{n - m\piso{\frac{n}{m}}}{m}} = \piso{\frac{n+m-1 - m\piso{\frac{n}{m}}}{m}} \\
 \equiv & \langle m \text{ positivo} \rangle \\
 & \techo{\frac{n \mathtt{mod} m}{m}} = \piso{\frac{n \mathtt{mod} m +m-1}{m}} \\
\end{align*}

Esto último es equivalente a $n \mathtt{mod} m > 0$, porque debe ser
$0 \leq n \mathtt{mod} m < m$.
Entonces es equivalente a $m > 0$. Por lo tanto, $m > 0$ implica la igualdad que se quería
demostrar.

% Pregunta 5
\section{}

Sea $N = \frac{3^77-1}{2}$. $N$ es impar porque 
$N$ es compuesto porque es divisible entre $\frac{3^7-1}{2}$:

\end{document}
