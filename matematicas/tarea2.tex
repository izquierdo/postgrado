\documentclass{article}

\usepackage{algorithm}
\usepackage{algorithmic}
\usepackage{amsmath}
\usepackage{amssymb}
\usepackage[utf8]{inputenc}
\usepackage{listings}
\usepackage[spanish]{babel}
\usepackage{qtree}

% Para codigo bonito
\newcommand{\singlespace}{\renewcommand{\baselinestretch}{1.0}}

\newcommand{\algoritmo}[2]
{
{
  \singlespace
  \begin{algorithm}[h]
    \caption{#1}
    \begin{algorithmic}[1]
      #2
    \end{algorithmic}
  \end{algorithm}
}
}

\newcommand{\theHalgorithm}{\arabic{algorithm}}

\lstset{mathescape=true,frame=single}

\newcommand{\asig}{\ensuremath{\leftarrow}}
\renewcommand{\lstlistingname}{Listado} %listings en español

\begin{document}

% Título

\title{Tarea 2 de Matemáticas para Computación}
\author{Daniel Izquierdo \\ \#08-86809}
\date{12 de mayo de 2009}

\maketitle

% Pregunta 1
\section{}

Supongamos que $F$ no termina. Entonces existe una cadena
infinita de llamadas

$$
P_0: F(x_0, y_0) \rightarrow F(x_1, y_1) \rightarrow F(x_2, y_2) \rightarrow \ldots
$$

donde defino $\rightarrow$ como ``llama a''.
Aquí, no puede ser que haya $x_i$ tal que $x_i = 0$ o $y_i$ tal que $y_i = 0$
porque
en estos casos $F$ no hace llamadas recursivas.
%Observemos que hay una
%correspondencia uno a uno $c$ entre llamadas a $F$ y el producto cartesiano de
%los naturales, de manera que $c(F(x,y)) = (x,y)$.
Observemos que cuando
$F(x,y) \rightarrow F(u,v)$, tenemos dos posibilidades:

\begin{itemize}
\item $(u,v) = (x-1,r)$ para algún $r$. Tenemos por definición del orden
lexicográfico que $(x,y) \succeq (x-1,r)$ porque $x-1 < x$.
\item $(u,v) = (x,y-1)$. Tenemos también por definición del orden lexicográfico
que $(x,y) \succeq (x,y-1)$ porque $x = x \wedge y-1 < y$.
\end{itemize}

De esta manera, $F(x,y) \rightarrow F(u,v)$ implica que $(x,y) \succeq (u,v)$.
Luego, la existencia de $P_0$ implica que existe una cadena infinita

$$
(x_0, y_0) \succeq (x_1, y_1) \succeq (x_2, y_2) \succeq \ldots
$$

lo cual es falso porque el orden es Noetheriano. Esta contradicción nos lleva a
concluir que $F$ termina.

% Pregunta 2
\section{}

% Pregunta 3
\section{}

% Pregunta 4
\section{}

% Pregunta 5
\section{}

% Pregunta 6
\section{}

\end{document}
