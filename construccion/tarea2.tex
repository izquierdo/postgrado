\documentclass{article}

\usepackage{algorithm}
\usepackage{algorithmic}
\usepackage{amsmath}
\usepackage{amssymb}
\usepackage[utf8]{inputenc}
\usepackage{listings}
\usepackage[spanish]{babel}
\usepackage{alltt}


% Para codigo bonito
\newcommand{\singlespace}{\renewcommand{\baselinestretch}{1.0}}

\newcommand{\algoritmo}[2]
{
{
  \singlespace
  \begin{algorithm}[h]
    \caption{#1}
    \begin{algorithmic}[1]
      #2
    \end{algorithmic}
  \end{algorithm}
}
}

\newcommand{\theHalgorithm}{\arabic{algorithm}}

\lstset{language=GCL,mathescape=true,frame=single}

\begin{document}

% Título

\title{Tarea 2 \\ Construcción Formal de Programas}
\author{Daniel Izquierdo \\ \#08-86809}
\date{2 de diciembre de 2008}

\maketitle

%que empiecen las secciones desde 0
\setcounter{section}{-1}

% Pregunta 0
\section{}

\setcounter{subsection}{-1}

\subsection{skip}

\begin{align*}
 & [ wp.skip.P \wedge wp.skip.Q \equiv wp.skip.(P \wedge Q) ] \\
 \equiv & <wp.skip.Q \equiv Q \text{ tres veces}> \\
 & [ P \wedge Q \equiv P \wedge Q ] \\
 \equiv & <P \equiv P \text{, } true \text{ constante}> \\
 & true
\end{align*}

\begin{align*}
 & [ wp.skip.P \vee wp.skip.Q \equiv wp.skip.(P \vee Q) ] \\
 \equiv & <wp.skip.Q \equiv Q \text{ tres veces}> \\
 & [ P \vee Q \equiv P \vee Q ] \\
 \equiv & <P \equiv P \text{, } true \text{ constante}> \\
 & true \\
\end{align*}

\subsection{asignación}

\begin{align*}
 & wp.(x := E).(P \wedge Q) \\
 \equiv & <wp.(x := E).Q \equiv Q(x := E)> \\
 & (P \wedge Q)(x := E) \\
 \equiv & <\text{sustitución}> \\
 & P(x := E) \wedge Q(x := E) \\
 \equiv & <wp.(x := E).Q \equiv Q(x := E) \text{ dos veces}> \\
 & wp.(x := E).P \wedge wp.(x := E).Q \\
\end{align*}

\begin{align*}
 & wp.(x := E).(P \vee Q) \\
 \equiv & <wp.(x := E).Q \equiv Q(x := E)> \\
 & (P \vee Q)(x := E) \\
 \equiv & <\text{sustitución}> \\
 & P(x := E) \vee Q(x := E) \\
 \equiv & <wp.(x := E).Q \equiv Q(x := E) \text{ dos veces}> \\
 & wp.(x := E).P \vee wp.(x := E).Q \\
\end{align*}

% Pregunta 1
\section{}

\end{document}
