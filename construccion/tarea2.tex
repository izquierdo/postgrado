\documentclass{article}

\usepackage{algorithm}
\usepackage{algorithmic}
\usepackage{amsmath}
\usepackage{amssymb}
\usepackage[utf8]{inputenc}
\usepackage{listings}
\usepackage[spanish]{babel}
\usepackage{alltt}


% Para codigo bonito
\newcommand{\singlespace}{\renewcommand{\baselinestretch}{1.0}}

\newcommand{\algoritmo}[2]
{
{
  \singlespace
  \begin{algorithm}[h]
    \caption{#1}
    \begin{algorithmic}[1]
      #2
    \end{algorithmic}
  \end{algorithm}
}
}

\newcommand{\theHalgorithm}{\arabic{algorithm}}

\lstset{language=GCL,mathescape=true,frame=single}

\begin{document}

% Título

\title{Tarea 2 \\ Construcción Formal de Programas}
\author{Daniel Izquierdo \\ \#08-86809}
\date{2 de diciembre de 2008}

\maketitle

%que empiecen las secciones desde 0
\setcounter{section}{-1}

% Pregunta 0
\section{}

\setcounter{subsection}{-1}

\subsection{skip}

\begin{align*}
 & [ wp.skip.P \wedge wp.skip.Q \equiv wp.skip.(P \wedge Q) ] \\
 \equiv & <wp.skip.Q \equiv Q \text{ tres veces}> \\
 & [ P \wedge Q \equiv P \wedge Q ] \\
 \equiv & <P \equiv P \text{, } true \text{ constante}> \\
 & true
\end{align*}

\begin{align*}
 & [ wp.skip.P \vee wp.skip.Q \equiv wp.skip.(P \vee Q) ] \\
 \equiv & <wp.skip.Q \equiv Q \text{ tres veces}> \\
 & [ P \vee Q \equiv P \vee Q ] \\
 \equiv & <P \equiv P \text{, } true \text{ constante}> \\
 & true \\
\end{align*}

\subsection{asignación}

\begin{align*}
 & wp.(x := E).(P \wedge Q) \\
 \equiv & <wp.(x := E).Q \equiv Q(x := E)> \\
 & (P \wedge Q)(x := E) \\
 \equiv & <\text{sustitución}> \\
 & P(x := E) \wedge Q(x := E) \\
 \equiv & <wp.(x := E).Q \equiv Q(x := E) \text{ dos veces}> \\
 & wp.(x := E).P \wedge wp.(x := E).Q \\
\end{align*}

\begin{align*}
 & wp.(x := E).(P \vee Q) \\
 \equiv & <wp.(x := E).Q \equiv Q(x := E)> \\
 & (P \vee Q)(x := E) \\
 \equiv & <\text{sustitución}> \\
 & P(x := E) \vee Q(x := E) \\
 \equiv & <wp.(x := E).Q \equiv Q(x := E) \text{ dos veces}> \\
 & wp.(x := E).P \vee wp.(x := E).Q \\
\end{align*}

\subsection{secuenciación}

Suponiendo que $S_0$ y $S_1$ son instrucciones que
cumplen la segunda, la tercera y la cuarta ``condición de salubridad'':

\begin{align*}
 & wp.(S_0 ; S_1).(P \wedge Q) \\
 \equiv & <wp.(S_0 ; S_1).Q \equiv wp.S_0.(wp.S_1.Q)> \\
 & wp.S_0.(wp.S_1.(P \wedge Q)) \\
 \equiv & <S_1 \text{ cumple tercera condición de salubridad}> \\
 & wp.S_0.(wp.S_1.P \wedge wp.S_1.Q) \\
 \equiv & <S_0 \text{ cumple tercera condición de salubridad}> \\
 & wp.S_0.(wp.S_1.P) \wedge wp.S_0.(wp.S_1.Q) \\
 \equiv & <wp.(S_0 ; S_1).Q \equiv wp.S_0.(wp.S_1.Q) \text{ dos veces}> \\
 & wp.(S_0 ; S_1).P \wedge wp.(S_0 ; S_1).Q \\
\end{align*}

\begin{align*}
 & wp.(S_0 ; S_1).P \vee wp.(S_0 ; S_1).Q \\
 \equiv & <wp.(S_0 ; S_1).Q \equiv wp.S_0.(wp.S_1.Q) \text{ dos veces}> \\
 & wp.S_0.(wp.S_1.P) \vee wp.S_0.(wp.S_1.Q) \\
 \Rightarrow & <S_0 \text{ cumple cuarta condición de salubridad}> \\
 & wp.S_0.(wp.S_1.P \vee wp.S_1.Q) \\
 \Rightarrow & <S_1 \text{ cumple cuarta condición de salubridad y} \\
             & S_0 \text{ cumple segunda condición de salubridad}> \\
 & wp.S_0.(wp.S_1.(P \vee Q)) \\
 \equiv & <wp.(S_0 ; S_1).Q \equiv wp.S_0.(wp.S_1.Q)> \\
 & wp.(S_0 ; S_1).(P \vee Q) \\
\end{align*}

\subsection{selección condicional}

Suponiendo que $S_i$ con $1 \leq i \leq n$ es una instrucción que
cumple la tercera y la cuarta ``condición de salubridad'',
sean

\begin{align*}
IF & = & \text{{\bf if }} B_1 \rightarrow S_1 | \ldots | B_n \rightarrow S_n \text{ {\bf fi}}  \\
R & \equiv & 1 \leq i \leq n \\
BB & \equiv & (\exists i : R : B_i) \\
\end{align*}

Entonces,

\begin{align*}
 & wp.IF.P \wedge wp.IF.Q \\
 \equiv & <wp.IF.Q \equiv BB \wedge (\forall i : R : B_i \Rightarrow wp.S_i.Q) \text{ dos veces}> \\
 & (BB \wedge (\forall i : R : B_i \Rightarrow wp.S_i.P)) \wedge (BB \wedge (\forall i : R : B_i \Rightarrow wp.S_i.Q)) \\
 \equiv & <P \wedge P \equiv P> \\
 & BB \wedge (\forall i : R : B_i \Rightarrow wp.S_i.P) \wedge (\forall i : R : B_i \Rightarrow wp.S_i.Q) \\
 \equiv & <(\forall x : R : A) \wedge (\forall x : R : B) \equiv (\forall x : R : A \wedge B)> \\
 & BB \wedge (\forall i : R : (B_i \Rightarrow wp.S_i.P) \wedge (B_i \Rightarrow wp.S_i.Q)) \\
 \equiv & <(P \Rightarrow Q) \wedge (P \Rightarrow R) \equiv P \Rightarrow Q \wedge R> \\
 & BB \wedge (\forall i : R : B_i \Rightarrow wp.S_i.P \wedge wp.S_i.Q) \\
 \equiv & <S_i \text{ cumple tercera condición de salubridad}> \\
 & BB \wedge (\forall i : R : B_i \Rightarrow wp.S_i.(P \wedge Q)) \\
 \equiv & <wp.IF.Q \equiv BB \wedge (\forall i : R : B_i \Rightarrow wp.S_i.Q)> \\
 & wp.IF.(P \wedge Q)
\end{align*}

\begin{align*}
 & wp.IF.P \vee wp.IF.Q \\
 \equiv & <wp.IF.Q \equiv BB \wedge (\forall i : R : B_i \Rightarrow wp.S_i.Q) \text{ dos veces}> \\
 & (BB \wedge (\forall i : R : B_i \Rightarrow wp.S_i.P)) \vee (BB \wedge (\forall i : R : B_i \Rightarrow wp.S_i.Q)) \\
 \equiv & <\text{distributividad } \wedge \text{ sobre } \vee> \\
 & BB \wedge ((\forall i : R : B_i \Rightarrow wp.S_i.P) \vee (\forall i : R : B_i \Rightarrow wp.S_i.Q)) \\
 \Rightarrow & <(\forall x : R : A) \vee (\forall x : R : B) \Rightarrow (\forall x : R : A \vee B)> \\
 & BB \wedge (\forall i : R : (B_i \Rightarrow wp.S_i.P) \vee (B_i \Rightarrow wp.S_i.Q)) \\
 \equiv & <(P \Rightarrow Q) \vee (P \Rightarrow R) \equiv P \Rightarrow Q \vee R> \\
 & BB \wedge (\forall i : R : B_i \Rightarrow wp.S_i.P \vee wp.S_i.Q) \\
 \Rightarrow & <S_i \text{ cumple cuarta condición de salubridad}> \\
 & BB \wedge (\forall i : R : B_i \Rightarrow wp.S_i.(P \vee Q)) \\
 \equiv & <wp.IF.Q \equiv BB \wedge (\forall i : R : B_i \Rightarrow wp.S_i.Q)> \\
 & wp.IF.(P \vee Q)
\end{align*}

% Pregunta 1
\section{}

Utilizo los $H_k$ dados en el paper de Dijkstra para definir $wp.DO.R$. Supongo que $k$ es
una variable que no ocurre en $Q$.

\begin{align*}
 & \neg B \wedge Q \\
 \equiv & <\text{definición de } H_0(R) \text{ con } BB = B> \\
 & H_0(Q) \\
 \equiv & <\text{sustitución, } k \text{ no ocurre en } Q> \\
 & (H_k(Q))(k := 0) \\
 \Rightarrow & <\text{introducción de existencial}> \\
 & (\exists k :: H_k(Q)) \\
 \equiv & <\text{precondición más débil de la iteración}> \\
 & wp.(do B \rightarrow I od).Q \\
\end{align*}

% Pregunta 2
\section{}

\begin{align*}
 & wp.(\text{{\bf if }} (| i \cdot G_i \rightarrow w : [G_i \wedge pre, post]) \text{ {\bf fi}}).Q \\
 \equiv & <\text{definiciones de } wp.(\text{{\bf if }} (| i \cdot G_i \rightarrow w : [G_i \wedge pre, post]) \text{ {\bf fi}}) \text{ y } GG> \\
 & (\exists i :: G_i) \wedge (\forall i :: G_i \Rightarrow wp.(w : [G_i \wedge pre, post]).Q \\
 \equiv & <\text{definición de } wp.[pre,post]> \\
 & (\exists i :: G_i) \wedge (\forall i :: G_i \Rightarrow (G_i \wedge pre \wedge (\forall w :: post \Rightarrow Q))) \\
 \equiv & <p \Rightarrow p \wedge q \equiv p \Rightarrow q> \\
 & (\exists i :: G_i) \wedge (\forall i :: G_i \Rightarrow (pre \wedge (\forall w :: post \Rightarrow Q))) \\
 \equiv & <\text{dist. } \wedge \text{/ } \exists \text{, renombramiento } i := j \text{, rango no vacío}> \\
 & (\exists i :: G_i \wedge (\forall j :: G_j \Rightarrow pre \wedge (\forall w :: post \Rightarrow Q))) \\
 \equiv & <\text{dist. } \wedge \text{/ } \forall \text{, rango no vacío}> \\
 & (\exists i :: (\forall j :: G_i \wedge (G_j \Rightarrow pre \wedge (\forall w :: post \Rightarrow Q)))) \\
 \Rightarrow & <\text{instanciación } \forall \text{, } j := i> \\
 & (\exists i :: G_i \wedge (G_i \Rightarrow pre \wedge (\forall w :: post \Rightarrow Q))) \\
 \Rightarrow & <\text{Modus Ponens}> \\
 & (\exists i :: pre \wedge (\forall w :: post \Rightarrow Q)) \\
 \equiv & <i \text{ no ocurre libre en } pre \wedge (\forall w :: post \Rightarrow Q) \text{, rango } true> \\
 & pre \wedge (\forall w :: post \Rightarrow Q) \\
 \equiv & <\text{definición de } wp.[pre,post]> \\
 & wp.(w : [pre,post]).Q
\end{align*}

\end{document}
