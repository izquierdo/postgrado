\documentclass{article}

\usepackage{algorithm}
\usepackage{algorithmic}
\usepackage{amsmath}
\usepackage{amssymb}
\usepackage[utf8]{inputenc}
\usepackage{listings}
\usepackage[spanish]{babel}
\usepackage{alltt}


% Para codigo bonito
\newcommand{\singlespace}{\renewcommand{\baselinestretch}{1.0}}

\newcommand{\algoritmo}[2]
{
{
  \singlespace
  \begin{algorithm}[h]
    \caption{#1}
    \begin{algorithmic}[1]
      #2
    \end{algorithmic}
  \end{algorithm}
}
}

\newcommand{\theHalgorithm}{\arabic{algorithm}}

\lstset{language=GCL,mathescape=true,frame=single}

\begin{document}

% Título

\title{Tarea 1 \\ Construcción Formal de Programas}
\author{Daniel Izquierdo \\ \#08-86809}
\date{4 de noviembre de 2008}

\maketitle

% Pregunta 1
\section{Ejercicio 2.0.0}

\begin{itemize}
\item $\{true\} S \{true\}$: el programa $S$ siempre termina.
\item $\{false\} S \{true\}$: no da información sobre la terminación de $S$.
\end{itemize}

% Pregunta 2
\section{Ejercicio 2.2.1}

Debo demostrar $\{true\} x := x + 1 \{true\}$. Tenemos

\begin{align*}
 & \{true\} x := x + 1 \{true\} \\
 \equiv & <\text{equivalencia TH / wp}> \\
 & [ true \Rightarrow wp.(x := x + 1).true ] \\
 \equiv & <true => P \equiv P> \\
 & [ wp.(x := x + 1).true ] \\
 \equiv & <wp.(x := E).Q \equiv Q(x := E)> \\
 & [ true(x := x + 1) ] \\
 \equiv & <\text{sustitución}> \\
 & [ true ] \\
 \equiv & <\text{cuantificación sin variables ligadas}> \\
 & true
\end{align*}

% Pregunta 3
\section{Ejercicio 2.4.1}

\begin{align*}
 & \{P\} \text{{\bf if }} B_0 \rightarrow S_0 ; S | B_1 \rightarrow S_1 ; S \text{ {\bf fi}} \{Q\} \\
 \equiv & <\text{definición de {\bf if}}> \\
 & [ P \Rightarrow B_0 \vee B_1 ] \wedge \{P \wedge B_0\} S_0 ; S \{Q\} \wedge \{P \wedge B_1\} S_1 ; S \{Q\} \\
 \equiv & <\text{equivalencia TH / wp}> \\
 & [ P \Rightarrow B_0 \vee B_1 ] \wedge [P \wedge B_0 \Rightarrow wp.(S_0 ; S).Q] \wedge [P \wedge B_1 \Rightarrow wp.(S_1 ; S).Q] \\
 \equiv & <[wp.(S ; T).Q \equiv wp.S.(wp.T.Q)]> \\
 & [ P \Rightarrow B_0 \vee B_1 ] \wedge [P \wedge B_0 \Rightarrow wp.S_0.(wp.S.Q)] \wedge [P \wedge B_1 \Rightarrow wp.S_1.(wp.S.Q)] \\
 \equiv & <\text{equivalencia TH / wp}> \\
 & [ P \Rightarrow B_0 \vee B_1 ] \wedge \{P \wedge B_0\} S_0 \{wp.S.Q\} \wedge \{P \wedge B_1\} S_1 \{wp.S.Q\} \\
 \equiv & <\text{definición de {\bf if}}> \\
 & \{P\} \text{{\bf if }} B_0 \rightarrow S_0 | B_1 \rightarrow S_1 \text{ {\bf fi}} \{wp.S.Q\} \\
 \equiv & <\text{equivalencia TH / wp}> \\
 & [P \Rightarrow wp.(\text{{\bf if }} B_0 \rightarrow S_0 | B_1 \rightarrow S_1 \text{ {\bf fi}}).(wp.S.Q)] \\
 \equiv & <[wp.(S ; T).Q \equiv wp.S.(wp.T.Q)]> \\
 & [P \Rightarrow wp.(\text{{\bf if }} B_0 \rightarrow S_0 | B_1 \rightarrow S_1 \text{ {\bf fi}} ; S).Q] \\
 \equiv & <\text{equivalencia TH / wp}> \\
 & \{P\} \text{{\bf if }} B_0 \rightarrow S_0 | B_1 \rightarrow S_1 \text{ {\bf fi}} ; S \{Q\}
\end{align*}

% Pregunta 4
\section{Ejercicio 2.6.2}

\begin{align*}
 & \{P\} S \{Q\} \\
 \equiv & <\text{equivalencia TH / wp}> \\
 & [P => wp.S.Q] \\
 \Rightarrow & <\text{introducción del existencial con } y=y> \\
 & [P => (\exists y : y \in \mathbb{Z} : wp.S.Q)] \\
 \Rightarrow & <wp.S.Q \vee wp.S.R \Rightarrow wp.S.(Q \vee R)> \\
 & [P => wp.S.(\exists y : y \in \mathbb{Z} : Q)] \\
 \equiv & <y \text{ no ocurre libre en } (\exists y : y \in \mathbb{Z} : Q)> \\
 & [P => wp.S.(\forall y : y \in \mathbb{Z} : (\exists y : y \in \mathbb{Z} : Q))] \\
 \equiv & <wp.S.Q \wedge wp.S.R \equiv wp.S.(Q \wedge R)> \\
 & [P => (\forall y : y \in \mathbb{Z} : wp.S.(\exists y : y \in \mathbb{Z} : Q))] \\
 \equiv & <[wp.|[\text{var } y : \text{int } ; S |].Q \equiv (\forall y : y \in \mathbb{Z} : wp.S.Q)]> \\
 & [P => wp.|[\text{var } y : \text{int } ; S |].(\exists y : y \in \mathbb{Z} : Q)] \\
 \equiv & <\text{equivalencia TH / wp}> \\
 & \{P\} |[\text{var } y : \text{int } ; S |] \{(\exists y : y \in \mathbb{Z} : Q)\} \\
\end{align*}

% Pregunta 5
\section{Ejercicio 4.1.3}

Para empezar, observo que $i = A*B$ satisface $i$ $mod$ $A = 0 \wedge i$ $mod$ $B = 0$.
Como estamos buscando el mínimo, esto me da una cota superior para el resultado.
Pongo como primera invariante

$$
P_0: 1 \leq x \leq A*B
$$

Usando la técnica de tomar una conjunción como invariante, pongo la invariante

$$
P_1: x \text{ {\bf mod} } A = 0
$$

y el guardia

$$
B: x \text{ {\bf mod} } B \neq 0
$$

Cuando tengamos $P_0 \wedge P_1 \wedge \neg B$ tendremos un múltiplo de $A$ y $B$, mas
no necesariamente el mínimo. Añado entonces una condición más a la invariante:

$$
P_2: (\forall i : 1 \leq i < x \wedge i \text{ {\bf mod} } A = 0 : i \text{ {\bf mod} } B \neq 0)
$$

De esta manera, si se cumplen las invariantes y no el guardia llegamos a la postcondición.
Es claro que a inicialización $x := A$ satisface $P_0$ y $P_1$. También satisface
$P_2$, porque no hay número $r$ tal que $1 \leq r < x \wedge r \text{ {\bf mod} } A = 0$.
Con todo esto tenemos una aproximación a la solución (listado \ref{prog413_parcial}).

\begin{lstlisting}[float,caption={Solución parcial del ejercicio 4.1.3},label=prog413_parcial]
|[
    con A, B : int {$A > 0 \wedge B > 0$};
    var x : int;

    x := A;

    {inv $P_0 \wedge P_1 \wedge P_2$}
    do $x mod B \neq 0 \rightarrow$
        S
    od

    {$x = A \text{ {\bf lcm} } B$}
]|
\end{lstlisting}

En el rango de la cuantificación en $P_2$ sólo considero números divisibles por $A$.
El próximo número a considerar en un momento dado es $x + A$. Investigo para $S$ la
asignación $x := x + A$. Es claro que $P_1 \Rightarrow P_1(x := x + A)$. Por otra parte,
suponiendo la invariante, sea $p > 0$ tal que $x=p*A$. Tenemos $1 \leq A*p = x < A*B$ (porque
$x \text{ {\bf mod} } B \neq 0$. Por lo tanto, $p < B$ y entonces

\begin{align*}
 & P_0(x := x + A) \\
 \equiv & <\text{sustitución y definición de } P_0> \\
 & 1 \leq x + A \leq A*B \\
 \equiv & <\text{x = p*A}> \\
 & 1 \leq p*A + A \leq A*B \\
 \equiv & <\text{aritmética}> \\
 & 1 \leq A*(p+1) \leq A*B \\
 \equiv & <p < B, A*p \geq 1, \text{aritmética}> \\
 & true
\end{align*}

Además,

\begin{align*}
 & P_2(x := x + A) \\
 \equiv & <\text{sustitución y definición de } P_2> \\
 & (\forall i : 1 \leq i < x+A \wedge i \text{ {\bf mod} } A = 0 : i \text{ {\bf mod} } B \neq 0) \\
 \equiv & <\text{separación de rangos}> \\
 & (\forall i : 1 \leq i < x \wedge i \text{ {\bf mod} } A = 0 : i \text{ {\bf mod} } B \neq 0) \wedge \\
 & (\forall i : x \leq i < x+A \wedge i \text{ {\bf mod} } A = 0 : i \text{ {\bf mod} } B \neq 0) \\
 \equiv & <\text{definición } P_2> \\
 & P_2 \wedge (\forall i : x \leq i < x+A \wedge i \text{ {\bf mod} } A = 0 : i \text{ {\bf mod} } B \neq 0) \\
 \equiv & <\text{separación de rangos, suposición } P_2 \text{, neutro de } \wedge \text{, } P_1> \\
 & x \text{ {\bf mod} } B \neq 0 \wedge (\forall i : x+1 \leq i < x+A \wedge i \text{ {\bf mod} } A = 0 : i \text{ {\bf mod} } B \neq 0) \\
 \equiv & <\text{suposición guardia, neutro de } \wedge> \\
 & (\forall i : x+1 \leq i < x+A \wedge i \text{ {\bf mod} } A = 0 : i \text{ {\bf mod} } B \neq 0) \\
 \equiv & <\text{rango vacío por } P_1> \\
 & true
\end{align*}

Para ver la terminación del ciclo, pongo la variante $A*B-x$. Noto primero que
$P_0$ implica que la variante es no negativa. Si tenemos $X = A*B-x$, entonces

\begin{align*}
 & X(x := x + A) \\
 = & <\text{sustitución}> \\
 & A*B-x+A \\
 < & <A > 0, \text{aritmética, definición de X}> \\
 & X
\end{align*}

Queda entonces el programa final (listado \ref{prog413}).

\begin{lstlisting}[float,caption={Solución del ejercicio 4.1.3},label=prog413]
|[
    con A, B : int {$A > 0 \wedge B > 0$};
    var x : int;

    x := A;

    {inv $P_0 \wedge P_1 \wedge P_2$}
    {bound A*B-x}
    do $x mod B \neq 0 \rightarrow$
        x := x + A
    od

    {$x = A \text{ {\bf lcm} } B$}
]|
\end{lstlisting}

% Pregunta 6
\section{Ejercicio 4.2.4}

Voy a utilizar un lazo con la técnica de sustitución de constante por variable
para resolver el problema. Introduzco la variable $n$ y entonces tenemos
las invariantes

$$
P_0: r = (\sum i : 0 \leq i < n : f.i * X^i)
$$

y

$$
P_1: 0 \leq n \leq N
$$

Por construcción, $P_0 \wedge n = N$ implica la postcondición. Como la sumatoria
sobre un rango vacío es 0, se satisfacen $P_0$ y $P_1$ con $n, r := 0, 0$. El
guardia del lazo será $n \neq N$. Al incrementar $n$ en 1 y suponer
$P_0 \wedge P_1 \wedge n \neq N$ tenemos 

\begin{align*}
 & (\sum i : 0 \leq i < n + 1: f.i * X^i) \\
 = & <\text{separación de rango con } 0 \leq n < n+ 1> \\
 & (\sum i : 0 \leq i < n : f.i * X^i) + f.n * X^n \\
 = & <P_0> \\
 & r + f.n * X^n
\end{align*}

En este punto noto que es necesaria una variable para el valor de $X^n$.
Introduzco entonces la variable $x$ con la invariante

$$
P_2: x = X^n
$$

Con esto queda que

\begin{align*}
 & (\sum i : 0 \leq i < n + 1: f.i * X^i) \\
 = & <\text{demostrado previamente}> \\
 & r + f.n * X^n \\
 = & <P_2> \\
 & r + f.n * x
\end{align*}

Por lo tanto, la asignación $n, r := n + 1, r + f.n * x$ basta para
satisfacer $P_0$. Queda por satisfacer $P_2$. Derivo

\begin{align*}
 & X^{(n+1)} \\
 = & <\text{definición de potencia}> \\
 & X^n * X \\
 = & <P_2> \\
 & x * X
\end{align*}

En conclusión, la asignación a ejecutar dentro del lazo es
$n, r, x := n + 1, r + f.n * x, x * X$. Agregar $x = 1$ a la sección
de inicialización permite que se cumpla $P_2$ al iniciar el lazo, ya
que $X^0 = 1$. Como función de cota usamos $N-n$. El programa final se
muestra en el listado \ref{prog424}.

\begin{lstlisting}[float,caption={Solución del ejercicio 4.2.4},label=prog424]
|[
    con N, X : int {$ N \geq 0$}; f : array[0..N) of int;
    var r : int;

    var n, x : int;

    n, r, x := 0, 0, 1;
    {inv $P_0 \wedge P_1 \wedge P_2$}
    {bound $N-n$}
    do $n \neq N \rightarrow$
        n, r, x := n + 1, r + f.n * x, x * X
    od

    {$r = (\sum i : 0 \leq i < N : f.i * X^i)$}
]|
\end{lstlisting}

% Pregunta 7
\section{Ejercicio 4.3.5}

Introduzco las variables $m, x$ e $y$. $m$ representará el crédito dado
por el segmento maximal, y $x$ e $y$ denotarán un segmento maximal. La
postcondición del programa a derivar es

$$
m = (\text{{\bf max} } p, q : 0 \leq p \leq q \leq N : credit.p.q) \wedge m = credit.x.y 
$$

Voy a utilizar un lazo con la técnica de sustitución de constante por variable. Para ello,
introduzco la variable $n$ en sustitución de $N$. Las invariantes del lazo son

$$
P_0: m = (\text{{\bf max} } p, q : 0 \leq p \leq q \leq n : credit.p.q)
$$

$$
P_1: m = credit.x.y
$$

$$
P_2: 0 \leq n \leq N
$$

La instrucción de inicialización $n, m, x, y := 0, 0, 0, 0$ satisface las invariantes
ya que $credit$.$0$.$0 = 0$. Al tratar de incrementar $n$ en $1$, tenemos

\begin{align*}
 & (\text{{\bf max} } p, q : 0 \leq p \leq q \leq n + 1 : credit.p.q) \\
 = & <\text{separación de rango con } n \geq 0, P_0> \\
 & m \text{ {\bf max} } (\text{{\bf max} } p : 0 \leq p \leq n + 1 : credit.p.(n+1)) \\
\end{align*}

Aquí noto la necesidad de una variable $s$ con invariante

$$
s = (\text{{\bf max} } p : 0 \leq p \leq n + 1 : credit.p.(n+1))
$$

Ese predicado no está definido para $n = N$, lo cual es admitido
por $P_2$. Entonces defino la invariante

$$
P_3: s = (\text{{\bf max} } p : 0 \leq p \leq n : credit.p.n)
$$

que es válida para $n = N$ e intento establecer $P_3(n := n+1)$ antes de la
instrucción de actualización de $m$. Para hacerlo, dados $P_3$ y $n \neq N$,
tenemos

\begin{align*}
 & (\text{{\bf max} } p : 0 \leq p \leq n + 1 : credit.p.(n+1)) \\
 = & <\text{separación de rango con } n \geq 0> \\
 & (\text{{\bf max} } p : 0 \leq p \leq n : credit.p.(n+1)) \text{ {\bf max} } credit.(n+1).(n+1) \\
 = & <\text{crédito con rango vacío es } 0> \\
 & (\text{{\bf max} } p : 0 \leq p \leq n : credit.p.(n+1)) \text{ {\bf max} } 0 \\
\end{align*}

Al llegar a este punto me doy cuenta de que hay tres posibilidades 
dependiendo del valor de $A.n$.

\begin{itemize}
 \item Caso 0 ($A.n = 0$):

 \begin{align*}
  & (\text{{\bf max} } p : 0 \leq p \leq n : credit.p.(n+1)) \text{ {\bf max} } 0 \\
  = & <\text{definición de } credit \text{, separación, aritmética, suposición } A.n = 0> \\
  & s \text{ {\bf max} } 0 \\
 \end{align*}

 \item Caso 1 ($A.n < 0$):

 \begin{align*}
  & (\text{{\bf max} } p : 0 \leq p \leq n : credit.p.(n+1)) \text{ {\bf max} } 0 \\
  = & <\text{definición de } credit \text{, separación, aritmética, suposición } A.n = 0> \\
  & s-1 \text{ {\bf max} } 0 \\
 \end{align*}

 \item Caso 2 ($A.n > 0$):

 \begin{align*}
  & (\text{{\bf max} } p : 0 \leq p \leq n : credit.p.(n+1)) \text{ {\bf max} } 0 \\
  = & <\text{definición de } credit \text{, separación, aritmética, suposición } A.n = 0> \\
  & s+1 \text{ {\bf max} } 0 \\
 \end{align*}

\end{itemize}

Para establecer $P_3(n := n + 1)$ puedo usar, entonces, la instrucción de selección. Con ese resultado,
intento de nuevo derivar la instrucción para mantener $P_0$.

\begin{align*}
 & (\text{{\bf max} } p, q : 0 \leq p \leq q \leq n + 1 : credit.p.q) \\
 = & <\text{separación de rango con } n \geq 0, P_0> \\
 & m \text{ {\bf max} } (\text{{\bf max} } p : 0 \leq p \leq n + 1 : credit.p.(n+1)) \\
 = & <\text{suposición de que se mantiene } P_3(n := n+1)> \\
 & m \text{ {\bf max} } s \\
\end{align*}

La inicialización $s := 0$ satisface $P_3$.
Hecho esto, escribo una aproximación parcial del programa (listado \ref{prog435_parcial}). Este programa
satisface $P_0$, $P_2$ y $P_3$ pero no necesariamente $P_1$. $P_1$ se puede ver afectado por la instrucción
que actualiza $m$ en caso que $m \neq m$ {\bf max} $s$.

\begin{lstlisting}[float,caption={Primera solución parcial del ejercicio 4.3.5},label=prog435_parcial]
|[
    con N : int { $N \geq 0$ };
    var A : array[0..N) of int;

    var n, m, x, y, s : int;

    n, m, x, y, s := 0, 0, 0, 0, 0;

    do $n \neq N \rightarrow$
        if $A.n = 0 \rightarrow$ $s$ := $s$ $max$ $0$
        |  $A.n < 0 \rightarrow$ $s$ := $(s-1)$ $max$ $0$
        |  $A.n > 0 \rightarrow$ $s$ := $(s+1)$ $max$ $0$
        fi;
        $m$ := $m$ $max$ $s$
    od
]|
\end{lstlisting}

Introduzco la nueva variable $e$ y la invariante

$$
P_4: s = credit.e.n
$$

Si $m \neq m$ {\bf max} $s$, entonces $s = m$ {\bf max} $s$. Luego de la asignación quedará $m = s$.
Es claro que, si realizamos $x, y := e, n+1$, entonces $P_1$ se satisfará si suponemos que $P_4(n := n+1)$ se
mantiene.

$P_4$ se cumple tras la inicialización $e := 0$. En el programa parcial que tengo se
puede ver afectada con la modificación de $s$. Observo que $s \geq 0$ se mantiene como
condición invariante y escribo una segunda aproximación del programa para ayudarme a lograr mantener
$P_4$ (listado \ref{prog435_parcial1}).

\begin{lstlisting}[float,caption={Segunda solución parcial del ejercicio 4.3.5},label=prog435_parcial1]
|[
    con N : int { $N \geq 0$ };
    var A : array[0..N) of int;

    var n, m, x, y, s : int;

    n, m, x, y, s, e := 0, 0, 0, 0, 0, 0;

    do $n \neq N \rightarrow$
        if $A.n = 0 \rightarrow$ skip
        |  $A.n < 0 \rightarrow$ $s$ :=
               if $(s-1) \neq (s-1)$ $max$ $0 \rightarrow$ s := 0
               |  $(s-1) = (s-1)$ $max$ $0 \rightarrow$ s := s-1
               fi
        |  $A.n > 0 \rightarrow$ $s$ := $(s+1)$
        fi;
        if $m = m$ $max$ $s \rightarrow$ skip
        |  $m \neq m$ $max$ $s \rightarrow$ m, x, y := s, e, n+1
        fi
    od
]|
\end{lstlisting}

Hay dos casos en los que cambia $s$. Por una parte tenemos $A.n > 0$:

\begin{align*}
 & (credit.e.n)(n := n+1) \\
 \equiv & <\text{sustitución, definición de credit}> \\
 & (\# i : e \leq i < n+1 : A.i > 0) - (\# i : e \leq i < n+1 : A.i < 0) \\
 \equiv & <\text{separación de rangos, suposición } A.n > 0> \\
 & (\# i : e \leq i < n : A.i > 0) + 1 - (\# i : e \leq i < n : A.i < 0) \\
 \equiv & <\text{aritmética, suposición } P_4> \\
 & s + 1 \\
\end{align*}

Como realizamos la asignación $s := s + 1$ en este caso, $P_4(n := n+1)$ se mantiene.
El segundo caso se presenta cuando $A.n < 0$ y se puede dividir en dos subcasos como se
ve en el listado \ref{prog435_parcial1}. Primero tenemos $(s - 1) \neq (s - 1)$ $max$ $0$.
Poner $e := n+1$ lo resuelve ya que el crédito de un rango vacío es $0$ y en este caso
asignamos $0$ a $s$. Cuando $(s - 1) = (s - 1)$ $max$ $0$ tenemos

\begin{align*}
 & (credit.e.n)(n := n+1) \\
 \equiv & <\text{sustitución, definición de credit}> \\
 & (\# i : e \leq i < n+1 : A.i > 0) - (\# i : e \leq i < n+1 : A.i < 0) \\
 \equiv & <\text{separación de rangos, suposición } A.n < 0> \\
 & (\# i : e \leq i < n : A.i > 0) - ((\# i : e \leq i < n : A.i < 0) + 1) \\
 \equiv & <\text{aritmética, suposición } P_4> \\
 & s - 1
\end{align*}

Como realizamos la asignación $s := s - 1$ en este caso, $P_4(n := n+1)$ se mantiene. Obtengo
así el programa final, el cual se puede ver en el listado \ref{prog435}.

\begin{lstlisting}[float,caption={Solución del ejercicio 4.3.5},label=prog435]
|[
    con N : int { $N \geq 0$ };
    var A : array[0..N) of int;

    var n, m, x, y, s : int;

    n, m, x, y, s, e := 0, 0, 0, 0, 0, 0;

    do $n \neq N \rightarrow$
        if $A.n = 0 \rightarrow$ skip
        |  $A.n < 0 \rightarrow$ $s$ :=
               if $(s-1) \neq (s-1)$ $max$ $0 \rightarrow$ s := 0, e := n+1
               |  $(s-1) = (s-1)$ $max$ $0 \rightarrow$ s := s-1
               fi
        |  $A.n > 0 \rightarrow$ $s$ := $(s+1)$
        fi;
        if $m = m$ $max$ $s \rightarrow$ skip
        |  $m \neq m$ $max$ $s \rightarrow$ m, x, y := s, e, n+1
        fi
    od
]|
\end{lstlisting}

% Pregunta 8
\section{Ejercicio 4.4.2}

El caso base para G.n es $n = N$. Por rango vacío de la sumatoria tenemos

$$
G.N = 0
$$

Para el caso recursivo voy a intentar relacionar $G.(n-1)$ con $G.(n)$ 
teniendo $0 < n \leq N$.

\begin{align*}
 & G.(n-1) \\
 = & <\text{definición de G.n}> \\
 & (\sum i : n-1 \leq i < N : f.i * X^{i-(n-1)}) \\
 = & <\text{separación de rango con } n \leq N> \\
 & f.(n-1) * X^{(n-1)-(n-1)} + (\sum i : n \leq i < N : f.i * X^{i-(n-1)}) \\
 = & <\text{aritmética y definición de potencia}> \\
 & f.(n-1) + (\sum i : n \leq i < N : f.i * X^{i-n}*X) \\
 = & <\text{distributividad de multiplicación sobre suma}> \\
 & f.(n-1) + X*(\sum i : n \leq i < N : f.i * X^{i-n}) \\
 = & <\text{definición de G.n}> \\
 & f.(n-1) + X*G.n \\
\end{align*}

Viendo que $r = G$.$0$ es equivalente a la postcondición solicitada, voy a derivar el
programa usando un ciclo. Introduzco la variable $n$ y pongo como invariante

$$
 r = G.n \wedge 0 \leq n \leq N
$$

y como variante $n$. La instrucción de inicialización es $n, r := N, 0$. Al intentar
reducir $n$ en 1 y suponiendo $n \neq 0$ tenemos

\begin{align*}
 & G.(n-1) \\
 = & <\text{visto previamente}> \\
 & f.(n-1) + X*G.n \\
 = & <\text{invariante}> \\
 & f.(n-1) + X*r \\
\end{align*}

Con esto puedo escribir el programa final (listado \ref{prog442}).

\begin{lstlisting}[float,caption={Solución del ejercicio 4.4.2},label=prog442]
|[
    con N, X : int { $N \geq 0$ }; f : array[0..N) of int;

    var r, n : int;

    n, r := N, 0;

    {inv $r = G.n$}
    {bound $n$}
    do $n \neq 0 \rightarrow$
        n, r := n-1, f.(n-1) + X*r
    od
]|
\end{lstlisting}

% Pregunta 9
\section{Ejercicio 10.2.0}

Introduzco la variable $m$ para derivar un programa con postcondición

$$
0 \leq m \leq N \wedge (\forall i : 0 \leq i < m : h.i \leq 0) \wedge (\forall i : m \leq i < N : h.i \geq 0)
$$

lo que satisfaría la postcondición de existencia de $p$ dada en el enunciado
del problema. Intento derivar el programa usando un lazo y la técnica de
sustitución de constante por variable. Las invariantes del lazo serían

$$
P_0: 0 \leq m \leq n \leq N
$$

$$
P_1: (\forall i : 0 \leq i < m : h.i \leq 0) \wedge (\forall i : m \leq i < n : h.i \geq 0)
$$

La instrucción de inicialización $m, n := 0, 0$ satisface las invariantes ya que $N \geq 0$ y
las cuantificaciones universales en $P_1$ quedan con rango vacío. El guardia $B$ del lazo será
$n \neq N$. Es claro que $P_0 \wedge P_1 \wedge \neg B$ implica la postcondición.
Al incrementar $n$ en $1$ podemos considerar el elemento $h.n$. Hay dos
posibilidades (no disjuntas): $h.n \leq 0$ y $h.n \geq 0$. Se puede ver una solución parcial en
el listado \ref{prog1020_parcial}.

\begin{lstlisting}[float,caption={Solución parcial del ejercicio 10.2.0},label=prog1020_parcial]
|[
    con N : int { $N \geq 0$ };
    var h : array[0..N) of int;

    var m, n : int;

    m, n := 0, 0;

    {inv $P_0 \wedge P_1$}
    {bound $N-n$}
    do $n \neq N \rightarrow$
        if $h.n \leq 0 \rightarrow$ $S_a$
        |  $h.n \geq 0 \rightarrow$ $S_b$
        fi
    od
]|
\end{lstlisting}

Ataco primero $S_b$ por ser el caso más fácil. Tenemos

\begin{align*}
 & (P_0 \wedge P_1)(n := n + 1) \\
 \equiv & <\text{n no ocurre en } P_0\text{, sustitución y definición de } P_1> \\
 & P_0 \wedge (\forall i : 0 \leq i < m : h.i \leq 0) \wedge (\forall i : m \leq i < n+1 : h.i \geq 0) \\
 \equiv & <\text{separación de rango con } n+1 > n \geq 0> \\
 & P_0 \wedge (\forall i : 0 \leq i < m : h.i \leq 0) \wedge (\forall i : m \leq i < n : h.i \geq 0) \wedge h.n \geq 0 \\
 \equiv & <\text{definición de } P_1> \\
 & P_0 \wedge P_1 \wedge h.n \geq 0
\end{align*}

Por lo tanto, dados que la invariante y el segundo guardia
son ciertos, basta con ejecutar $n := n + 1$ para mantener la invariante con
$n$ incrementada en $1$. Para conseguir $S_a$ considero la operación
$swap.m.n$, la cual no afecta la primera parte de $P_1$.
De la segunda parte de $P_1$ se infiere que
$m < n \Rightarrow h.m \geq 0$. Como suponemos que se mantiene $P_0$, eso
es equivalente a $m = n \vee h.m \geq 0$. En el primer caso, $m = n$ y
entonces

\begin{align*}
 & \{P_0 \wedge P_1 \wedge m = n \wedge h.n \leq 0\} \\
 & swap.m.n \\
 & \{P_0 \wedge (\forall i : 0 \leq i < m : h.i \leq 0) \wedge m = n \wedge h.m \leq 0\} \\
 & ;m, n := m+1, n+1 \\
 & \{P_0 \wedge (\forall i : 0 \leq i < m : h.i \leq 0) \wedge m = n\} \\
\end{align*}

Tenemos que

\begin{align*}
 & \{P_0 \wedge (\forall i : 0 \leq i < m : h.i \leq 0) \wedge m = n\} \\
 \Rightarrow & <\text{cuantificación universal con rango vacío, true es neutro de } \wedge> \\
 & \{P_0 \wedge (\forall i : 0 \leq i < m : h.i \leq 0) \wedge (\forall i : m \leq i < n : h.i \geq 0)\} \\
 \equiv & <\text{definición de } P_1> \\
 & \{P_0 \wedge P_1\} \\
\end{align*}

Por lo tanto, si $m = n$, es admisible que $S_a = swap.m.n ; m, n := m+1, n+1$. Queda ver el
segundo caso, en el que se mantiene que $h.m \geq 0$. Sean $P_{11}$ la primera parte de $P_1$
y $P_{12}$ la segunda. Entonces

\begin{align*}
 & \{P_0 \wedge P_1 \wedge h.m \geq 0 \wedge h.n \leq 0\} \\
 & swap.m.n \\
 & \{P_0 \wedge P_{11} \wedge h.m \leq 0 \wedge (\forall i : m + 1 \leq i < n : h.i \geq 0) \wedge h.n \geq 0\} \\
 & ;m, n := m+1, n+1 \\
 & \{P_0 \wedge P_{11} \wedge P_{12}\} \\
\end{align*}

La misma instrucción que cumple con lo necesario en el primer caso también sirve en el segundo.
Habiendo entonces resuelto $S_a$ se obtiene el programa final (listado \ref{prog1020}).

\begin{lstlisting}[float,caption={Solución del ejercicio 10.2.0},label=prog1020]
|[
    con N : int { $N \geq 0$ };
    var h : array[0..N) of int;

    var m, n : int;

    m, n := 0, 0;

    {inv $P_0 \wedge P_1$}
    {bound $N-n$}
    do $n \neq N \rightarrow$
        if $h.n \leq 0 \rightarrow$ swap.m.n ; m, n := m+1, n+1
        |  $h.n \geq 0 \rightarrow$ n := n+1
        fi
    od
]|
\end{lstlisting}

\end{document}
