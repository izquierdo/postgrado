\section{Planteamiento del problema}

Bajo la sombra de la Web Semántica y con el apoyo de arquitecturas orientadas a
servicios, el número de fuentes de datos y servicios Web ha explotado en los
últimos años. Por ejemplo, actualmente la colección de bases de datos de
biología molecular incluye 1.170 bases de datos~\cite{Galperin09}, lo cual es 95
más que el año pasado~\cite{Galperin2008} y 110 más que hace dos
años~\cite{Galperin2007}. Tanto las herramientas y servicios como el número de
instancias publicadas por estos recursos siguen una progresión
similar~\cite{Benson07}. Gracias a este tesoro de datos, los usuarios se apoyan
más en varias tareas digitales como obtención de datos de fuentes públicas y
análisis de datos con herramientas o servicios Web organizados en flujos de
trabajo complejos.

Para diseñar una aplicación, los usuarios frecuentemente especifican la
funcionalidad en términos de un flujo de trabajo de servicios abstractos,
mientras que la calidad deseada se define restringiendo los valores de los
parámetros QoS (calidad de servicio) que describen los servicios concretos
disponibles. Para evaluar el flujo propuesto, se instancian los servicios
abstractos con concretos de manera que se satisfagan las restricciones sobre los
parámetros QoS. Sin embargo, el rendimiento de estas tareas es impactado por el
número de fuentes disponibles. Se requieren técnicas que permitan escalar
eficientemente a un número grande de servicios.

En este trabajo se intenta proponer una solución a este problema y proveer
técnicas que recorran eficientemente el espacio de servicios disponibles y
seleccionen combinaciones que cumplan con los requerimientos funcionales y
no-funcionales que describen una aplicación diseñada. Se llama 
\mcdsatc a la solución propuesta.

En \mcdsatc, las funcionalidades de servicios concretos se describen
como vistas sobre servicios abstractos y los flujos de trabajo se representan
como consultas conjuntivas sobre servicios abstractos; esta representación de
los servicios es similar a la generada de manera semiautomática para el
sistema {\cal DEIMOS}~\cite{AmbiteISWC09}. Adicionalmente, las reglas
que definen flujos de trabajo y servicios concretos en el catálogo de
\mcdsatc son creadas de manera que las restricciones de entrada y salida
de los servicios abstractos combinados sean satisfechas. Para representar las
medidas de QoS, cada descripción de servicio concreto es anotada con un número
real que representa un valor general para las medidas de QoS del servicio en
particular. Finalmente, \mcdsatc modela el problema de selección de
servicios web como el problema de reescritura de consultas usando vistas
\cite{halevy:survey}. Este problema es importante en el contexto de integración
de datos \cite{Chen05,JaudoinPRST05} y de optimización de consultas y
mantenimiento de datos \cite{AfratiLU07,levy:bucket} y se han definido varias
aproximaciones para escalar eficientemente a un número grande de
vistas~\cite{arvelo:aaai06,pods:DuschkaG97,sac:DuschkaG97,levy:bucket,pottinger:minicon}.
En este trabajo se propone la infraestructura \mcdsatc que extiende la
aproximación propuesta en  ~\cite{arvelo:aaai06}, con la capacidad de trabajar
con constantes y de identificar la composición de servicios que maximice una
función de utilidad dada.

\mcdsatc  traduce una instancia del problema de selección de servicios
en el problema de enumerar los modelos de una teoría proposicional que codifica
información sobre el flujo de trabajo, los servicios concretos y las condiciones
que deben ser satisfechas para generar una composición que maximice la función
de utilidad. Esta teoría se representa como una fórmula CNF que es compilada a
su representación d-DNNF, a partir de la cual los modelos que maximicen la
función global de utilidad se obtienen en tiempo lineal usando técnicas modernas
de SAT~\cite{darwiche:dnnf}. Se estudia empíricamente la calidad de \mcdsatc
sobre una variedad de \emph{benchmarks} y se observa que la técnica
propuesta provee una solución eficiente al problema de selección de servicios
que es capaz de escalar a flujos de trabajo grandes así como a números de vistas
grandes.

This paper is comprised of six additional sections. The next section motivates
our approach by using an example, and  we summarize existing related work in
section 3. Section 4 describes the \mcdsatc system architecture, and  the experimental study is reported in section 5. 
In section 6 we present the formalization of the web service selection problem as a Propositional Logic theory, and finally, section 7 outlines our conclusions and future work.

\section{Motivación}

Consideremos los servicios abstractos y concretos en la tabla
~\ref{table:services} que representan aplicaciones en el dominio de vuelos.

\begin{table}
\begin{tabular}{| l| l |}
\hline
Servicio Abstracto & Descripción \\
\hline
$fight(x1,x2)$ & relaciona las ciudades x1 y x2 si hay un vuelo entre ellas\\
\hline
$uscity(x1)$ &  determina si una ciudad x1 es una ciudad de Estados Unidos\\
\hline \hline
Servicio Concreto & Descripción \\
\hline
$national(x1,y1)$&  retorna dos ciudades de Estados Unidos que están conectadas
por vuelos directos \\
\hline
$oneway(x1,y1)$ & retorna dos ciudades que están conectadas por vuelos directos
de ida\\
\hline
$onestop(x1,y1)$ & retorna dos ciudades que están conectadas por vuelos con una
escala\\
\hline
$ flightToCCS(x1,"CCS")$ & retorna una ciudad que está conectada a Caracas por
un vuelo directo\\
 \hline
$ onestopToCCS(x1,y1)$ & retorna dos ciudades x1 y y1 tales que hay un vuelo con
una escala desde la ciudad x1 hasta Caracas con parada en la ciudad y1\\
 \hline
 $fromWAS("WAS",y1)$ &  retorna una ciudad y1 a la cual llegan vuelos desde
Washington.\\
 \hline
\end{tabular}
\label{table:services}
\caption{Servicios Abstractos y Concretos}
\end{table}

Usando la aproximación \emph{Local As View} (LAV) ~\cite{AmbiteISWC09}, los
servicios concretos en la tabla \ref{table:services} son descritos
semánticamente como vistas sobre los servicios abstractos {\it flight} y
{\it uscity} como sigue:   

\begin{alignat*}{1}
national(x1,y1)\ &\qrule\ flight(x1,y1),\,  uscity(x1),\,  uscity(y1)\,. \\
oneway(x2,y2)\ &\qrule\ flight(x2,y2)\,. \\
onestop(x3,z3)\ &\qrule\ flight(x3,y3),\, flight(y3,z3)\,. \\
flightToCCS(x2,"CCS")\ &\qrule\ flight(x2,"CCS")\,. \\
onestopToCCS(x3,y3)\ &\qrule\ flight(x3,y3),\, flight(y3,"CCS")\,. \\
fromWAS("WAS",y3)\ &\qrule\ flight("WAS",x3)\,. \\
\end{alignat*}

Ahora supongamos que un usuario está interesado en implementar un flujo de
trabajo capaz de obtener los vuelos ida y vuelta con una escala de ciudades de
Estados Unidos a cualquier ciudad en el mundo donde las escalas pueden ser
hechas en cualquier ciudad. Consideremos la siguiente consulta conjuntiva que
representa el flujo de trabajo que define esta solicitud en términos de
servicios abstractos. Este flujo fue definido de manera que los parámetros de
entrada de cada servicio abstracto están ligados a constantes o a
atributos proyectados por servicios abstractos previos.

\begin{alignat*}{1}
w(x,y1,y2,y3)\ &\qrule\ flight(x,y1),\, flight(y1,y2),\,  flight(y2,y3),\,  flight(y3,x),\,  \\
                   \ & uscity(x)\,. \\
\end{alignat*}

Las implementaciones de este flujo de trabajo corresponden a composiciones de
servicios concretos donde cada servicio concreto puede implementar un
subconjunto de los servicios abstractos del flujo de trabajo, pero cada servicio
abstracto puede ser implementado por exactamente un servicio concreto. La
siguiente composición de servicios concretos corresponde a una de las XX
implementaciones del flujo de trabajo.

\begin{alignat*}{1}
wi1(x,y1,"CCS",y3)\ &\qrule\ national(x,y1),\, flightToCCS(y1,"CCS"),\,  oneway("CCS",y3),\,  \\
\ & national(y3,x)\,.  \\
\end{alignat*}

Sin embargo, nótese que las siguientes composiciones no son válidas:

\begin{alignat*}{1}
wi2(x,y1,"CCS"y3)\ &\qrule\ national(x,y1),\, flightToCCS(y1,"CCS"),\, \\
\ & fromWAS("WAS",y3),\,  national(y3,x)\,. \\
wi3(x,y1,y2,y3)\ &\qrule\ national(x1,y1),\, onestop(x1,y2),\,  oneway(y2,y3),\,  \\
\ & national(y3,x)\,. \\
\end{alignat*}

La composición {\it wi2} no es válida porque asocia la variable del flujo de
trabajo {\it y2} a las constantes {\it ``CCS''} y {\it ``WAS''} que denotan
ciudades
distintas. Por otra parte, la composición {\it wi3} no implementa el flujo de
trabajo porque el servicio concreto {\it onestop} no recibe como entrada o
produce como salida la ciudad donde el vuelo se detiene (representada por la
variable {\it y1}); por esto, no es posible garantizar que la ciudad donde el
vuelo se detiene y la ciudad final obtenida por el servicio {\it national} sean
las mismas.

Para evitar producir implementaciones del flujo de trabajo inválidas, se
requiere manejar constantes de manera que constantes distintas no sean asociadas
entre sí directa o indirectamente (por transitividad). Además, todos los
atributos en la salida del flujo o necesarios para unir otros servicios en el
flujo de trabajo deben ser producidos por el servicio concreto seleccionado.

Adicionalmente, supongamos que cada servicio concreto está anotado con una
función de utilidad que agrega los valores de parámetros diferentes de QoS que
caracterizan el comportamiento del servicio. Entonces, como el espacio de
composiciones de servicios válidas de un flujo abstracto puede ser muy grande,
se requiere una técnica capaz de identificar aquellos que maximicen la función
de utilidad sin tener que enumerar el conjunto completo de composiciones
válidas. En este trabajo se propone una aproximación proposicional basada en
lógica que toma provecho del poder de los \emph{SAT solvers} modernos para
enumerar eficientemente las composiciones de servicios que corresponden a
implementaciones válidas de un flujo de trabajo dado, así como las composiciones
que maximizan una función de utilidad dada.

