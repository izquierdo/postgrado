\section{Justificación e Importancia}

\subsection{Ejemplo Motivante}

% flights
\newcommand{\flight}{\text{\it flight}}
\newcommand{\UScity}{\text{\it uscity}}
\newcommand{\national}{\text{\it national}}
\newcommand{\oneway}{\text{\it one-way}}
\newcommand{\onestop}{\text{\it one-stop}}
\newcommand{\flightPA}{\text{\it flight-to-pa}}
\newcommand{\onestopPA}{\text{\it onestop-to-pa}}
\newcommand{\fromNY}{\text{\it from-ny}}
\newcommand{\PA}{\text{PA}}
\newcommand{\NY}{\text{NY}}
\newcommand{\AL}{\text{AL}}
% end flights

Considere un sistema simple de información de vuelos que contiene información
sobre vuelos entre ciudades e información sobre qué ciudades están en los
Estados Unidos. Un sistema así puede ser descrito usando LAV con los dos servicios
abstractos $\flight(x,y)$ y $\UScity(x)$. El primero relaciona dos ciudades $x$ e $y$ si
hay un vuelo directo entre ellas, y el segundo dice si $x$ es una ciudad de
E.E.U.U.  o no. Para los servicios concretos, supongamos que las fuentes de
datos disponibles en Internet contienen la siguiente información:

\begin{enumerate}[--]
\item $\national(x,y)$ relaciona dos ciudades de E.E.U.U. conectadas por un vuelo directo,
\item $\oneway(x,y)$ relaciona dos ciudades conectadas por un vuelo de ida,
\item $\onestop(x,y)$ relaciona dos ciudades conectadas por un vuelo con una escala,
\item $\flightPA(x)$ dice si hay un vuelo directo desde $x$ a París,
\item $\onestopPA(x,y)$ relaciona $x$ y $y$ si hay un vuelo desde $x$ a París con una escala en $y$, y
\item $\fromNY(x)$ dice si hay un vuelo de Nueva York a $x$.
\end{enumerate}

Adicionalmente, los servicios concretos se describen con los siguientes
servicios abstractos:

\begin{alignat*}{1}
\national(x,y)\   &\qrule\ \flight(x,y),\,\UScity(x),\,\UScity(y)\,. \\
\oneway(x,y)\     &\qrule\ \flight(x,y)\,. \\
\onestop(x,z)\    &\qrule\ \flight(x,y),\,\flight(y,z)\,. \\
\flightPA(x)\     &\qrule\ \flight(x,\PA)\,. \\
\onestopPA(x,y)\  &\qrule\ \flight(x,y),\,\flight(y,\PA)\,. \\
\fromNY(x)\       &\qrule\ \flight(\NY,x)\,.
\end{alignat*}

Ahora supongamos que un usuario está interesado en construir un flujo de trabajo
capaz de recuperar los vuelos ida y vuelta con una parada hacia cualquier ciudad
en el mundo, tales que los vuelos puedan detenerse en cualquier ciudad.

\[ W(x,w,y,z) \qrule \UScity(x),\,\flight(x,w),\,\flight(w,y),\,\flight(y,z),\,\flight(z,x)\,. \]

La siguiente consulta conjuntiva representa el flujo de trabajo que define esta
solicitud en términos de servicios abstractos. El flujo de trabajo es definido
de manera que todo lo relacionado sobre el ligamiento de parámetros de
entrada/salida es resuelto. Así, cualquier instanciación de los servicios
abstractos en términos de los servicios concretos es una implementación válida
del flujo de trabajo. Las implementaciones corresponden a composiciones de
servicios concretos en las cuales un servicio concreto puede implementar uno o
más servicios abstractos del flujo, pero cada servicio abstracto puede ser
implementado por exactamente un servicio concreto. Por ejemplo, la siguiente
composición corresponde a una de estas implementaciones.

\[ I(x,w,y,z)\ \qrule\ \national(x,w),\,\flightPA(w),\,\oneway(\PA,z),\,\national(z,x)\,. \]

Sin embargo, las siguientes dos composiciones no son válidas.

\begin{alignat*}{1}
I'(x,w,y,z)\  \qrule\ &\national(x,y),\,\flightPA(y),\,\fromNY(z),\,\national(z,x)\,. \\
I''(x,w,y,z)\ \qrule\ &\onestop(x,y),\,\oneway(y,z),\,\national(z,x)\,.
\end{alignat*}

La primera composición no es válida porque asocia la variable del flujo $y$ a
las constantes \PA\ y \NY\ que denotan ciudades diferentes. En la otra mano, $I''$
no lo implementa porque el servicio concreto $\onestop(x,y)$ no recibe como entrada
o produce como salida la ciudad intermedia donde el vuelo se detiene y entonces
no es posible asegurarse de que esa ciudad esté asociada a la ciudad $w$ que es
retornada por el flujo de trabajo.

Estos ejemplos muestran que las instanciaciones correctas de un flujo deben
manejar constantes de manera que dos constantes distintas no sean asociadas
entre sí de manera directa o indirectamente vía transitividad, y que todos los
atributos que aparezcan en un \emph{join} o en la salida sean producidos por los
servicios concretos seleccionados.

Los parámetros QoS son modelados anotando los servicios concretos con utilidades
que caracterizan su comportamiento y que luego son agregados durante la
instanciación. Así, como se mencionó previamente, la mejor instanciación es la
que minimiza (o maximiza) la agregación de utilidades.

\subsection{Complejidad del Problema}

Estamos tratando con un problema difícil \cite{lenzerini:dataintegration}. 
El cuadro~\ref{cqcomplexity} ilustra la complejidad de distintos tipos de
problemas de resolución de consultas basadas en vistas. El caso particular que
se está tratando en este trabajo es CQ con búsqueda exacta, el cual está en
coNP.

\begin{table}
\begin{center}
  \begin{tabular}{|c||c|c|c|c|c|}
\hline
Sound  &  CQ  & CQ$\neq$  & PQ  &Datalog &FOL \\
\hline
  CQ  & PTIME &coNP  &PTIME &PTIME  &undec. \\
 CQ$\neq$  & PTIME &coNP  &PTIME &PTIME  &undec. \\
  PQ  &  coNP &coNP  & coNP & coNP  &undec. \\
Datalog &coNP &undec.& coNP &undec. &undec. \\
 FOL  & undec.&undec.&undec.&undec. &undec. \\
\hline
 Exact  & CQ &  CQ$\neq$     &  PQ  &Datalog &FOL \\
\hline
  CQ    &coNP &coNP  & coNP & coNP  &undec. \\
 CQ$\neq$    &coNP &coNP  & coNP & coNP  &undec. \\
  PQ    &coNP &coNP  & coNP & coNP  &undec. \\
Datalog&undec.&undec.&undec.&undec. &undec. \\
 FOL   &undec.&undec.&undec.&undec. &undec. \\
\hline
  \end{tabular}
  \caption{Complejidad de responder consultas basadas en vistas (obtenido de
\cite{lenzerini:dataintegration})}
  \label{cqcomplexity}
\end{center}
\end{table}
