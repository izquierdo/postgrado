\begin{titlepage}

\begin{flushleft}
\includegraphics[scale=0.3]{logo_usb.png} \\
Universidad Simón Bolívar \\
Coordinación de Ciencias de la Computación \\
Maestría en Ciencias de la Computación \\
\end{flushleft}

\vspace{0.8cm}

\begin{center}
{\large \bf \textsf{Selección de Servicios Web De Máxima Utilidad Usando Circuitos}}
\end{center}

\vspace{0.2cm}

\begin{abstract}
En las arquitecturas orientadas a servicios (SOA), los servicios concretos se
describen en términos de parámetros de funcionalidad y calidad de servicio,
mientras que las aplicaciones de \emph{software} se especifican como flujos de
trabajo de servicios abstractos y criterios no-funcionales que establecen la
funcionalidad deseada y la calidad esperada respectivamente. En tiempo de
ejecución de un flujo de trabajo, los servicios abstractos se reescriben con los
servicios concretos que cumplen con las restricciones funcionales y no
funcionales del flujo de trabajo. Dado que la selección de los servicios es
hecha al momento y el espacio de servicios concretos puede ser muy grande, se
requiere eficiencia y escalabilidad. En este trabajo se propone la
infraestructura \mcdsatc para resolver eficientemente el problema de
selección de servicios en presencia de un número grande de servicios concretos.
En primer lugar se utiliza la aproximación \emph{Local As View} para describir
la funcionalidad de los servicios concretos en términos de vistas de servicios
abstractos, la calidad como una función de utilidad general que combina los
distintos valores de calidad que describen cada servicio, y los flujos de
trabajo como consultas conjuntivas sobre servicios abstractos. Basándose en esta
representación, se modela el problema de selección de servicios como un problema
conocido en el área de integración de sistemas conocido como reescritura de
consultas usando vistas. Se codifica el problema de selección de servicios como
una teoría proposicional cuyos modelos corresponden a la composición de los
servicios que implementan el flujo abstracto; se explotan relaciones conocidas
entre teorías d-DNNF y circuitos aritméticos para proveer una solución eficiente
y escalable. Se reporta sobre el rendimiento de la aproximación propuesta y se
observa que \mcdsatc escala a instancias grandes del problema de
selección de servicios.
\end{abstract}

\vspace{\fill}

\begin{minipage}{0.3\textwidth}
\begin{flushleft}
\emph{Estudiante:} \\
Daniel \textsc{Izquierdo} \\
Carnet 08-86809 \\
\end{flushleft}
\end{minipage}
\begin{minipage}{0.3\textwidth}
\begin{center}
Fecha estimada de culminación: \\
Marzo de 2010
\end{center}
\end{minipage}
\begin{minipage}{0.3\textwidth}
\begin{flushright}
\emph{Asesor:} \\
Prof. María Esther \textsc{Vidal}
\end{flushright}
\end{minipage}

\end{titlepage}
