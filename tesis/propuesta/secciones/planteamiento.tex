\section{Planteamiento del Problema}

En el contexto de la Web Semántica, y con el soporte de las Arquitecturas Orientadas a
Servicios (SOA), el número de fuentes de datos y servicios Web ha explotado en
los últimos años. Por ejemplo, la colección de bases de datos de biología
molecular actualmente contiene 1.170 bases de datos~\cite{Galperin09}, un número
que es
mayor que el del último año por 95~\cite{Galperin2008}, y que el de hace dos
años por 110~\cite{Galperin2007};
las herramientas y servicios, así como el número de instancias
publicados por estos recursos, siguen una progresión similar~\cite{Benson07}.
Gracias a
esta variedad de datos, la tendencia de los usuarios es hacia depender cada vez
más de métodos automáticos para manejarlos tales como recuperación de datos de
fuentes públicas y análisis utilizando herramientas o servicios Web compuestos
en flujos de trabajo complejos.

La SOA típica consiste de dos capas. La capa concreta que está hecha de
servicios concretos cuyas funcionalidades son descritas en términos de pre- y
post-condiciones y propiedades no funcionales en conjuntos de parámetros de
calidad de servicio (QoS), y
la capa abstracta compuesta de aplicaciones de software cuyas funcionalidades
son descritas en términos de flujos de trabajo abstractos y propiedades no
funcionales en términos de restricciones de QoS. La ejecución de un flujo de
trabajo abstracto involucra la instanciación de los servicios abstractos en
servicios concretos que cumplan con los requerimientos funcionales y no
funcionales. Este proceso de instanciación puede ser visto como una búsqueda de
una instanciación objetivo en el espacio combinatorio de todas las
instanciaciones válidas. Entonces se está interesado en técnicas eficientes para
realizar esta búsqueda que sean capaces de escalar a medida que el número de
servicios concretos o la complejidad (número
de objetivos, tamaño) de la estructura del flujo de trabajo  aumenta. Se llama al
problema de instanciar un flujo de trabajo dado con servicios concretos de un
conjunto de servicios concretos dado, de manera que ciertas restricciones sobre QoS sean
cumplidas, el Problema de Instanciación de Flujos de Trabajo (WIP).

Bajo estas suposiciones, WIP puede ser convertido en el bien conocido Problema
de Reescritura de Consultas (QRP) para LAV que es central a los sistemas de
integración \cite{halevy:survey}. QRP consiste de una consulta conjuntiva que debe ser
respondida en términos de vistas donde la consulta y las vistas son descritas
usando LAV con relaciones abstractas. Este problema es importante en el contexto
de integración de datos \cite{Chen05,JaudoinPRST05}, y optimización de consultas y mantenimiento
de datos \cite{AfratiLU07,levy:bucket}, y varias soluciones que escalan a un número grande de
vistas han sido definidas \cite{arvelo:aaai06,pods:DuschkaG97,sac:DuschkaG97,levy:bucket,pottinger:minicon}.

La solución propuesta por Arvelo et al. \cite{arvelo:aaai06} está basada en la enumeración
eficiente de modelos de una teoría lógica proposicional. Dada una consulta conjuntiva
y un conjunto de fuentes definidas como vistas, se construye una teoría
lógica de manera que cada modelo de la teoría codifica una
reescritura válida, y así todas las reescrituras son obtenidas enumerando los
modelos de la teoría. Esta enumeración puede ser realizada eficientemente si la
teoría lógica está en cierta forma normal llamada la forma normal de negación
determinística y descomponible (d-DNNF) \cite{darwiche:d-dnnfs}. Así,
la solución consiste en transformar (llamado compilar en el campo de
compilación del conocimiento) la teoría lógica en el formato d-DNNF para
enumerar sus modelos eficientemente.

Pero las teorías d-DNNF no sólo soportan la enumeración eficiente de sus
modelos, sino también otras operaciones. Entre ellas se encuentra la enumeración
de los modelos con mejor calidad. Dada una función de calidad de literales
$r(\ell)$ que asigne calidades a cada literal $\ell$, se define la calidad
$r(\omega)$ de un modelo $\omega$ como la suma de las calidades de los literales
activados por $\omega$ (es decir,
$r(\omega)\doteq\sum_{\omega\vDash\ell}r(\ell)$), y se dice que $\omega$ es un modelo
de mejor calidad si no hay modelo $\omega'$ tal que $r(\omega')<r(\omega)$.

Dada una teoría en d-DNNF, se computa la calidad de los mejores modelos, y
los mejores modelos, en tiempo lineal en el tamaño del d-DNNF. Este cómputo
transforma el GAD del d-DNNF en un circuito aritmético reemplazando los nodos
AND con `+' y los nodos OR con `min'. La función de calidad de literales
asigna valores a las hojas del circuito aritmético (grafo acíclico dirigido usado
para calcular polinomios) que son propagados a la raíz en tiempo
lineal. El valor de la raíz es la calidad del mejor modelo \cite{darwiche:weighted}.

En este trabajo se propone explotar las propiedades de los d-DNNF construyendo una
teoría lógica cuyos modelos codifican las instanciaciones del flujo de trabajo
y cuyos mejores modelos codifican las instanciaciones óptimas (mejores). Así, la
búsqueda combinatoria se reduce a la computación de un mejor modelo de una
teoría lógica que puede ser realizada eficientemente una vez que la teoría es
transformada al formato d-DNNF.

\subsection{Ejemplo Motivante}

% flights
\newcommand{\vuelo}{\text{\it vuelo}}
\newcommand{\ciudadVE}{\text{\it ciudadVE}}
\newcommand{\nacional}{\text{\it nacional}}
\newcommand{\ida}{\text{\it ida}}
\newcommand{\unaescala}{\text{\it una-escala}}
\newcommand{\vueloCCS}{\text{\it vuelo-hacia-ccs}}
\newcommand{\unaparadaCCS}{\text{\it unaescala-hacia-ccs}}
\newcommand{\desdeVAL}{\text{\it desde-val}}
\newcommand{\PA}{\text{CCS}}
\newcommand{\NY}{\text{VAL}}
\newcommand{\AL}{\text{AL}}
% end flights

Considere un sistema simple de información de vuelos que contiene información
sobre vuelos entre ciudades e información sobre qué ciudades están en los
Estados Unidos. Un sistema así puede ser descrito usando LAV con los dos servicios
abstractos $\vuelo(x,y)$ y $\ciudadVE(x)$. El primero relaciona dos ciudades $x$ e $y$ si
hay un vuelo directo entre ellas, y el segundo dice si $x$ es una ciudad de
Venezuela  o no. Para los servicios concretos, supongamos que las fuentes de
datos disponibles en Internet contienen la siguiente información:

\begin{enumerate}[--]
\item $\nacional(x,y)$ relaciona dos ciudades de Venezuela conectadas por un vuelo directo,
\item $\ida(x,y)$ relaciona dos ciudades conectadas por un vuelo de ida,
\item $\unaescala(x,y)$ relaciona dos ciudades conectadas por un vuelo con una escala,
\item $\vueloCCS(x)$ dice si hay un vuelo directo desde $x$ a Caracas,
\item $\unaparadaCCS(x,y)$ relaciona $x$ y $y$ si hay un vuelo desde $x$ a Caracas con una escala en $y$, y
\item $\desdeVAL(x)$ dice si hay un vuelo de Valencia a $x$.
\end{enumerate}

Adicionalmente, los servicios concretos se describen con los siguientes
servicios abstractos:

\begin{alignat*}{1}
\nacional(x,y)\   &\qrule\ \vuelo(x,y),\,\ciudadVE(x),\,\ciudadVE(y)\,. \\
\ida(x,y)\     &\qrule\ \vuelo(x,y)\,. \\
\unaescala(x,z)\    &\qrule\ \vuelo(x,y),\,\vuelo(y,z)\,. \\
\vueloCCS(x)\     &\qrule\ \vuelo(x,\PA)\,. \\
\unaparadaCCS(x,y)\  &\qrule\ \vuelo(x,y),\,\vuelo(y,\PA)\,. \\
\desdeVAL(x)\       &\qrule\ \vuelo(\NY,x)\,.
\end{alignat*}

Ahora supongamos que un usuario está interesado en construir una aplicación
capaz de recuperar los vuelos ida y vuelta con una parada hacia cualquier ciudad
en el mundo, tales que los vuelos puedan detenerse en cualquier ciudad.

\[ W(x,w,y,z) \qrule \ciudadVE(x),\,\vuelo(x,w),\,\vuelo(w,y),\,\vuelo(y,z),\,\vuelo(z,x)\,. \]

La siguiente consulta conjuntiva representa el flujo de trabajo que define esta
solicitud en términos de servicios abstractos. El flujo de trabajo es definido
de manera que todo lo relacionado sobre el ligamiento de parámetros de
entrada/salida es resuelto. Así, cualquier instanciación de los servicios
abstractos en términos de los servicios concretos es una implementación válida
del flujo de trabajo. Las implementaciones corresponden a composiciones de
servicios concretos en las cuales un servicio concreto puede implementar uno o
más servicios abstractos del flujo, pero cada servicio abstracto puede ser
implementado por exactamente un servicio concreto. Por ejemplo, la siguiente
composición corresponde a una de estas implementaciones.

\[ I(x,w,y,z)\ \qrule\ \nacional(x,w),\,\vueloCCS(w),\,\ida(\PA,z),\,\nacional(z,x)\,. \]

Sin embargo, las siguientes dos composiciones no son válidas.

\begin{alignat*}{1}
I'(x,w,y,z)\  \qrule\ &\nacional(x,y),\,\vueloCCS(y),\,\desdeVAL(z),\,\nacional(z,x)\,. \\
I''(x,w,y,z)\ \qrule\ &\unaescala(x,y),\,\ida(y,z),\,\nacional(z,x)\,.
\end{alignat*}

La primera composición no es válida porque asocia la variable del flujo $y$ a
las constantes \PA\ y \NY\ que denotan ciudades diferentes. En la otra mano, $I''$
no lo implementa porque el servicio concreto $\unaescala(x,y)$ no recibe como entrada
o produce como salida la ciudad intermedia donde el vuelo se detiene y entonces
no es posible asegurarse de que esa ciudad esté asociada a la ciudad $w$ que es
retornada por el flujo de trabajo.

Estos ejemplos muestran que las instanciaciones correctas de un flujo deben
manejar constantes de manera que dos constantes distintas no sean asociadas
entre sí de manera directa o indirectamente vía transitividad, y que todos los
atributos que aparezcan en un \emph{join} o en la salida sean producidos por los
servicios concretos seleccionados.

Los parámetros QoS son modelados anotando los servicios concretos con utilidades
que caracterizan su comportamiento y que luego son agregados durante la
instanciación. Así, como se mencionó previamente, la mejor instanciación es la
que minimiza (o maximiza) la agregación de utilidades.

\subsection{Propuesta}

En este trabajo se considera una versión restringida de WIP que adopta el
enfoque Local-As-View (LAV) \cite{levy:bucket}. En LAV, todos los elementos de un
problema son especificados con un lenguaje común basado en servicios abstractos
tal que los servicios concretos se describen como vistas de servicios
abstractos, la calidad de una instanciación como una función de utilidad global
que combina los diferentes parámetros QoS, y el flujo de trabajo abstracto como
una consulta conjuntiva sobre los servicios abstractos; esta representación es
similar a la que es generada de manera semiautomática para el sistema DEIMOS
\cite{AmbiteISWC09}. En esta versión de WIP, las reglas que definen flujos de trabajo
(consultas conjuntivas) y servicios concretos (vistas) son creados de manera que
todas las restricciones funcionales sobre pre- y post-condiciones de los
servicios y sus combinaciones sean satisfechas, y las medidas de QoS son
representadas anotando cada descripción de servicio concreto con un número real
que representa la utilidad QoS global del servicio.

El resto de esta propuesta está distribuido como sigue. En la sección 2 se
discute trabajo previo relacionado a la propuesta y conseguido en la literatura
actual. En la sección 3 se justifica el trabajo sobre este problema. En la
sección 4 se exponen los objetivos del trabajo. En la sección 5 se propone una
metodología para alcanzar los objetivos. En la sección 6 se presenta un
cronograma de actividades que permitan desarrollar el trabajo. Por último, en la
sección 7 se dan a conocer los resultados del trabajo ya realizado.
