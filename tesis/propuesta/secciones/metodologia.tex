\section{Metodología Propuesta}

La fase inicial de este trabajo consiste en realizar una investigación sobre el
problema a ser tratado. Se debe conocer el estado del arte en cuanto a resultados obtenidos
relacionados al problema y las líneas de investigación que se están siguiendo.

Una vez conocidas las propuestas actuales se procede a identificar sus
fortalezas y especialmente sus limitaciones. Se analizarán las causas de
dichas limitaciones y se propondrán maneras de superarlas. Las propuestas deben
estar acompañadas de un basamento formal que permita demostrar su correctitud y
deben ser implementables en un lapso razonable de manera de evaluar su
rendimiento y su capacidad de resolver el problema.

Luego comenzará la implementación de la nueva solución. Para ello se podrá
utilizar como punto de partida sistemas existentes como el propuesto por Arvelo
et al. \cite{arvelo:aaai06} y modificarlos para mejorar
sus características siguiendo las ideas propuestas.

La finalización del proceso de implementación dará paso a la fase de evaluación de
la propuesta. Se propone realizar series de experimentos diversos sobre el
producto final que ilustren sus capacidades y rendimiento. En casos en
que soluciones ya existentes tengan las mismas capacidades, se debe comparar el
rendimiento de ambas sobre una variedad de casos de prueba que persigan revelar
sus peores comportamientos. Esto permitirá determinar empíricamente las virtudes
y desventajas de la solución propuesta.
