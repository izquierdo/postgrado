\documentclass{article}

\usepackage[utf8x]{inputenc}
\usepackage[spanish]{babel}

\begin{document}

\title{Instanciación Óptima de Flujos de Trabajo Abstractos Usando Lógica y Circuitos}
\maketitle

\section{Resumen}

La Arquitectura Orientada a Servicios típica consiste de dos capas principales.
La capa concreta con servicios concretos cuyas funcionalidades son descritas en
términos de pre- y post-condiciones y propiedades no funcionales en términos de
parámetros de calidad de servicio (Quality Of Service, QoS), y la capa abstracta
con aplicaciones de software cuyas funcionalidades son descritas en términos de
flujos de trabajo abstractos y propiedades no funcionales en términos de
restricciones sobre QoS. Durante la ejecución de un flujo de trabajo, los
servicios abstractos son instanciados en servicios concretos que cumplen con los
requerimientos funcionales y no funcionales. Esta instanciación, que debe ser
hecha en el momento, consiste de una búsqueda en un espacio de posibilidades
combinatorio. En este trabajo se propone una infraestructura para resolver
eficientemente el problema de instanciación que está firmemente basada en
lógica. La infraestructura adopta la aproximación Local-As-View en la cual la
funcionalidad de servicios concretos se describe usando vistas de servicios
abstractos, la calidad de una instanciación como una función de utilidad global
que combina los diferentes parámetros QoS, y los flujos de trabajo abstractos
como consultas conjuntivas sobre servicios abstractos. Usando esta
representación, el problema de instanciación de flujos de trabajo se convierte
en un problema de reescritura de consultas, del área de sistemas de integración.
Entonces, construyendo sobre trabajo relacionado, se idea una codificación del
problema de instanciación de flujos de trabajo como una teoría lógica cuyos
modelos están en correspondencia con las instanciaciones del flujo de trabajo, y
los modelos mejor puntuados en correspondencia con instanciaciones óptimas del
flujo de trabajo. Así, explotando propiedades conocidas de teorías lógicas en
formato d-DNNF, se provee una solución eficiente y escalable al problema de
instanciación de flujos de trabajo. Esta aproximación no sólo escala a
instancias grandes como muestran resultados experimentales ya realizados, sino
que también es correcta y completa dado que, estándo basada en la lógica, es
amigable al análisis formal.

\section{Introducción}

Bajo la sombra de la Web Semántica, y soportado por Arquitecturas Orientadas a
Servicios (SOA), el número de fuentes de datos y servicios Web ha explotado en
los últimos años. Por ejemplo, la colección de bases de datos de biología
molecular actualmente contiene 1170 bases de datos XREF22W, un número que es
mayor que el del último año XREF21W por 95, y que el de hace dos años por 110
XREF20W; las herramientas y servicios, así como el número de instancias
publicados por estos recursos, siguen una progresión similar XREF6W. Gracias a
este tesoro de datos, la tendencia de los usuarios es hacia depender cada vez
más en métodos automáticos para manejarlos tales como recuperación de datos de
fuentes públicas y análisis utilizando herramientas o servicios Web compuestos
en flujos de trabajo complejos.

La SOA típica consiste de dos capas. La capa concreta que está hecha de
servicios concretos cuyas funcionalidades son descritas en términos de pre- y
post-condiciones y propiedades no funcionales en términos de parámetros QoS, y
la capa abstracta compuesta de aplicaciones de software cuyas funcionalidades
son descritas en términos de flujos de trabajo abstractos y propiedades no
funcionales en términos de restricciones de QoS. La ejecución de un flujo de
trabajo abstracto involucra la instanciación de los servicios abstractos en
servicios concretos que cumplen con los requerimientos funcionales y no
funcionales. Este proceso de instanciación puede ser visto como una búsqueda de
una instanciación objetivo en el espacio combinatorio de todas las
instanciaciones válidas. Así, uno está interesado en técnicas eficientes para
realizar esta búsqueda que sean capaces de escalar a medida que el número de
servicios concretos o la complejidad del flujo de trabajo aumenta. Se llama al
problema de instanciar un flujo de trabajo dado con servicios concretos, de un
conjunto dado de servicios concretos, de manera que ciertas demandas de QoS sean
cumplidas, el Workflow Instantiation Problem (WIP, "Problema de Instanciacion de
Flujos de Trabajo").

En este trabajo se considera una versión restringida de WIP que adopta la
aproximación Local-As-View (LAV) XREF28W. En LAV, todos los elementos de un
problema son especificados con un lenguaje común basado en servicios abstractos
tal que los servicios concretos se describen como vistas de servicios
abstractos, la calidad de una instanciación como una función de utilidad global
que combina los diferentes parámetros QoS, y el flujo de trabajo abstracto como
una consulta conjuntvia sobre los servicios abstractos; esta representación es
similar a la que es generada de manera semiautomática para el sistema DEIMOS
XREF3W. En esta versión de WIP, las reglas que definen flujos de trabajo
(consultas conjuntivas) y servicios concretos (vistas) son creados de manera que
todas las restricciones funcionales sobre pre- y post-condiciones de los
servicios y sus combinaciones sean satisfechas, y las medidas de QoS son
representadas anotando cada descripción de servicio concreto con un número real
que representa la utilidad QoS global del servicio.

Bajo estas suposiciones, WIP puede ser convertido en el bien conocido Problema
de Reescritura de Consultas (QRP) para LAV que es central a los sistemas de
integración XREF23W. QRP consiste de una consulta conjuntiva que debe ser
respondida en términos de vistas donde la consulta y las vistas son descritas
usando LAV con relaciones abstractas. Este problema es importante en el contexto
de integración de datos XREF13X24W, y optimización de consultas y mantenimiento
de datos XREF1,28W, y varias aproximaciones que escalan a un número grande de
vistas han sido definidas XREF4,18,19,28,29W.

La aproximación recente de Arvelo et al. XREF4W está basada en la enumeración
eficiente de modelos de una teoría lógica proposicional. Dado unQRP, una teoría
lógica es construída de manera que cada modelo de la teoría codifica una
reescritura válida, y así todas las reescrituras son obtenidas enumerando los
modelos de la teoría. Esta enumeración puede ser realizada eficientemente si la
teoría lógica está en cierta forma normal llamada la forma normal de negación
determinística y descomponible (d-DNNF por sus siglas en inglés) XREF14W. Así,
la aproximación consiste en transformar (llamado compilar en el campo de
compilación del conocimiento) la teoría lógica en el formato d-DNNF para
enumerar sus modelos eficientemente.

Pero las teorías d-DNNF no sólo soportan la enumeración eficiente de sus
modelos, sino también otras operaciones. Entre ellas se encuentra la enumeración
de los modelos con mejor puntuación. Dada una función de puntuación de literales
$r(l)$ que asigne puntuaciones a cada literal $l$, se define la puntuación
$r(w)$ de un modelo $w$ como la suma de las puntuaciones de los literales
activados por $w$ (es decir, $r(w) = suma...$, y se dice que $w$ es un modelo
mejor puntuado si no hay modelo $wpriemrie$ tal que $r(w)...$.

Dada una teoría en d-DNNF, se computa la puntuación de los mejores modelos, y
los mejores modelos, en tiempo lineal en el tamaño del d-DNNF. Este cómputo
transforma el GAD del d-DNNF en un circuito aritmético reemplazando los nodos
AND con '+' y los nodos OR con 'min'. La función de puntuación de literales
asigna valores a las hojas del circuito que son propagados a la raíz en tiempo
lineal. El valor de la raíz es la puntuación del mejor modelo XREF17W.

En este trabajo, se explotan las propiedades de los d-DNNF construyendo una
teoría lógica cuyos modelos codifican las instanciaciones del flujo de trabajo

\end{document}
