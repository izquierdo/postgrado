\documentclass{article}

\usepackage[utf8x]{inputenc}
\usepackage[spanish]{babel}

\begin{document}

\title{Instanciación Óptima de Flujos de Trabajo Abstractos Usando Lógica y Circuitos}
\maketitle

\section{Resumen}

La Arquitectura Orientada a Servicios típica consiste de dos capas principales.
La capa concreta con servicios concretos cuyas funcionalidades son descritas en
términos de pre- y post-condiciones y propiedades no funcionales en términos de
parámetros de calidad de servicio (Quality Of Service, QoS), y la capa abstracta
con aplicaciones de software cuyas funcionalidades son descritas en términos de
flujos de trabajo abstractos y propiedades no funcionales en términos de
restricciones sobre QoS. Durante la ejecución de un flujo de trabajo, los
servicios abstractos son instanciados en servicios concretos que cumplen con los
requerimientos funcionales y no funcionales. Esta instanciación, que debe ser
hecha en el momento, consiste de una búsqueda en un espacio de posibilidades
combinatorio. En este trabajo se propone una infraestructura para resolver
eficientemente el problema de instanciación que está firmemente basada en
lógica. La infraestructura adopta la aproximación Local-As-View en la cual la
funcionalidad de servicios concretos se describe usando vistas de servicios
abstractos, la calidad de una instanciación como una función de utilidad global
que combina los diferentes parámetros QoS, y los flujos de trabajo abstractos
como consultas conjuntivas sobre servicios abstractos. Usando esta
representación, el problema de instanciación de flujos de trabajo se convierte
en un problema de reescritura de consultas, del área de sistemas de integración.
Entonces, construyendo sobre trabajo relacionado, se idea una codificación del
problema de instanciación de flujos de trabajo como una teoría lógica cuyos
modelos están en correspondencia con las instanciaciones del flujo de trabajo, y
los modelos mejor puntuados en correspondencia con instanciaciones óptimas del
flujo de trabajo. Así, explotando propiedades conocidas de teorías lógicas en
formato d-DNNF, se provee una solución eficiente y escalable al problema de
instanciación de flujos de trabajo. Esta aproximación no sólo escala a
instancias grandes como muestran resultados experimentales ya realizados, sino
que también es correcta y completa dado que, estándo basada en la lógica, es
amigable al análisis formal.

\end{document}
