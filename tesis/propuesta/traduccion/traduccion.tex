\documentclass{article}

\usepackage[utf8x]{inputenc}
\usepackage[spanish]{babel}

\begin{document}

\tableofcontents

\section{Trabajo Relacionado}

En esta sección se resumen las aproximaciones existentes que proveen soluciones
a los problemas de selección de servicios y reescritura de consultas y se
discute el trabajo relacionado en el área de inteligencia artificial llamado
compilación del conocimiento.

\subsection{Selección de Servicios}

El problema de seleccionar los servicios que implementan un flujo de trabajo
abstracto y cumplen mejor los criterios basados en QoS es conocido como el
problema de selección o composición de servicios consciente de QoS, que ha sido
demostrado ser NP-difícil XREF34W. Este es un problema de optimización
combinatoria y varias heurísticas han sido propuestas para conseguir una
solución relativamente buena en un período razonablemente corto de tiempo.

Rahmani et al. XREF31W presenta una heurística basada en métrica de distancia
que guía un algoritmo de búsqueda hacia atrás; esta métrica induce un orden de
los servicios en una manera en que los nodos vertedero tienen baja probabilidad
de ser visitados. En una serie de artículos, Berardi y otros XREF7-9W describen
servicios y flujos de trabajo en términos de máquinas de estados finitos
determinísticas que son codificadas usando teorías de Lógica de Descripción
cuyos modelos corresponden a soluciones del problema. Aunque se podrían explotar
métodos para formalismos de Lógicas de Descripción, no han sido reportados la
escalabilidad o el rendimiento de la solución propuesta.

Ko et al. XREF25W propone una aproximación basada en restriccciones que codifica
los valores permisibles no funcionales como un conjunto de restricciones cuya
violación debe ser minimizada; para recorrer el espacio de soluciones
posiblemente óptima, se implementa un algoritmo híbrido que combina las
metaheurísticas TODOemphtabusearch y TODOemphsimulatedannealing. Los resultados
experimentales muestran que la solución propuesta es capaz de escalar a un
número grande de servicios y procesos abstractos. Cardellini et al. XREF12W
codifican una parte del problema de composición de servicios consciente de QoS
como un problema de programación lineal XREF12W. Por otra parte, Wada et al.
XREF34W trata el problema como uno de optimización con múltiples objetivos donde
los diferentes parámetros de QoS son considerados igualmente importanes en vez
de ser agregados en una función. Luego se propone un algoritmo basado en
algoritmos genéticos que identifica un conjunto de composiciones de servicios
no dominadas que mejor encajan en los requerimientos QoS.

Alrifai y Risse XREF2W proponen una solución de doble cara que usa un
algoritmo de programación entera híbrido para obtener la descomposición de QoS
globales en restricciones locales y luego selecciona los servicios que mejor
cumplen las restricciones locales.

Recientemente, dos aproximaciones basadas en planificación han sido propuestas.
Kuter y Golbek XREF26W extienden el algoritmo de planificación SHOP2 para
seleccionar la composición confiable de servicios que implementan un modelo de
procesos OWL-S dado, mientras que Sohrabi y McIlraith XREF33W proponen una
solución basada en planificación HTN donde las métricas de preferencia del
usario y regulaciones de dominio son utilizadas para guíar el planificador hacia
el espacio de composiciones relevantes. Finalmente, Lécué XREF27W propone un
algoritmo basado en genética para identificar la composición de servicios que
mejor cumple con los criterios de calidad para un conjunto de parámetros QoS.

Aunque estas soluciones son capaces de resolver el problema de optimización y
escalar a un número de procesos abstractos, ninguno de ellos está ajustado para
describir semánticamente los servicios en términos de procesos abstractos, ni
para usar estas descripciones para identificar los servicios que implementan un
flujo de trabajo dado o que mejor cumplen los criterios no funcionales del
usuario.

\subsection{Reescritura de Consultas}

Un número de algoritmos han sido desarrollados para encontrar las reescrituras
de una consulta dada; los más prominentes son el algoritmo bucket XREF28W, el
algoritmo de reglas inversas XREF18,30W, el algoritmo MiniCon XREF29, y el
algoritmo McdSat XREF4W. Generalmente, la reescritura de consultas trabaja en
dos fases. Durante la primera, el algoritmo identifica las vistas que reescriben
al menos un subobjetivo de la consulta, y durante el segundo estas reescrituras
parciales son combinadas en reescrituras completas. La diferencia principal
entre los algoritmos es el conjunto de criterios usado para elegir las vistas
relevantes para reducir el espacio de reescrituras no útiles.

El algoritmo bucket reduce el número de posibilidades considerando sólo cada
subobjetivo de la consulta aislado, y seleccionando las vistas que son capaces
de producir al menos los atributos proyectados por la consulta. Como los
atributos involucrados en los joins de la consulta no son verificados, un número
grande de reescrituras comprendido por productos cartesianos puede ser generado.

El algoritmo de reglas inversas construye un conjunto de reglas que invierten
las definiciones de vistas y establecen cómo computar tuplas para las relaciones
de la base de datos desde las tuplas de las vistas. Al igual que el algoritmo
bucket, puede producir un gran número de reescrituras no útiles.

El algoritmo MiniCon supera las limitaciones de los algoritmos previos
identificando solamente vistas que reescriben un conjunto de los subobjetivos de
la consulta, y que pueden ser combinados con el resto de los subobjetivos. La
idea principal es identificar las asociaciones entre variables de cada
subobjetivo a las variables en uno o más subojbetivos en las vistas, de manera
que las variables de join de la consulta son asociadas a las variables de join
del cuerpo de una vista o a variables distinguidas de la vista. Asociaciones
entre variables y subojbetivos son representadas en las llamadas TODOemphMiniCon
Descriptions (MCDs) XREF29W.

Finalmente, el algoritmo McdSat es capaz de identificar las reescrituras de una
consulta traduciendo el problema de reescritura al problema de enumeración de
modelos de una teoría proposicional cuyos modelos están en correspondencia con
las reescrituras de la consulta. El algoritmo explota las propiedades de los
d-DNNF para enumerar eficientemente los modelos de la teoría. El algoritmo
McdSat ha demostrado escalar mejor que el algoritmo MiniCon sobre un número
grande de experimentos donde frecuentemente muestra mejoras de rendimiento de
varios órdenes de magnitud. Sin embargo, el algoritmo McdSat no fue diseñado
para problemas de reescritura que involucren constantes explícitas, ni para
computar las mejores reescrituras respecto a una función de utilidad o modelo de
costo dado. En este trabajo se propone una codificación extendida que supera
estas limitaciones y se aplica la codificación al problema de instanciación de
flujos de trabajo.

\subsection{Compilación del conocimiento}
TODO Fig1

Compilación del conocimiento es el área en inteligencia artificial que se ocupa
del problema de traducir teorías lógicas en fragmentos apropiados que hagan
tratables ciertas operaciones deseadas XREF11W. Diferentes lenguajes de
compilación han sido definidos. Por ejemplo están los TODOemphtodoesto Ordered
Binary Decision Diagrams (OBDDs) XREF10W, Negation Normal Form (NNF) XREF5W, y
Decomposable Negation Normal Form (DNNF) XREF16W. En este trabajo se usan las
propiedades de los DNNFs determinísticos (d-DNNF) XREF14W para proveer una
solución eficiente y escalable al problema de selección de serviccios.

Una teoría en Negation Normal Form es construida desde literales usando sólo
conjunciones y disjunciones XREF5W, y puede ser representada como un grafo
acíclico dirigido en que las hojas son etiquetadas con literales y los nodos
internos con $TODOand$ y $TODOor$; ver ejemplo en la figura  1TODOLABELREF. Este
fragmento es universal, lo que significa que para cada fórmula lógica hay una
equivalente en formato NNF. Se dice que un NNF es descomponible (DNNF) XREF14W
si para cada conjunción $TODOformula$, sus variables son disjuntos por pares; es
decir, $TODO formula vars y vars$ para $i TODOnoigual j$. Un DNNF soporta un
número de operaciones en tiempo polinomial en el tamaño de su GAD. Por ejemplo,
podemos probar si un DNNF es satisfactible haciendo una simple pasada de abajo
hacia arriba sobre su GAD en tiempo linear. Se dice que un DNNF es
determinístico (d-DNNF) XREF14W si para cada disjunción $TODOformula$, los
disjuntos sob lógicamente contradictorios por pares, es decir, $TODO formula
false idistintoj$. El NNF en la figura 1TODOLABELREF, por ejemplo, es
decomposableTODOemph y determinístico. Un d-DNNF soporta conteo de modelos en
tiempo polinomial en el tamaño de su GAD, y enumeración de modelos en tiempo
polinomial en el tamaño de la salida. Además, dada una función de puntuación de
literales $r$, se puede computar la puntuación del mejor modelo en tiempo
polinomial para DNNFs XREF17W.

Los fragmentos DNNF y d-DNNF son universales pero traducir una teoría CNF a
formrato DNNF tiene costo exponencial en el peor caso. Esta traducción se
llama compilación en este campo. Hay un compilador públicamente disponible,
llamado c2d TODOfootnote que realiza este proceso de compilación y que hace uso
de técnicas de SAT modernas como backtracking dirigido por conflictos,
aprendizaje de cláusulas y caching XREF15W. Este compilador entra en el peor
caso en espacio exponencial en un parámetro llamado el ancho del árbol de
descomposición que está relacionado a la TODO"conectividad" de la teoría CNF.
Sin embargo, en nuestros experimentos, las teorías CNF que son compiladas son de
ancho pequeño.

\section{Arquitectura del sistema}

TODOfig2 system arquitecthuere

Se utiliza una arquitectura comprendida por un catálogo de descripciones de
servicio, el Codificador, el compilador c2d, el Buscador de mejores modelos, y
el Decodificador. La figura TODOREF2 muestra la arquitectura global del sistema.
En estta infraestructura, una instancia del problema de instanciación de lfujos
de trabajo se define como un flujo abstracto representado por una consulta
conjuntiva sobre servicios abstractos que es dada como entrada junto con un
conjunto de servicios concretos definidos por vistas de servicios abstractos.

El catálogo se pobla con descripciones de servicios abstractos y concretos; cada
servicio es descrito en términos de atributos de entrada y salida y anotado con
un valor real que representa la utilidad QoS del servicio. La descripción de los
servicios concretos, que son definidos como vistas de los servicios abstractos,
puede ser generada de manera semiautomática o automática usando herramientas
tales como el sistema DEIMOS XREF3W.

Una instancia de entrada de WIP es codificada como una teoría CNF cuyos modelos
corresponden a las intanciaciones del flujo de trabajo por el Codificador. El
compilador c2d, un componente TODOemphofftheshelf, compila la fórmula CNF a
d-DNNF. El Codificador traduce las instancias WIP en teorías CNF que luego son
convertidas en d-DNNF usando c2d. El Buscador computa un mejor modelo dados los
parámetros QoS en tiempo lineal en el tamaño del d-DNNF resultante. Es
importante remarcar que el proceso de compilación necesita ser realizado sólo
una vez ya que no depende del valor de los parámetros QoS. Así, incluso si la
compilación resulta ser costosa en términos de tiempo, este costo puede ser
amortizado dado que el d-DNNF resultante puede ser usado para conseguir mejores
instanciaciones con respecto a múltiples valores de los parámetros QoS.
Finalmente, el Decodificador traduce el mejor modelo retornado por el Buscardor
a una instanciación de flujo de trabajo que resuelve el WIP.

Dado un CNF que codifica un WIP, su d-DNNF es una representación compacta de
todas las instanciaciones del flujo de trabajo. Es decir, uno puede generar de
una manera libre de TODOemphbacktracking todas las instanciaciones del flujo de
trabajo. Si el usuario está interesado en una mejor instanciación dados
parámetros de QoS, entonces esta se puede computar en tiempo lineal en el tamñao
del d-DNNF. Si el usuario está interesado en todas las mejores instanciaciones,
estas pueden ser computadas en tiempo lineal en su número. FInalmente, si el
usuario está interesado en todas las instanciaciones, estas también pueden ser
computadas en tiempo lineal en su número. En los últimos dos casos, si ese
número es exponencial (en el tamaño de la entrada), la enumeración de las
instanciaciones también lo es pero esta complejidad es intrínsica al problema y
por lo tanto no puede ser evitada.

\section{Formalización}

Consideramos cat[alogos de servicio de la forma $C = ST$ donde $S$ es un
conjunto de predicados que representa servicios abstractos y $T=TTODO$ es una
colección de tablas que representa el resultado de evaluar los servicios
abstractos. En el contexto del WIP, un catálogo de servicio $C$ es una
descripción idealizada de la salida producida por flujos abstractos
implementados por servicios concretos descritos como vistas. Un flujo de trabajo
$W$ sobre $S$ es una consulta conjuntiva de la forma

$$
WX
$$

donde $si,x$ es un vector de variables y cada $xi$ es un vector de variables y
constantes. El resultado de $W$ sobre $C$, denotado como $W(C)$, es la tabla con
$|x|$ columnas que resulta de la proyección del TODOemphjoin relacional
$JOINTODO$ sobre $x$. Los átomos en el cuerpo de $W$ son llamados los
(sub)objetivos de $W$, y las variables en la cabeza de $W$ son llamados
distinguidos.

Un servicio concreto es descrito como una vista $V$ sobre $C$ que, siguiendo
LAV, es una consulta sobre $S$. Dado un catálogo $C$, un flujo de trabajo $W$ y
una colección de $n$ vistas $E=$, se nos pide conseguir todas las tuplas en
$W(C)$ que puedan ser obtenidas de vistas en $E$. Es decir, se necesitan
conseguir las instanciaciones

$$
R(x)
$$

tales que $R(E)...$. Un problema de instanciación de flujo de trabajo es una
tupla $tUTPLA$ donde $S$ es un conjunto de predicados que representa servicios
abstractos, $W$ es un flujo de trabajo sobre $S$, y $VI$ una colección de vistas
que define servicios concretos. Suponemos trabajo con problemas \emph{seguros}
en el sentido en que todas las variables mencionadas en la cabeza del flujo de
trabajo (respectivamente en la cabeza de cada vista) aparecen en el cuerpo del
flujo de trabajo (respectivamente en el cuerpo de cada vista). Además, sólo
lidiamos con WIPs sin predicados aritméticos dentro del flujo de trabajo o
vistas. Una instanciación $R$ es válida si para todos los catálogos $C=$ y
extensiones $EIII$, $R(E\#)$. Una colección $R$ de instanciaciones válidas es una
solución si para todos los catálogos de servicio $C=$ y extensiones $EII$, no
existe $R'$ tal que $R(E= C$.

\subsection{Teorías Lógicas}

Se usa una aproximación similar a la descrita en XREF4W para codificar el WIP.
Se identifican instanciaciones del flujo de trabajo enumerando los modelos de
una teoría lógica.

\end{document}
