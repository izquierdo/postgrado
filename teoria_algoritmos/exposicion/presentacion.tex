\documentclass{beamer}

\usepackage{beamerthemesplit}
\usepackage[spanish,activeacute]{babel}
\usepackage[utf8]{inputenc}

\begin{document}
\title{Solución exacta de problemas difíciles: Backtraking para "Satisfiability"}
\author{Daniel Izquierdo}
\date{25 de noviembre de 2008}

\frame{\titlepage}

\section[Contenido]{}
\frame
{
  \begin{itemize}
  \item Definición del problema.
  \item Opciones para resolver el problema.
  \item Técnicas para mejorar backtracking.
  \end{itemize}
}

\section{Introducción}

\frame
{
  \frametitle{Ejemplos}

  Problema de satisfactibilidad: ¿es posible satisfacer las siguientes expresiones?

  \begin{itemize}
  \item $v_1 \vee v_2$
  \item $(v_1 \vee \neg v_2) \wedge (v_2 \vee v_3)$
  \item $(v_1 \wedge \neg v_2) \wedge (v_2 \wedge v_3)$
  \end{itemize}
}

\frame
{
  \frametitle{Un ejemplo más complicado}

  \begin{itemize}
  \item $(\neg v_{1} \vee v_5 \vee v_3) \wedge (\neg v_{3} \vee \neg v_{4} \vee \neg v_{6}) \wedge (\neg v_{1} \vee \neg v_{8} \vee v_{7}) \wedge$
        $(\neg v_{4} \vee \neg v_{1} \vee v_{5}) \wedge (v_4 \vee v_2 \vee v_8) \wedge (\neg v_3 \vee \neg v_9 \vee v_1) \wedge (v_7 \vee v_4 \vee v_8)$
  \end{itemize}
}

\frame
{
  \frametitle{CNF}

  \begin{itemize}
  \item CNF: ``Conjunctive normal form'' o ``Forma normal conjuntiva''
  \item Es una forma ``estándar'' de representar fórmulas.
  \item La CNF de una fórmula consiste en una conjunción de disyunciones $(A \vee B \vee \ldots) \wedge (C \vee D \vee \ldots) \wedge \ldots$
  \item Cualquier expresión proposicional se puede convertir a CNF preservando
        equivalencia o satisfactibilidad.
  \end{itemize}
}

\frame
{
  \frametitle{¿Cuál es el problema y cómo resolverlo?}

  Dada una fórmula $F$ con $n$ variables $X_1, \ldots, X_n$, decir si es
  posible asignar valores a las variables de manera que $F$ se satisfaga.

  Enfoques:

  \begin{itemize}
  \item Heurísticas
  \item Backtracking
  \end{itemize}
}

\section{Backtracking}

\frame
{
  \frametitle{Enfoque ``fuerza bruta''}

  Usar backtracking para intentar todas las posibles asignaciones de valores a variables. Por
  ejemplo, para dos variables $X$ e $Y$ podemos probar:

  $$
  (X=0,Y=0), (X=0,Y=1), (X=1,Y=0), (X=1,Y=1)
  $$

  Problema con este enfoque: toma tiempo exponencial ($2^n$)
}

\section{Preguntas}

\frame
{
  \frametitle{Preguntas}
}

\end{document}
