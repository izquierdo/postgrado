\documentclass{article}

\usepackage{algorithm}
\usepackage{algorithmic}
\usepackage{amsmath}
\usepackage{amssymb}
\usepackage[utf8]{inputenc}
\usepackage{listings}
\usepackage[spanish]{babel}

% Para codigo bonito
\newcommand{\singlespace}{\renewcommand{\baselinestretch}{1.0}}

\newcommand{\algoritmo}[2]
{
{
  \singlespace
  \begin{algorithm}[h]
    \caption{#1}
    \begin{algorithmic}[1]
      #2
    \end{algorithmic}
  \end{algorithm}
}
}

\newcommand{\theHalgorithm}{\arabic{algorithm}}

\lstset{mathescape=true}

\begin{document}

% Título

\title{Tarea 1 de Teoría de Algoritmos}
\author{Daniel Izquierdo \\ \#08-86809}
\date{9 de octubre de 2008}

\maketitle

% Pregunta 1
\section{}

\begin{align*}
 & (n+\alpha)^{n+\beta} \\
 = & \\
 & n^{n+\beta} * (1+\frac{\alpha}{n})^{n+\beta} \\
 = & \\
 & n^{n+\beta} * e^{lg (1+\frac{\alpha}{n})^{n+\beta}} \\
 = & \\
 & n^{n+\beta} * e^{(n+\beta) * lg (1+\frac{\alpha}{n})} \\
 = & \\
 & n^{n+\beta} * e^{(n+\beta) * (\frac{\alpha}{n} - \frac{\alpha^2}{2n^2}+O(n^{-3})}) \\
 = & \\
 & n^{n+\beta} * e^{\alpha-\frac{\alpha^2}{2n}+n*O(n^{-3})+\frac{\alpha\beta}{n}-\frac{\alpha^2\beta}{2n^2}+\beta*O(n^{-3})} \\
 = & \\
 & n^{n+\beta} * e^{\alpha-\frac{\alpha^2}{2n}+O(n^{-2})+\frac{\alpha\beta}{n}-\frac{\alpha^2\beta}{2n^2}+O(n^{-3})} \\
 = & \\
 & n^{n+\beta} * e^{\alpha-\frac{\alpha^2}{2n}+O(n^{-2})+\frac{\alpha\beta}{n}-\frac{\alpha^2\beta}{2n^2}} \\
 = & \\
 & n^{n+\beta} * e^{\alpha-\frac{\alpha^2}{2n}+\frac{\alpha\beta}{n}+O(n^{-2})} \\
 = & \\
 & n^{n+\beta} * e^{\alpha} * e^{-\frac{\alpha^2}{2n}+\frac{\alpha\beta}{n}+O(n^{-2})} \\
 = & \\
 & n^{n+\beta} * e^{\alpha} * (1-\frac{\alpha^2}{2n}+\frac{\alpha\beta}{n}+O(n^{-2}) + O(n^2)) \\
 = & \\
 & n^{n+\beta} * e^{\alpha} * (1+\alpha(\beta-\frac{1}{2}\alpha)n^{-1}+O(n^{2})) \\
\end{align*}


% Pregunta 2
\section{}

La longitud de la representación de un entero en binario o decimal es logarítmica con
respecto a su magnitud, pero en palabras de un computador es la longitud entre el tamaño
de la palabra. Es este factor el que hace la diferencia al usar algoritmos exponenciales.

% Pregunta 3
\section{}

Para el análisis del algoritmo descrito en el libro tomo la suma

\begin{equation}
 C(n-1,k-1) + C(n-1,k)
\end{equation}

como barómetro. En todos los casos excepto los base se realiza esa suma.
La función $t(n)$ a calcular se define de manera recursiva.
En el caso en que $n = 0$, obligatoriamente se mantiene que $k = 0$ ya que
tenemos $0 \leq k \leq n$. El algoritmo retorna una respuesta sin realizar otro
cálculo. En caso contrario, el algoritmo realiza dos llamadas recursivas que
toman --en el peor caso-- tiempo $t(n-1)$. Luego realiza una suma
en tiempo constante. Entonces,

$$
 t(n) = \left\{ \begin{array}{ll}
 0 & \textrm{si $n = 0$} \\
 2*t(n-1)+1 & \textrm{si $n > 0$}
 \end{array} \right.
$$

La solución a esta recurrencia es dada en el ejemplo {\bf 4.7.6} del libro, y es

\begin{equation}
 t(n) = 2^n-1
\end{equation}

Por lo tanto,

\begin{equation}
t(n) = \Theta(2^n)
\end{equation}

% Pregunta 4
\section{}

Decir que $q$ es el cociente de la divisón entera de $A$ entre $B$ y $r$ es el
resto de la división entera de $A$ entre $B$ es equivalente a decir que:

\begin{align*}
 A = q*B + r \wedge 0 \leq r < b
\end{align*}

La última instrucción del programa es una asignación. Tenemos:

\begin{align*}
 & wp.(r \leftarrow A-q*B).(A=q*B+r \wedge 0 \leq r < B) \\
 \equiv & <\text{precondición más débil de la asignación}> \\
 & (A=q*B+r \wedge 0 \leq r < b)[r := A-q*B] \\
 \equiv & <\text{sustitución}> \\
 & A=q*B+(A-q*B) \wedge 0 \leq (A-q*B) < B \\
 \equiv & <\text{aritmética}> \\
 & true \wedge 0 \leq (A-q*B) < B \\
 \equiv & <\text{true es neutro de } \wedge> \\
 & 0 \leq (A-q*B) < B \\
 \equiv & <\text{definición de } \leq \text{ y } <> \\
 & 0 \leq A-q*B \wedge A-q*B < B
\end{align*}

Llamemos $P$ a este resultado. Entonces $P$ es la postcondición que debe ser
satisfecha por el ciclo \texttt{while} para
que el programa sea correcto. Si defino la invariante $Inv$ y la cota $Bound$
del ciclo

\begin{eqnarray*}
 Inv & \equiv & A - q*B \geq 0 \\
 Bound & \equiv & A-q*B \\
\end{eqnarray*}

y tenemos el guardia $B$ definido

\begin{align*}
 B \equiv A-(q+1)*B \geq 0
\end{align*}

entonces debo demostrar lo siguiente, suponiendo que $X$ es alguna constante y
$A$ y $B$ son enteros:

\begin{eqnarray}
 \label{divmod:init} \text{\{} A \geq 0 \wedge B > 0 \text{\}} q \leftarrow 0 \text{\{}Inv\text{\}} & & \\
 \label{divmod:inv} \text{\{} Inv \wedge B \text{\}} q \leftarrow q + 1 \text{\{}Inv\text{\}} & & \\
 \label{divmod:final} Inv \wedge \neg B \Rightarrow P & & \\
 \label{divmod:bounddecr} \text{\{} Inv \wedge X = Bound \text{\}} q \leftarrow q + 1 \text{\{} Bound < X \text{\}} & & \\
 \label{divmod:boundnn} Inv \Rightarrow Bound \geq 0 & & 
\end{eqnarray}

Empiezo con el predicado \ref{divmod:init}:

\begin{align*}
 & wp.(q \leftarrow 0).Inv \\
 \equiv & <\text{precondición más débil de la asignación}> \\
 & Inv[q := 0] \\
 \equiv & <\text{definición de } Inv \text{ y sustitución}> \\
 & A - 0*B \geq 0 \\
 \equiv & <\text{aritmética}> \\
 & A \geq 0 \\
 \Leftarrow & <P \wedge Q \Rightarrow P> \\
 & A \geq 0 \wedge B > 0 \\
 \therefore & <\text{relación entre tripletas de Hoare y precondición más débil}> \\
 & \text{\{} A \geq 0 \wedge B > 0 \text{\}} q \leftarrow 0 \text{\{}Inv\text{\}}
\end{align*}

Luego demuestro el predicado \ref{divmod:inv}:

\begin{align*}
 & wp.(q \leftarrow q + 1).Inv \\
 \equiv & <\text{precondición más débil de la asignación}> \\
 & Inv[q := q + 1] \\
 \equiv & <\text{definición de } Inv \text{ y sustitución}> \\
 & A-(q+1)*B \geq 0 \\
 \equiv & <\text{definición de } B> \\
 & B \\
 \Leftarrow & <P \wedge Q \Rightarrow P> \\
 & Inv \wedge B \\
 \therefore & <\text{relación entre tripletas de Hoare y precondición más débil}> \\
 & \text{\{} Inv \wedge B \text{\}} q \leftarrow q + 1 \text{\{}Inv\text{\}}
\end{align*}

Continúo con el predicado \ref{divmod:final}:

\begin{align*}
 & Inv \wedge \neg B \\
 \equiv & <\text{definiciones de } Inv \text{ y de } B \text{ y aritmética}> \\
 & A - q*B \geq 0 \wedge \neg (A \geq (q+1)*B) \\
 \equiv & <\text{negación de } \geq> \\
 & A - q*B \geq 0 \wedge A < (q+1)*B \\
 \equiv & <\text{aritmética}> \\
 & A - q*B \geq 0 \wedge A - q*B < B \\
 \equiv & <\text{definición de P}> \\
 & P
\end{align*}

Queda comprobar la terminación del algoritmo. Por una parte,
si suponemos $Inv \wedge X = Bound = A-q*B$, entonces

\begin{align*}
 & wp.(q \leftarrow q + 1).(Bound < X) \\
 \equiv & <\text{precondición más débil de la asignación}> \\
 & (Bound < X)[q := q + 1] \\
 \equiv & <\text{definición de Bound y sustitución}> \\
 & A-(q+1)*B < X \\
 \Leftarrow & <\text{aritmética y suposición } X = Bound = A-q*B> \\
 & A-q*B-B < A-q*B \\
 \equiv & <\text{aritmética}> \\
 & 0 < B \\
 \Leftarrow & <\text{suposición } Inv> \\
 & true
\end{align*}

con lo que queda demostrado el predicado \ref{divmod:bounddecr}. El predicado
\ref{divmod:boundnn} sigue trivialmente a partir de las definiciones de $Inv$ y
$Bound$.

% Pregunta 5
\section{}

Llamaré \texttt{convertir} al algoritmo que presento para convertir un arreglo 
en un heap binario. El algoritmo utiliza un procedimiento auxiliar llamado
\texttt{heapify}. Es igual al descrito en el ``Introduction to Algorithms'' de
Cormen y otros. Suponemos que $tamano(A)$ es el tamaño del heap $A$, y
escribimos

\algoritmo{heapify(A,i)}
{
\STATE $l \leftarrow 2*i$
\STATE $r \leftarrow 2*i+1$
\IF {$l \leq tamano(A) \wedge A[l] < A[i]$}
\STATE $maspequeno \leftarrow l$
\ELSE
\STATE $maspequeno \leftarrow i$
\ENDIF
\IF {$r \leq tamano(A) \wedge A[r] < A[maspequeno]$}
\STATE $maspequeno \leftarrow r$
\ENDIF
\IF {$maspequeno \neq i$}
\STATE intercambiar A[i] con A[masgrande]
\STATE heapify(A, masgrande)
\ENDIF
}

\algoritmo{convertir(A)}
{
\STATE $tamano(A) \leftarrow longitud [A]$
\FOR {$i \leftarrow piso(longitud[A]/2)$ hasta $1$}
\STATE heapify(A, i)
\ENDFOR
}

El procedimiento \texttt{heapify} toma un nodo del árbol formado en el arreglo
y tiene como precondición que sus dos hijos sean raíces de un heap. Ejecutar el
procedimiento tiene como resultado que el valor del nodo dado ``descienda''
por el árbol
hasta una posición tal que se preserve la propiedad de heap. Si el valor es
mayor
que el menor de sus hijos, se intercambian sus posiciones y se continúa sobre
la posición del nodo que está descendiendo. Al final, el árbol termina
cumpliendo la propiedad de heap a partir del nodo dado.

Los
elementos en la segunda mitad del arreglo dado representan árboles de un sólo
nodo, por lo que cumplen la propiedad del heap.
\texttt{convertir} trabaja sobre la primera mitad del arreglo de derecha a
izquierda, llamando a \texttt{heapify} sobre cada nodo visto. \texttt{heapify}
funciona porque cada nodo a la derecha del nodo que se está viendo ya es un heap
por llamadas previas a \texttt{heapify}. Al final, hacer \texttt{heapify} sobre
el primer elemento del arreglo asegura que esté organizado como un heap.

Definimos la altura de un nodo de un heap como el número de lados a recorrer
desde ese nodo hasta la hoja del heap más cercana y la altura del heap como la
altura de su raíz. Se puede demostrar que la altura de un heap con $n$ elementos
tiene altura $\lfloor lg n\rfloor$ y que un heap cualquiera tiene como máximo
$\lceil n / 2^{h+1} \rceil$ nodos con altura $h$. Por otra parte,
\texttt{heapify} toma tiempo $O(h)$ para trabajar sobre un árbol de altura $h$.
Además, \texttt{convertir} realiza $n/2 = O(n)$ llamadas a \texttt{heapify}
sobre un arreglo de tamaño $n$.

Por todo esto, el tiempo de ejecución de \texttt{convertir} está acotado por la
siguiente fórmula suponiendo que $n$ es potencia de $2$:

\begin{align*}
 & \sum_{h=0}^{lg n} \frac{n}{2^{h+1}} O(h) \\
 = & <\text{aritmética, propiedades de O(f(n))}> \\
 & O(n \sum_{h=0}^{lg n} \frac{h}{2^{h}})
\end{align*}

Usando la ecuación {\bf A.8} del Cormen, tenemos

\begin{align*}
 \sum_{h=0}^{\infty} \frac{h}{2^h} = 2 \\
\end{align*}

Esto acota el tiempo de ejecución de \texttt{convertir}:

\begin{align*}
 O(n \sum_{h=0}^{lg n} \frac{h}{2^{h}}) &= O(n \sum_{h=0}^{\infty} \frac{h}{2^h}) \\
 &= O(n) \\
\end{align*}

Sabemos que $n$ es uniforme. Además,

\begin{equation}
 \sum_{h=0}^{lg n} \frac{n}{2^{h+1}} O(h) \\
\end{equation}

es no decreciente sobre $n$ porque es una suma de términos positivos cuyo
rango aumenta al aumentar $n$. Entonces, por la regla de la suavidad, el tiempo
de ejecución de \texttt{convertir} es $O(n)$.

% Pregunta 6
\section{}

La estrucura de datos cola tiene dos operaciones: \texttt{encolar} y
\texttt{desencolar}. Para implementar una cola con dos pilas $S_1$ y $S_2$,
añado una operación auxiliar \texttt{mudar}. Para una cola $Q$, voy a llamar a
sus dos pilas $Q.S_1$ y $Q.S_2$. Supongo que nunca se ejecuta
\texttt{desencolar} sobre una cola vacía (lo cual es un error), y defino
entonces:

\begin{lstlisting}
encolar(Q, x) = empilar(Q.$S_1$, x)

desencolar(Q) = si vacia(Q.$S_2$)
                    mudar(Q.$S_1$, Q.$S_2$)
                return desempilar(Q.$S_2$)

mudar(Q.$S_1$, Q.$S_2$) = mientras $\neg$vacia(Q.$S_1$)
                              x $\leftarrow$ desempilar(Q.$S_1$)
                              empilar(Q.$S_2$, x)
\end{lstlisting}

Ahora hago un análisis del costo en tiempo amortizado de las operaciones
\texttt{encolar} y \texttt{desencolar} para verificar que ambas son $O(1)$. Para
ello utilizo el método del potencial. Defino la función potencial $\Phi$ para
una cola $Q$ como 

\[\Phi(Q) = 3 * \#(Q.S_1) + \#(Q.S_2)\]

donde

\[\#(S)\]

es el número de elementos en la pila $S$. Es claro que, como el número de
elementos en una pila es no negativo, $\Phi$ es no negativo. Entonces, si $Q_0$
es una pila recién creada y $Q_i$ es la misma pila luego de haberle aplicado
alguna secuencia de operaciones, tenemos

\begin{align*}
\Phi(Q_i) &\geq 0 \\
          &= \Phi(Q_0)
\end{align*}

Así, $\Phi$ cumple esa propiedad necesaria del método del potencial. Ahora
calculo el costo amortizado $\hat{c}_i$ de las operaciones que definí:

\begin{itemize}
 \item \texttt{encolar($Q$, $x$)}:
 Realizamos una operación de inserción en una pila y el potencial de la
 estructura aumenta en 3.
 \begin{align*}
  \hat{c}_{i+1} &= c_{i+1} + \Phi(D_{i+1}) - \Phi(D_i) \\
          &= 1 + (3 + \Phi(D_i)) - \Phi(D_i) \\
          &= 1 + 3 \\
          &= 4
 \end{align*}

 \item \texttt{mudar($Q.S_1$, $Q.S_2$)}:
 Si $Q.S_1$ tiene $k$ elementos, entonces realizamos $k$ operaciones de
 desempilar sobre $Q.S_1$ y $k$ operaciones de empilar sobre $Q.S_2$. Eliminar
 $k$ elementos de $Q.S_1$ hace que el potencial disminuya en $3*k$ y agregarlos
 a $Q.S_2$ hace que aumente en $k$. Entonces,
 \begin{align*}
  \hat{c}_{i+1} &= c_{i+1} + \Phi(D_{i+1}) - \Phi(D_i) \\
                &= k + k + \Phi(D_{i+1}) - \Phi(D_i) \\
                &= 2*k + (k - 3*k + \Phi(D_i)) - \Phi(D_i) \\
                &= 0
 \end{align*}

 \item \texttt{desencolar($Q$)}: Si $Q.S_2$ no tiene elementos, realizamos la
 la operación \texttt{mudar} --cuyo costo amortizado es $0$-- y estamos en el
 caso en que sí tiene. Cuando $Q.S_2$ tiene al menos un elemento, realizamos una
 operación de \texttt{desempilar} y el potencial de la estructura disminuye en
 $1$.
 \begin{align*}
  \hat{c}_{i+1} &= c_{i+1} + \Phi(D_{i+1}) - \Phi(D_i) \\
                &= 1 + (\Phi(D_i) -1) - \Phi(D_i) \\
                &= 0
 \end{align*}

\end{itemize}

En conclusión, tenemos una estructura de datos cola implementada con dos pilas
y cuyas operaciones tienen costo amortizado $O(1)$.

\end{document}
