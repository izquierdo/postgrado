\documentclass{article}

\usepackage{algorithm}
\usepackage{algorithmic}
\usepackage{amsmath}
\usepackage{amssymb}
\usepackage[utf8]{inputenc}
\usepackage{listings}
\usepackage[spanish]{babel}

% Para codigo bonito
\newcommand{\singlespace}{\renewcommand{\baselinestretch}{1.0}}

\newcommand{\algoritmo}[2]
{
{
  \singlespace
  \begin{algorithm}[h]
    \caption{#1}
    \begin{algorithmic}[1]
      #2
    \end{algorithmic}
  \end{algorithm}
}
}

\newcommand{\theHalgorithm}{\arabic{algorithm}}

\lstset{mathescape=true}

\begin{document}

% Título

\title{Tarea de recuperación de Teoría de Algoritmos}
\author{Daniel Izquierdo \\ \#08-86809}
\date{23 de octubre de 2008}

\maketitle

% Pregunta 1
%\section{}

Para demostrar que $(C,\Gamma)$ es una matroide basta demostrar que

\begin{itemize}
 \item $C$ es un conjunto finito.
 \item $\Gamma$ es un conjunto de subconjuntos de C hereditario. Es decir,
 
 $$
 (x \in \Gamma \Rightarrow x \subseteq C) \wedge (B \in \Gamma \wedge A \subseteq B \Rightarrow A \in \Gamma)
 $$

 \item Si $A \in \Gamma$, $B \in \Gamma$ y $|A| < |B|$, entonces existe $x \in B - A$ tal que
 $A \bigcup \{x\} \in \Gamma$.
\end{itemize}

Las dos primeras condiciones siguen de la definición de una estructura combinatoria
vista en clases. %Para demostrar que la tercera condición se cumple, voy a suponer al
%absurdo que no se cumple y llegar a una contradicción.


Supongamos al absurdo que la tercera condición no se cumple.
Si se cumple lo descrito en el enunciado, entonces si tenemos una función de costos $c$ tal que

$$
x \in C \Rightarrow c(x) = -1
$$

tendremos al final del algoritmo que $F$ es el conjunto de menor costo en $\Gamma$.
Sean $A, B$ conjuntos tales que $A \in \Gamma$, $B \in \Gamma$ y $|A| < |B|$.
$A \bigcup B$ debe pertenecer
a $\Gamma$ porque $F$ es el conjunto de menor costo en $\Gamma$ y, por lo tanto, contiene
los elementos de $A$ y $B$.
Todos los subconjuntos de $A \bigcup B$ también deben pertenecer a $\Gamma$. Llamemos $D$ al conjunto resultante de
quitar de $A \bigcup B$ todos los elementos de $B$ menos uno (se puede porque $|B| > |A| \geq 0$). Entonces $D$ es subconjunto
de $A \bigcup B$ y pertenece a $\Gamma$. Además, $D = A \bigcup \{x\}$ para algún $x \in B$.

Esto contradice la suposición inicial de que la tercera condición no se cumple. En conclusión, $(C,\Gamma)$ es una 
matroide.


%Como todos los elementos de C
%tienen el mismo costo negativo, la elección del menor elemento de $C^*$ en el algoritmo
%del enunciado es no
%determinística. Luego de $|A|$ iteraciones en el algoritmo, es posible tener que $F = A$
%ya que $A \in \Gamma$ y el algoritmo puede haber elegido únicamente elementos de A hasta ese
%momento. Como $c(A)>c(B)$, el algoritmo no puede haber terminado ya que $A$ no es el elemento
%de menor costo en $\Gamma$ (porque por lo menos $B$ tiene menor costo). Entonces va a agregar
%algún otro elemento de $C^*$.

\end{document}
