\documentclass{article}

\usepackage{algorithm}
\usepackage{algorithmic}
\usepackage{amsmath}
\usepackage{amssymb}
\usepackage[utf8]{inputenc}
\usepackage{listings}
\usepackage[spanish]{babel}
\usepackage{qtree}

% Para codigo bonito
\newcommand{\singlespace}{\renewcommand{\baselinestretch}{1.0}}

\newcommand{\algoritmo}[2]
{
{
  \singlespace
  \begin{algorithm}[h]
    \caption{#1}
    \begin{algorithmic}[1]
      #2
    \end{algorithmic}
  \end{algorithm}
}
}

\newcommand{\theHalgorithm}{\arabic{algorithm}}

\lstset{mathescape=true,frame=single}

\newcommand{\asig}{\ensuremath{\leftarrow}}
\renewcommand{\lstlistingname}{Listado} %listings en español

\begin{document}

% Título

\title{Tarea 3 de Teoría de Algoritmos}
\author{Daniel Izquierdo \\ \#08-86809}
\date{2 de diciembre de 2008}

\maketitle

\renewcommand{\labelenumi}{\alph{enumi})}

% Pregunta 1
\section{}

\begin{enumerate}

 \item

Los vértices del grafo son subconjuntos de $A$, representados por una fila
de unos y ceros. El vértice objetivo es $A$. La búsqueda comienza a partir del vértice
que representa el conjunto vacío. Los sucesores de un vértice que representa el conjunto
$V$ son todos los $V \bigcup U$ tales que

$$
U \neq \emptyset \wedge U \in F \wedge V \bigcap U = \emptyset \text{ }(*)
$$

Este grafo no tiene ciclos. La explicación informal del por qué es que un nodo sólo tiene
como sucesores conjuntos con cardinalidad estrictamente mayor a la suya.

Si en un momento del backtracking consigo un nodo que ya visité anteriormente, entonces no hay solución
por ese camino. Esto se debe al hecho de que seguí buscando luego de visitarlo, lo que implica que ni él ni
ninguno de sus sucesores es el nodo final. Por lo tanto, al llegar a un nodo visitado previamente podemos
ignorar su rama del árbol. Un criterio de poda puede ser, entonces, eliminar ramas previamente visitadas.
Este criterio sólo es razonable para instancias pequeñas, ya que necesitamos espacio exponencial para marcar
los nodos como visitados (puede haber hasta $2^n$ nodos distintos para un $A$ de cardinalidad $n$).

 \item

\renewcommand{\labelenumii}{\arabic{enumii}.}

% F={{c,e,f}, {a,d,g}, {b,c,f}, {a,d}, {b,a,g}, {b,g}}
\begin{enumerate}
\item Comienzo la búsqueda en el conjunto vacío $\emptyset$. \\
      Nodo actual: $\emptyset$

\item Elijo para añadir el primer elemento de $F$ elegible según $(*)$: \{c,e,f\}
      Nodo actual: $\{c, e, f\}$

\item Elijo para añadir el primer elemento de $F$ elegible según $(*)$: \{a,d,g\}
      Nodo actual: $\{c,e,f,a,d,g\}$

\item No quedan sucesores del nodo actual; me retracto de mi última decisión.
      Nodo actual: $\{c, e, f\}$

\item Elijo para añadir el segundo elemento de $F$ elegible según $(*)$: \{a,d\}
      Nodo actual: $\{c, e, f, a, d\}$

\item Elijo para añadir el primer elemento de $F$ elegible según $(*)$: \{b,g\}
      Nodo actual: $\{c, e, f, a, d, b, g\}$

\item Llegué al objetivo. Fin. El algoritmo responde ``sí''.
\end{enumerate}

\end{enumerate}

% Pregunta 2
\section{}

\begin{enumerate}

 \item

Tomemos $A$ como un arreglo (sin pérdida de generalidad). En el grafo implícito, un vértice es
una asignación parcial $(A_1, A_2)$: cada elemento en $A[0 \ldots k)$ para algún $k \geq 0$ a uno de los dos multiconjuntos.
Se tiene una asignación válida (aunque no necesariamente óptima) cuando se llega a un nodo en que $k = n$.
Los sucesores de un vértice no final con $k$ variables asignadas
son los resultantes de incluir $A[k]$ en $A_1$ o en $A_2$.

Supongamos que, para una instancia del problema, $(S, T)$ es una solución óptima
y $A[0] \in S$. Entonces $(T, S)$ también es una solución óptima, ya que

$$
\left|\sum_{e \in S} e - \sum_{e \in T} e\right| = \left|\sum_{e \in T} e - \sum_{e \in S} e\right|
$$

Por lo tanto, el multiconjunto en el cual se incluye el primer elemento del arreglo no cambia la solución.
Entonces se puede elegir como nodo inicial del grafo $(A_1 = \{A[0]\}, A_2 = \emptyset)$ y no explorar
la asignación de $A[0]$ a $A_2$.

 \item


 \item


\end{enumerate}

% Pregunta 3
\section{}

\end{document}
