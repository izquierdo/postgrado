\documentclass{article}

\usepackage{algorithm}
\usepackage{algorithmic}
\usepackage{amsmath}
\usepackage{amssymb}
\usepackage[utf8]{inputenc}
\usepackage{listings}
\usepackage[spanish]{babel}
\usepackage{qtree}

% Para codigo bonito
\newcommand{\singlespace}{\renewcommand{\baselinestretch}{1.0}}

\newcommand{\algoritmo}[2]
{
{
  \singlespace
  \begin{algorithm}[h]
    \caption{#1}
    \begin{algorithmic}[1]
      #2
    \end{algorithmic}
  \end{algorithm}
}
}

\newcommand{\theHalgorithm}{\arabic{algorithm}}

\lstset{mathescape=true,frame=single}

\newcommand{\asig}{\ensuremath{\leftarrow}}
\renewcommand{\lstlistingname}{Listado} %listings en español

\begin{document}
% Título

\title{Tarea 2 de Teoría de Algoritmos}
\author{Daniel Izquierdo \\ \#08-86809}
\date{6 de noviembre de 2008}

\maketitle

% Pregunta 1
\section{}

% Pregunta 2
\section{}

En ambos casos, el algoritmo
devuelve la matriz $partidos$ donde $partidos[d][c]$
es el competidor contra el cual
jugará el competidor $c$ en el día $d$. Si
$partidos[d][c] = 0$, el competidor $c$ no juega el día $d$. Supongo que
$matriz(n,m)$ devuelve una matriz de tamaño $n x m$ con todas las casillas
inicializadas en cero. También supongo que tengo los siguientes procedimientos disponibles
como ayuda:

\begin{lstlisting}[float,caption={Procedimientos auxiliares para el ejercicio 3},label=alg:aux-ejercicio3]
procedure horizontal(M1[1..n][1..m1]; M2[1..n][1..m2])
  resultado $\asig matriz(n,m1+m2)$

  for i $\asig 1$ to n do
    for j $\asig 1$ to $m1$ do
      resultado[i][j] $\asig M1[i][j]$
 
    for j $\asig 1$ to $m2$ do
      resultado[i][m1+j] $\asig M2[i][j]$

  return resultado

procedure vertical(M1[1..n1][1..m1]; M2[1..n2][1..m2])
  resultado $\asig matriz(n,m1+m2)$

  for j $\asig 1$ to $m$ do
    for i $\asig 1$ to $n1$ do
      resultado[i][j] $\asig M1[i][j]$
 
    for i $\asig 1$ to $n2$ do
      resultado[n1+i][j] $\asig M2[i][j]$

  return resultado
\end{lstlisting}

\begin{itemize}

\item Caso $n$ potencia de dos:

No considero los problemas de tamaño $1$ por no tener sentido un torneo con $1$ equipo y de $0$
días de duración. Para resolver el problema creo el procedimiento auxiliar $tabla$ donde
$tabla(n, inicio)$ es un torneo válido para los $n$ jugadores $inicio, inicio+1, ..., inicio+n-1$.
Entonces la solución del problema es igual a $tabla(n, 1)$.

El algoritmo divide un problema de tamaño $n$ en dos sub-problemas
de tamaño $n/2$, lo cual se puede hacer por ser $n$ potencia de dos. El caso base $n = 2$
se resuelve en un sólo día con un sólo partido donde los dos competidores juegan entre sí.
En el caso recursivo, tengo dos sub-problemas $T_1$ y $T_2$, donde $T_1$ es un torneo de $n/2-1$ días
de duración para los primeros $n/2$ jugadores y $T_2$ para los últimos $n/2$ jugadores. Tras resolver
recursivamente ambos sub-problemas debo unir los resultados para obtener la solución. Observo que ningún
jugador que aparece en $T_1$ aparece en $T_2$ y vice-versa. Por lo tanto puedo agregar $n/2-1$ juegos
asignando juegos entre cada jugador $x$ del primer sub-problema y los contrincantes asignados a $x+n/2$
en el segundo sub-problema. Eso se logra con la matriz

$$
\left[
\begin{array}{cc}
T_1 & T_2 \\
T_2 & T_1 \\
\end{array}
\right]
$$

De esta manera me queda un torneo con $n/2-2$ juegos. Se puede ver que sólo faltan los juegos
entre $x$ y $x+n/2$ con $x$ en $T_1$, ya que $x+n/2$ no aparece como contrincante de sí mismo
en $T_2$. Entonces puedo agregar en el último día del torneo todos los juegos con esa forma para
obtener un resultado válido de $n/2-1$ días. El algoritmo se puede observar en el listado
\ref{alg:ejercicio3}.

\begin{lstlisting}[float,caption={Solución al ejercicio 3},label=alg:ejercicio3]
procedure torneo(n)
  return tabla(n, 1)

procedure tabla(n, inicio)
  if n = 2 do
    partidos[1][inicio] $\asig inicio+1$
    partidos[1][inicio+1] $\asig inicio$
    return partidos

  m1 $\asig tabla(n/2, inicio)$
  m2 $\asig tabla(n/2, inicio+(n/2))$

  vi $\asig vertical(m1, m2)$
  vd $\asig vertical(m2, m1)$

  piso $\asig matriz(1,n)$

  for i $\asig 1$ to $n / 2$ do
    piso[1][i] $\asig i+(n/2)$
    piso[1][$i+(n/2)$] $\asig i$

  total $\asig horizontal(vi, vd)$

  return $vertical(total, piso)$
\end{lstlisting}

\item Caso $n > 1$:

\end{itemize}

% Pregunta 3
\section{}

\renewcommand{\labelenumi}{(\alph{enumi})}


\begin{enumerate}
 \item El costo promedio en comparaciones a realizar es

$$
0.1 + (0.25+0.3)*2 + (0.2+0.05+0.05+0.05)*3 = 2.25
$$

El costo del árbol en la figura 8.8 del libro es menor y, por lo tanto, mejor.

 \item

Para construir un arbol con un conjunto de elementos se elige el elemento con mayor probabilidad
de ser accedido y se coloca como raíz. Su sub-árbol izquierdo es el resultado de aplicar el algoritmo
sobre el subconjunto del conjunto original formado por los elementos menores a la raíz. El derecho es el
resultado
de aplicar el algoritmo sobre el subconjunto de elementos mayores a la raíz. En caso de que alguno
de los conjuntos sea vacío, entonces el resultado para ese sub-árbol es una hoja.

 \item

El siguiente árbol es más eficiente en promedio:

\Tree [.12 6 [.34 [.20 18 27 ] 35 ] ]

El costo promedio en comparaciones a realizar es

$$
0.25 + (0.2+0.3)*2 + (0.1+0.05)*3 + (0.05+0.05)*4 = 2.1
$$

Podemos ver que es menor que el costo promedio del árbol en la figura 8.8 del libro.
Esto refuerza el hecho de que es usual que un problema no pueda ser resuelto con un
algoritmo voraz (como el utilizado para construir dicho árbol).

\end{enumerate}

% Pregunta 4
\section{}

\end{document}
