\documentclass{article}

\usepackage{algorithm}
\usepackage{algorithmic}
\usepackage{amsmath}
\usepackage{amssymb}
\usepackage[utf8]{inputenc}
\usepackage{listings}
\usepackage[spanish]{babel}
\usepackage{qtree}

% Para codigo bonito
\newcommand{\singlespace}{\renewcommand{\baselinestretch}{1.0}}

\newcommand{\algoritmo}[2]
{
{
  \singlespace
  \begin{algorithm}[h]
    \caption{#1}
    \begin{algorithmic}[1]
      #2
    \end{algorithmic}
  \end{algorithm}
}
}

\newcommand{\theHalgorithm}{\arabic{algorithm}}

\lstset{mathescape=true,frame=single}

\newcommand{\asig}{\ensuremath{\leftarrow}}
\renewcommand{\lstlistingname}{Listado} %listings en español

\begin{document}
% Título

\title{Tarea 2 de Teoría de Algoritmos}
\author{Daniel Izquierdo \\ \#08-86809}
\date{6 de noviembre de 2008}

\maketitle

% Pregunta 1
\section{}

% Pregunta 2
\section{}

En ambos casos, el algoritmo
devuelve la matriz $partidos$ donde $partidos[d][c]$
es el competidor contra el cual
jugará el competidor $c$ en el día $d$. Si
$partidos[d][c] = 0$, el competidor $c$ no juega el día $d$. Supongo que
$matriz(n,m)$ devuelve una matriz de tamaño $n x m$ con todas las casillas
inicializadas en cero. También supongo que tengo los siguientes procedimientos disponibles
como ayuda:

\begin{lstlisting}[float,caption={Procedimientos auxiliares para el ejercicio 3},label=alg:aux-ejercicio3]
procedure horizontal(M1[1..n][1..m1]; M2[1..n][1..m2])
  resultado $\asig matriz(n,m1+m2)$

  for i $\asig 1$ to n do
    for j $\asig 1$ to $m1$ do
      resultado[i][j] $\asig M1[i][j]$
 
    for j $\asig 1$ to $m2$ do
      resultado[i][m1+j] $\asig M2[i][j]$

  return resultado

procedure vertical(M1[1..n1][1..m1]; M2[1..n2][1..m2])
  resultado $\asig matriz(n,m1+m2)$

  for j $\asig 1$ to $m$ do
    for i $\asig 1$ to $n1$ do
      resultado[i][j] $\asig M1[i][j]$
 
    for i $\asig 1$ to $n2$ do
      resultado[n1+i][j] $\asig M2[i][j]$

  return resultado
\end{lstlisting}

\begin{itemize}

\item Caso $n$ potencia de dos:

No considero los problemas de tamaño $1$ por no tener sentido un torneo con $1$ equipo y de $0$
días de duración. Para resolver el problema creo el procedimiento auxiliar $tabla$ donde
$tabla(n, inicio)$ es un torneo válido para los $n$ jugadores $inicio, inicio+1, ..., inicio+n-1$.
Entonces la solución del problema es igual a $tabla(n, 1)$.

El algoritmo divide un problema de tamaño $n$ en dos sub-problemas
de tamaño $n/2$, lo cual se puede hacer por ser $n$ potencia de dos. El caso base $n = 2$
se resuelve en un sólo día con un sólo partido donde los dos competidores juegan entre sí.
En el caso recursivo, tengo dos sub-problemas $T_1$ y $T_2$, donde $T_1$ es un torneo de $n/2-1$ días
de duración para los primeros $n/2$ jugadores y $T_2$ para los últimos $n/2$ jugadores. Tras resolver
recursivamente ambos sub-problemas debo unir los resultados para obtener la solución. Observo que ningún
jugador que aparece en $T_1$ aparece en $T_2$ y vice-versa. Por lo tanto puedo agregar $n/2-1$ juegos
asignando juegos entre cada jugador $x$ del primer sub-problema y los contrincantes asignados a $x+n/2$
en el segundo sub-problema. Eso se logra con la matriz

$$
\left[
\begin{array}{cc}
T_1 & T_2 \\
T_2 & T_1 \\
\end{array}
\right]
$$

De esta manera me queda un torneo con $n/2-2$ juegos. Se puede ver que sólo faltan los juegos
entre $x$ y $x+n/2$ con $x$ en $T_1$, ya que $x+n/2$ no aparece como contrincante de sí mismo
en $T_2$. Entonces puedo agregar en el último día del torneo todos los juegos con esa forma para
obtener un resultado válido de $n/2-1$ días. El algoritmo se puede observar en el listado
\ref{alg:ejercicio3}.

\begin{lstlisting}[float,caption={Solución al ejercicio 3},label=alg:ejercicio3]
procedure torneo(n)
  return tabla(n, 1)

procedure tabla(n, inicio)
  if n = 2 do
    partidos[1][inicio] $\asig inicio+1$
    partidos[1][inicio+1] $\asig inicio$
    return partidos

  m1 $\asig tabla(n/2, inicio)$
  m2 $\asig tabla(n/2, inicio+(n/2))$

  vi $\asig vertical(m1, m2)$
  vd $\asig vertical(m2, m1)$

  piso $\asig matriz(1,n)$

  for i $\asig 1$ to $n / 2$ do
    piso[1][i] $\asig i+(n/2)$
    piso[1][$i+(n/2)$] $\asig i$

  total $\asig horizontal(vi, vd)$

  return $vertical(total, piso)$
\end{lstlisting}

\item Caso $n > 1$:

\end{itemize}

% Pregunta 3
\section{}

\renewcommand{\labelenumi}{(\alph{enumi})}
\begin{enumerate}
 \item El costo promedio en comparaciones a realizar es

$$
0.1 + (0.25+0.3)*2 + (0.2+0.05+0.05+0.05)*3 = 2.25
$$

El costo del árbol en la figura 8.8 del libro es menor y, por lo tanto, mejor.

 \item

Para construir un arbol con un conjunto de elementos se elige el elemento con mayor probabilidad
de ser accedido y se coloca como raíz. Su sub-árbol izquierdo es el resultado de aplicar el algoritmo
sobre el subconjunto del conjunto original formado por los elementos menores a la raíz. El derecho es el
resultado
de aplicar el algoritmo sobre el subconjunto de elementos mayores a la raíz. En caso de que alguno
de los conjuntos sea vacío, entonces el resultado para ese sub-árbol es una hoja.

 \item

El siguiente árbol es más eficiente en promedio:

\Tree [.12 6 [.34 [.20 18 27 ] 35 ] ]

El costo promedio en comparaciones a realizar es

$$
0.25 + (0.2+0.3)*2 + (0.1+0.05)*3 + (0.05+0.05)*4 = 2.1
$$

Podemos ver que es menor que el costo promedio del árbol en la figura 8.8 del libro.
Esto refuerza el hecho de que es usual que un problema no pueda ser resuelto con un
algoritmo voraz (como el utilizado para construir dicho árbol).

\end{enumerate}

% Pregunta 4
\section{}

Supongo que --como aparece en el libro-- la secuencia de entrada está ordenada.
Voy a usar la definición de $C_{i,j}$ dada en el libro con la nueva condición de que
$C_{i,j} = 0$ si $j < i$ ya que un árbol sin elementos tiene como tiempo promedio de
búsqueda $0$. Supongamos que tenemos una solución $X$
óptima para la secuencia de elementos entre $i$ y $j$. Sea $k$ el índice del elemento
raíz de esa solución. El sub-árbol izquierdo contiene los elementos menores que el elemento
$k$, es decir, los elementos con índices entre $i$ y $k-1$ inclusive. El sub-árbol derecho contiene
los elementos mayores que el elemento $k$, es decir, los elementos con índices entre $k+1$ y $j$
inclusive. Sean $K_i$ y $K_d$ los costos de los sub-árboles izquierdo y
derecho de la solución respectivamente. El costo de la solución $X$ es

$$
K_i + K_d + \sum_{k=i}^j p_k
$$

ya que debemos considerar el costo de ambos sub-árboles más el costo de acceder al
elemento raíz más el costo añadido de incrementar la profundidad de cada nodo en
cualquiera de los sub-árboles en uno (ya que estamos agregando la raíz a altura mayor).
Si tenemos que $K_i \neq C_{i,k-1}$, entonces una solución mejor consiste en reemplazar
el sub-árbol $K_i$ por uno tal que su costo sea igual a $C_{i,k-1}$, el cual es el costo
mínimo para un sub-árbol con los elementos entre $i$ y $k-1$ inclusive. Esto contradice
la suposición de que $X$ es una solución óptima.

Por lo tanto tenemos obligatoriamente que el costo de $K_i$ es igual a $C_{i,k-1}$.
Ocurre de manera análoga con el sub-árbol derecho y un árbol de costo $C_{k+1,j}$.
Entonces el costo de la solución $X$ debe ser 

$$
C_{i,k-1} + C_{k+1,j} + \sum_{k=i}^j p_k
$$

Ahora, supongamos al absurdo que existe $m \neq k$ tal que $i \leq m \leq j$ y
$C_{i,m-1}+C_{m+1,j} < C_{i,k-1}+C_{k+1,j}$. Entonces el árbol que contiene los
elementos entre $i$ y $j$ inclusive y tiene como raíz el elemento $m$ tiene menor
costo, ya que 

$$
C_{i,m-1} + C_{m+1,j} + \sum_{k=i}^j p_k < C_{i,k-1} + C_{k+1,j} + \sum_{k=i}^j p_k
$$

Esto contradice el hecho de que $X$ sea una solución óptima. Con esto, $k$ debe ser tal que
no exista índice $m$ como el descrito. Así, he probado con el principio de optimalidad
que

$$
C_{i,j} = (\text{{\bf min} } k : i \leq k \leq j : C_{i,k-1} + C_{k+1,j}) + \sum_{k=i}^j p_k
$$

Para obtener el algoritmo de programación dinámica solicitado,
defino las tablas $S$ y $C$, ambas de tamaño $n$. Usaré la tabla $S$ para pre-calcular las sumatorias

$$
\sum_{k=i}^j p_k
$$

para todo $i < j$. Hago esto ya que necesito esos valores para resolver el problema. La tabla $C$ contendrá
los valores $C_{i,j}$.

Para obtener la tabla $S$ utilizo el siguiente algoritmo

\begin{lstlisting}[caption={Cálculo de la tabla S},label=alg:tablaS]
for i $\asig 1$ to $n$ do
    suma $\asig p_i$

    for j $\asig i+1$ to $n$ do
        suma $\asig suma + p_j$
        S[i][j] $\asig suma$
\end{lstlisting}

En el algoritmo encontramos dos ciclos, uno dentro de otro, ambos realizando $O(n)$ operaciones. Entonces
el costo en tiempo del algoritmo es $O(n^2)$. Hecho esto utilizaré programación dinámica para calcular
$C_{1,n}$. Recordemos que los casos para $C$ son

\begin{equation*}
C_{i,j} = 
\begin{cases}
p_i & \text{si $i=j$,} \\
0 &\text{si $i>j$,} \\
C_{i,j} = (\text{{\bf min} } k : i \leq k \leq j : C_{i,k-1} + C_{k+1,j}) + \sum_{k=i}^j p_k &\text{si $i<j$.} \\
\end{cases}
\end{equation*}

Para empezar relleno en la tabla los casos base ($i \geq j$):

\begin{lstlisting}[caption={Inicialización de casos base de la tabla C},label=alg:tablaC_base]
for i $\asig 1$ to $n$ do
    S[i][i] = p_i

for i $asig 1$ to $n$ do
    for j $asig 1$ to $i-1$ do
        S[i][j]=0
\end{lstlisting}

Nos encontramos otra vez con un algoritmo de tiempo $O(n^2)$. Una vez resueltos los
casos base intento llenar el resto de la tabla. Hay que observar que el valor de $C_{i,j}$
depende únicamente de $S_{i,j}$ (ya calculado) y los $C_{x,y}$ tales que $x=i$ y $y<j$ o
$y=j$ y $x<i$. En otras palabras, depende de los elementos que están a la izquierda en su
fila y de los que están abajo en su columna. Puedo calcular los valores de la diagonal inmediatamente
a la derecha de la diagonal recién calculada,


\begin{lstlisting}[caption={Cálculo de la tabla C},label=alg:tablaC]

\end{lstlisting}

\begin{lstlisting}[caption={Cálculo del árbol},label=alg:arbol]
for i $\asig 1$ to $n$ do
    for j $\asig i+1$ to $n$ do
        suma $\asig 0$

        for k $\asig i$ to $j$ do
            suma $\asig suma + p_k$

        S[i][j] $\asig suma$
\end{lstlisting}

\end{document}
